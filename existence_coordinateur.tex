
\documentclass{article}
\usepackage{amsmath}
\usepackage{algpseudocode}
\usepackage{amssymb}
\usepackage{amsthm}
\usepackage{dsfont}
\usepackage[left=2cm, right=2cm, top=2cm]{geometry}
\usepackage[utf8]{inputenc}
\newtheorem{lemma}{Théorème}

\title{Établissement d'une condition nécessaire pour la résolubilité du consensus avec départs asynchrones}
\date{7 mai 2020}
\author{Louis Penet de Monterno}

\begin{document}
\maketitle

\begin{lemma}

Si une machine $(A, \mathcal{P})$ résout le consensus, on a
$\mathcal{P} \Rightarrow \mathcal{P}_{coord}$ où $\mathcal{P}_{coord} = \exists p \in \Pi,
\forall q \in \Pi, \forall n \in \mathds{N}, \exists k \in \mathds{N},
\exists (p_0 = p, p_1, ... p_k = q), \exists (r_0, r_1, ... r_k), r_0 \geq n
\wedge \forall i \in \{1, ..., k\}, p_i \in HO(r_i, p_{i-1}) \wedge r_i > r_{i-1}$

On introduit la notation $path(p, q, n)$ où
$\exists p \in \Pi, \forall q \in \Pi, \forall n \in \mathds{N}, path(p, q, n)$ 

Intuitivement, il existe un coordianteur, dont les messages peuvent toujours parvenir à tous les processus.

\end{lemma}

\begin{proof}

On suppose par l'absurde que $(A, \mathcal{P})$ résout le consensus, et $\mathcal{P}$ ne satisfait pas $\mathcal{P}_{coord}$.

Ainsi, $\forall p \in \Pi, \exists q \in \Pi, \exists n \in \mathds{N}, \neg path(p, q, n)$ 

\vspace{0.5cm}

Soit $p_0$ un processus quelquonque. On obtient ainsi $n_0$ et $p_1$ tel que $\neg path(p_0, p_1, n_0)$. On a de plus $p_0 \neq p_1$ à cause de la "self-loop".

De la même manière, on obtient $p_2, p_3, ...$ et $n_1, n_2, ...$ tels que $\neg path(p_i, p_{i+1}, n_i)$. 

Comme les processus sont en nombre fini, la suite des $(p_i)_{i \in \mathds{N}}$ contient un processus plusieurs fois.

On obtient ainsi un cycle $(p_i, p_{i+1}, ... p_{i+h}, p_{i+h+1} =  p_i)$,
vérifiant $\forall, i \leq k \leq i+h, \neg path(p_k, p_{k+1}, n_k)$.

On pose $M = max \{ n_i, ..., n_h \}$.

\vspace{0.5cm}

On montre maintenant par l'absurde la proposition suivante :
$\exists n \in \mathds{N}, \exists (p, q) \in \Pi^2, \neg path(p,q, n) \wedge \neg path(q,p,n)$

On suppose donc que 
$\forall n \in \mathds{N}, \forall (p, q) \in \Pi^2, path(p,q, n) \wedge path(q,p,n)$.

Intuitivement, la contradiction vient du fait que cette proposition permet de construire un chemin
entre $p_i$ et $p_{i+1}$ en suivant le cycle construit ci-dessus, dans le sens inverse.

Sachant $\neg path(p_{i+h}, p_i, n_{i+h})$, on obtient $\neg path(p_{i+h}, p_i, M)$, donc
$path(p_i, p_{i+h}, M)$.

Ce chemin se construit entre le round M et un certain round maximal $n_{i+h}' > M$.

\vspace{0.2cm}

Sachant $\neg path(p_{i+h-1}, p_{i+h}, n_{i+h-1})$,
on obtient $\neg path(p_{i+h-1}, p_{i+h}, n_{i+h}')$,
donc $path(p_{i+h}, p_{i+h-1}, n_{i+h}')$.

Ce chemin se construit entre le round M et un certain round maximal $n_{i+h-1}' > n_{i+h}'$.

\vspace{0.2cm}

Sachant $\neg path(p_{i+h-2}, p_{i+h-1}, n_{i+h-2})$,
on obtient $\neg path(p_{i+h-2}, p_{i+h-1}, n_{i+h-1}')$,
donc $path(p_{i+h-1}, p_{i+h-2}, n_{i+h-1}')$.

Ce chemin se construit entre le round M et un certain round maximal $n_{i+h-2}' > n_{i+h-1}'$.

\vspace{0.2cm}

On construit ainsi 
un chemin $path(p_{i}, p_{i+h}, M)$ s'étendant jusqu'au round $n_{i+h}'$,
un chemin $path(p_{i+h}, p_{i+h-1}, n_{i+h}')$ s'étendant jusqu'au round $n_{i+h-1}'$,
un chemin $path(p_{i+h-1}, p_{i+h-2}, n_{i+h-1}')$ s'étendant jusqu'au round $n_{i+h-2}'$,
etc, jusqu'au chemin $path(p_{i+2}, p_{i+1}, n_{i+1}')$ s'étendant jusqu'au round $n_{i}'$.

Cela forme ainsi un chemin du processus $p_i$ jusqu'au processus $p_{i+1}$, ce qui contredit 
une construction précédente.

\vspace{0.5cm}

La proposition 
$\exists n \in \mathds{N}, \exists (p, q) \in \Pi^2, \neg path(p,q, n) \wedge \neg path(q,p,n)$
est donc démontrée.

Or cela rend impossible toute forme de consensus. En effet, en supposant que de tels processus p et q ne se réveillent qu'à partir du round n. Ils n'ont aucun moyen de décider en assurant l'accord.

Cela prouve donc le théorème

\end{proof}
\end{document}
