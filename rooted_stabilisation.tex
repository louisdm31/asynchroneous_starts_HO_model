\documentclass[11pt,letterpaper]{article}
\special{papersize=8.5in,11in}

\usepackage{fullpage}
\usepackage{amsmath,amssymb,amsthm,dsfont}
\newtheorem{thm}{Theorem}
\newtheorem{pro}[thm]{Proposition}
\newtheorem{lem}[thm]{Lemma}
\newtheorem{cor}[thm]{Corollary}
\newtheorem{defi}[thm]{Definition}

\title{Stabilisation in rooted communication graph}
\author{Bernadette Charron-Bost\textsuperscript{1} \and Louis Penet de Monterno\textsuperscript{1}}
\date{\textsuperscript{1} \'Ecole polytechnique, 91128 Palaiseau, France\\~\\ \today}

\begin{document}

\maketitle

Let $source_G(i)$ the set of nodes from which $i$ is accessible.
Formally, $u \in source_G(i) \Rightarrow (u,i) \in \bar V$ where $\bar V$ is the set of edges of the transitive closure of $G$.
Let $G[i]$ be the sub-graph of $G$ induced by $source_G(i)$.

\begin{lem} \label{lem:viewconst}
	For any node $u$, from round $t > D_u$, we have
	$\exists c \in \mathds{N}, T_u^t = \widetilde{G[u]}^u \uparrow t+c$.
\end{lem}

\begin{lem} \label{lem:safB} 
	Let $G = (N_G, A_G)$ be a graph and $i \in N_G$.
	Let $B$ be a fibration-prime graph.
	If $\exists j \in N_G, h(\tilde{G}^i \wedge \tilde{B}^j) \geq n_G + D_i$, then $\widehat{G[i]} = B$.
\end{lem}

\begin{lem} \label{lem:getid}
	If $\mathcal{B}(T)$ is defined, there exist a unique node $i$ from $\mathcal{B}(T)$ such that $T \leq \widetilde{\mathcal{B}(T)}^i$.
\end{lem}

\begin{lem} \label{lem:commuthat} 
	Any graph $G$ verifies $\forall u \in N_G, \widehat{G[u]} \backsimeq \hat G[\mu_G(u)]$.
\end{lem}
\begin{proof}
	Let $u$ be a node from $N_G$.
	We construct an isomorphism $\phi : \widehat{G[u]} \rightarrow \hat G[\mu_G(u)]$.

	Let $j \in \widehat{G[u]}$.
	This node is a fiber containing several nodes from $G[u]$.
	Let $j' \in G[u]$ be one of them.
	Then we can construct a node $\bar j = \mu_G(j')$ belonging to $\hat G$.

	Since $j' \in G[u]$, there exists a path in $G$ from $j'$ to $u$.
	By lifting this path to $\hat G$, we obtain a path from $\bar j$ to $\mu_G(u)$.
	Thus $\bar j \in \hat G[\mu_G(u)]$, we can set $\phi(j) = \bar j$.

	We now need to prove that $\phi$ is an isomorphism.

	\noindent \textbf{Injectivity: } Let $j_1$ and $j_2$ be two nodes in $\widehat{G[u]}$ verifying $\phi(j_1) = \phi(j_2)$.
	This means that $j_1' \in G[u]$ and $j_2' \in G[u]$ both belong to the same fiber in $\hat G$.
	They must also belong to the same fiber in $\widehat{G[u]}$ because the fibration $\mu_{G[u]}$ is minimal.
	That proves $j_1 = j_2$.

	\noindent \textbf{Surjectivity: } Let $\bar j$ be a node in $\hat G[\mu_G(u)]$.
	We know that there exists a path $a$ from $\bar j$ to $\mu_G(u)$.
	The lifted path $\tilde{a}^u$ must be a path from some node $j'$ verifying $\mu_G(j') = \bar j$ to $u$.
	This node $j'$ must belong to $G[u]$.
	Now, we claim that $\phi(\mu_{G[u]}(j')) = \bar j$.
	Since any node $j''$ from $\mu_{G[u]}(j')$ and $j'$ must belong to the same fibre in $\widehat{G[u]}$, 
	they must also belong to the same fiber in $\hat G$ because the fibration $\mu_G$ is minimal.
	That justifies the claim, and proves surjectivity.

	\noindent \textbf{Definition over edges:} We want $\phi$ to be an isomorphism. We must map each edge from $\widehat{G[u]}$ to an edge from $\hat G[\mu_G(u)]$.
	This is possible if for any $v$ and $v'$ in $G[u]$, the multiplicity of edges between $v$ and $v'$ in $\widehat{G[u]}$
	is equal to the multiplicity between $\phi(v)$ and $\phi(v')$ in $\hat G[\mu_G(u)]$.
	This results from the fact that $\mu_G$ and $\mu_{G[u]}$ are both fibrations.

\end{proof}

\begin{thm}
	Let $\mathcal{C}$ be a network class containing rooted graphs.
	Let $S$ be a behaviour.
	The following two conditions are equivalent:
	\begin{itemize}
		\item There exists a protocol $P$ which stabilizes on $S$.
		\item There exists a function $y$ which map any fibration-prime $B$ to a sequence $y_B^0, y_B^1, \dots \in (Y^{N_B})^\omega$.
			This function must verify $\forall G \in \mathcal{C}, \exists T \in \mathds{N}, (y_{\hat G}^0)^{\mu_G}, (y_{\hat G}^1)^{\mu_G}, \dots \in S_G$
			and $\forall G \in \mathcal{C}, \forall u \in N_{G[u]}, \forall i \in \mathds{N}, proj_{\hat G[u]}(y^i_{\hat G}) = y^i_{\hat G[u]}$.
	\end{itemize}
\end{thm}
\begin{proof}
	First, we prove the direct implication. We assume the existence of a protocol $P$ stabilizing on $S$.
	We obtain the sequence $y_B$ by executing $P$ on any fibration-prime graph $B$.
	Let $G$ be a graph from $\mathcal{C}$.
	The lifting lemma implies that $(y_{\hat G}^0)^{\mu_G}, (y_{\hat G}^1)^{\mu_G}, \dots$ is an execution on $G$.
	By definition of stabilization, we obtain $\exists T \in \mathds{N}, (y_{\hat G}^0)^{\mu_G}, (y_{\hat G}^1)^{\mu_G}, \dots \in S_G$.
	Moreover, a node $u$ cannot locally distinguish $G$ and $G[u]$ (then none of its predecessor can). This remark justifies the proposition
	$\forall G \in \mathcal{C}, \forall u \leq N_G, \forall i \in \mathds{N}, proj_{\hat G[u]}(y^i_{\hat G}) = y^i_{\hat G[u]}$.

	Now, we prove the other implication. We construct a protocol which can stabilize on $S$:
	$$T_i^{t+1} = U_i^t$$
	$$B_i^{t+1} = \mathcal{B}(T_i^{t+1})$$
	$$E_i^{t+1} = \epsilon_j T_i^{t+1} \leq B_i^{t+1}$$
	$$h_i^{t+1} = todo$$
	$$x_i^{t+1} = y_{B_i^{t+1}}^{h_i^{t+1}}(E_i^{t+1})$$

	According to lemma \ref{lem:viewconst}, the value of $T_i^t$ eventually verifies $T_i^t = \tilde{G}_i \uparrow t+c$.
	According to lemma \ref{lem:safB}, we eventually have $\forall u \in N_G, B_u^t = \widehat{G[u]}$.
	According to lemma \ref{lem:getid}, we ultimately have $E_i^t = \mu_G(i)$.
	Moreover, we know that $\forall u \in N_{\hat G}, proj_{G[u]}(y^t_{\hat G}) = y^t_{\hat G[u]}$.
	We ultimately get $\forall u \in N_G, x_u^t = y_{\widehat{G[u]}}^{h_i^t}(E_i^t) = y_{\hat G[\mu_G(u)]}^{h_i^t}(\mu_G(u)) = y^{h_i^t}_{\hat G}(\mu_G(u))$.
\end{proof}

\end{document}
