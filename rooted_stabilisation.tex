\documentclass[11pt,letterpaper]{article}
\special{papersize=8.5in,11in}

\usepackage{fullpage}
\usepackage{amsmath,amssymb,amsthm,dsfont}
\newtheorem{thm}{Theorem}
\newtheorem{pro}[thm]{Proposition}
\newtheorem{lem}[thm]{Lemma}
\newtheorem{cor}[thm]{Corollary}
\newtheorem{defi}[thm]{Definition}

\title{Stabilisation in rooted communication graph}
\author{Bernadette Charron-Bost\textsuperscript{1} \and Louis Penet de Monterno\textsuperscript{1}}
\date{\textsuperscript{1} \'Ecole polytechnique, 91128 Palaiseau, France\\~\\ \today}

\begin{document}

\maketitle

Given a graph $G = (N_G, A_G)$, we define $roots(G)$ as the set of roots of $G$.
We also define $G_S$ as the restriction of $G$ over a set of nodes $S$ (i.e. $G_S = (S, A_G \cap (S \times S))$).
We define $source(G)$ as $G_{roots(G)}$.


\begin{lem} \label{lem:safB} 
	Let $G = (N_G, A_G)$ be a graph and $i$ a node reachable from any node of $G$.
	Let $B$ be a fibration-prime graph.
	If $\exists j \in N_G, h(\tilde{G}^i \wedge \tilde{B}^j) \geq n_G + D_G$, then $\hat G = B$.
\end{lem}

\begin{thm}
	For any rooted graph $G$, for any node $i$, for any $k \geq n_G + D_G$ we have $source(\mathcal{B}(\tilde{G}^i \uparrow k)) \backsimeq \widehat{source(G)}$
\end{thm}
\begin{proof}
	First, using lemma \ref{lem:safB}, we can show that $\mathcal{B}(\tilde{G}^i \uparrow k) = \widehat{G_{S(i)}}$ where $S(i)$ is the set of nodes from which $i$ is reachable.
	We now construct an isomorphism $\phi : \widehat{source(G)} \rightarrow source(\mathcal{B}(\tilde{G}^i \uparrow k))$.

	Let $j \in \widehat{source(G)}$.
	This node is a fiber containing several nodes from $roots(G)$.
	Let $j' \in roots(G)$ be one of them. This node belongs to $S(i)$.
	Thus we can construct a node $\bar j = \mu_{G_{S(i)}}(j')$ belonging to $\widehat{G_S(i)}$.

	Now we show that $\bar j$ belongs to $roots(\mathcal{B}(\tilde{G}^i \uparrow k)) = roots(\widehat{G_S(i)})$
	To do so, we  choose an arbitrary node $l \in G_{S(i)}$.
	Since $j' \in roots(G)$, there exists a path in $G$ from $j'$ to $l$.
	By lifting this path to $\hat G$, we obtain a path from $\bar j$ to any node of $\widehat{G_{S(i)}}$.
	Thus $\bar j \in roots(\widehat{G_{S(i)}})$, we can set $\phi(j) = \bar j$.

	We now need to prove that $\phi$ is an isomorphism.

	\noindent \textbf{Injectivity: } Let $j_1$ and $j_2$ be two nodes in $\widehat{source(G)}$ verifying $\phi(j_1) = \phi(j_2)$.
	This means that $j_1' \in roots(G)$ and $j_2' \in roots(G)$ both belong to the same fiber in $\widehat{G_S(i)}$.
	They must also belong to the same fiber in $\widehat{source(G)}$ because the fibration $\mu_{source(G)}$ is minimal.

	\noindent \textbf{Surjectivity: } Let $j$ be a node in $source(\widehat{G_S(i)})$.
	This node is a fiber containing at least one node from $G_{S(i)}$.

\end{proof}


\end{document}
