\documentclass{article}
\usepackage{amsmath}
\usepackage[noend]{sources_Heard-Of/distribalgo}
\usepackage{algorithm}
\usepackage{amssymb}
\usepackage{amsthm}
\usepackage{dsfont}
\usepackage{stmaryrd}
\usepackage[left=2cm, right=2cm, top=2cm]{geometry}
\usepackage[utf8]{inputenc}
\newtheorem{lemma}{Lemme}
\newtheorem{theorem}{Théorème}
\newtheorem{definition}{Définition}
\setlength{\parskip}{0.22cm}
%\newcommand{\st0}{q^0}


\title{Algorithme de synchronisation modulo k}
\date{10 juillet 2020}
\author{Louis Penet de Monterno}

\begin{document}

\maketitle

\section{Problème}

Il est courant de rencontrer des algorithmes de consensus, structurés par phases : 
à chaque rounds pairs, les processus exécutent une certaine fonction de transition. Et au round impairs, une autre fonction de transition est exécutée.
Lorsque les départs sont synchrones, tous les processus exécutent la même fonction de transition à chaque round.
Lorsqu'il y a des départs asynchrones, les phases paires des différents processus ne sont plus simultanées, et cela peut compromettre la sûreté de l'algorithme.
Le but de ce document est de proposer un algorithme permettant de synchroniser les différentes phases malgré les départs asynchrones.

\begin{definition}

	Soit un paramètre $k > 1$. Soit $s_{exit} \in States_p$ un état final.
	Étant donnée une exécution d'un algorithme $A$, pour tout $p \in \Pi$, on note $t_p$ le round auquel $p$ atteint $s_{exit}$ ($t_p = \infty$ sinon).
	Cette exécution de l'algorithme $A$ est sûr vis à vis du problème de synchronisation modulo $k$ si,
	$$\exists c \in \mathds{N}, \forall p \in \bigcup\limits_{i \in \mathds{N}} \mathcal{A}_i, t_p \neq \infty \Rightarrow t_p~mod~k = c$$

\end{definition}

\begin{definition}

	Une exécution de l'algorithme $A$ vérifie la vivacité vis à vis du problème de synchronisation modulo $k$ si,
	$$\forall p \in \bigcup\limits_{i \in \mathds{N}} \mathcal{A}_i, t_p \neq \infty$$

\end{definition}
\begin{definition}
	L'algorithme $A$ résoud le problème de synchronisation modulo $k$, si toute exécution de $A$ est sûre et termine.
\end{definition}

Pour la suite, on considère une valeur $k > 2$ quelquonque.

\section{Algorithme}

L'objectif est de résoudre ce problème dans les conditions suivantes :
\begin{itemize}

	\item Le graphe dynamique contient une étoile fixe : $\mathcal{P}_{\xi-nek} \equiv \exists \xi \in \Pi, \forall p \in \Pi, \forall r \in \mathds{N}, \xi \in HO(p, r)$
	\item Tous les processus s'activent en temps fini : $\mathcal{P}_{non-inf} \equiv \exists r \in \mathds{N}, \mathcal{A}_r = \Pi$

\end{itemize}

L'idée de l'algorithme ci-dessous est que chaque processus possède un compteur modulo $k$. À chaque round, il incrémente son compteur, et l'envoie à tous.
Il reçoit alors les valeurs de compteur de tous ses voisins entrants. Il essaye de se synchroniser avec les autres processus s'ils s'accordent tous sur une même valeur de compteur
(deuxième branche dans la fonction de transition).

Sinon, il fait une tentative de "synchronisation forcée", c'est-à-dire il essaye de convaincre tous les processus de se synchroniser sur son propre compteur :
il adopte la valeur $k$ (troisième branche). Au prochain round, il envoie $k$.  Les processus qui reçoivent ce $k$ sont obligés d'adopter 1 (quatrième branche).
Chaque processus ne fait qu'une seule fois cette tentative de synchronisation forcée (la variable $try_p$ sert justement à limiter $p$ à une seule tentative).
On note $t_p$ le round auquel le processus $p$ exécute cette branche, si cela se produit. $t_p = 0$ sinon.

Lorsqu'un processus s'active, il essaye de s'aligner sur les valeurs qu'il entend, avant d'envoyer sa propre valeur.
On introduit donc une valeur $\bot$, qui sera envoyé lors du round d'activation, qui sera traité comme un $nil$ par les processus récepteurs.

\pagebreak[1]

\begin{algorithm}[htb]
\begin{distribalgo}[1]
\BLANK \INDENT{\textbf{Initialization:}}
	\STATE $x_p \in \mathds{Z}/k\mathds{Z}$
	\STATE $try_p = true$
	\STATE $fire_p = false$

\ENDINDENT \BLANK

\INDENT{\textbf{Round 0:}}
 \INDENT{$S_p:$}
    \STATE send $\bot$ to all processes
  \ENDINDENT
\ENDINDENT
  \BLANK
\INDENT{\textbf{Round $r+1$:}}
 \INDENT{$S_p:$}
    \STATE send $\langle x_p \rangle$ to all processes
  \ENDINDENT
\ENDINDENT
  \BLANK
\INDENT{\textbf{Round $r$:}}
	\INDENT{$T_p(M):$}
	\IF{$M(\Pi) = \{k\}$}
	\STATE $fire_p = true$ ~~~~\COMMENT{si tous les messages reçus valent 0, aucun $nil$, l'algorithme fait feu}
	\IF{$M(\Pi) \setminus \{nil, \bot\} = \{v\}$}
	\STATE $x_p = v+1~mod~k$ ~~~~\COMMENT{si tous les messages ont la même valeur, s'aligner}
	\ELSIF{$try_p \wedge k \notin M(\Pi)$}
	\STATE $try_p = false$ ~~~~\COMMENT{si plusieurs messages sont discordants, faire une tentative de synchronisation forcée}
	\STATE $x_p = k$
	\ELSE
	\STATE $x_p = 1$ ~~~~\COMMENT{si un k a été reçu, toujours s'aligner}
	\ENDIF
  \ENDINDENT
\ENDINDENT 
\caption{The {\em SyncMod} algorithm} \label{algo:R}
\end{distribalgo}

\end{algorithm}

\subsection{Notations}

Soit $V \subseteq \mathds{Z}/k\mathds{Z}$. On dit que le système est V-valent au round $t$ si $V = \{v \in \mathds{Z}/k\mathds{Z}, \exists p \in \Pi, x_p(t) = v\}$.
On dit que le système est v-monovalent si le système est $\{v\}$-valent.

On dit que $p$ "fait feu" au round $t$ si $p$ exécute la première branche au round $t$.

\subsection{Preuve}

\begin{lemma}
	Si $x_\xi(t) = k$, le système est 1-monovalent au round $t+1$.
\end{lemma}
\begin{proof}
	$\xi$ est le centre de l'étoile.
	Si $x_\xi(t) = k$, au round $t+1$, $\xi$ envoie $k$ à tous les processus. Donc tous les processus, recevant un $k$, exécutent nécessairement la deuxième ou quatrième branche,
	et adoptent donc $1$. Le système devient donc 1-monovalent.
\end{proof}

\begin{lemma}
	Si, au round $s$, $\xi$ est actif, et le système est v-monovalent, pour tout $i \in \mathds{N}$, le système est $v+i~mod~k$-monovalent.
\end{lemma}
\begin{proof}
	On montre cela par récurrence sur $i$.
	\begin{description}
		\item[Initialisation :] Vrai par hypothèse.
		\item[Hérédité :] On suppose que $\forall p \in \mathcal{A}_{s+i}, x_p(s+i) = v+i~mod~k$.
			On veut montrer que $\forall p \in \mathcal{A}_{s+i+1}, x_p(s+i+1) = v+i+1~mod~k$.
			Au round $s+i+1$, les processus dans $\mathcal{A}_{s+i}$ envoyent $v+i~mod~k$, d'après l'hypothèse de récurrence.
			Les processus de $\mathcal{A}_{s+i+1} \setminus \mathcal{A}_{s+i}$ envoyent $\bot$. Les autres envoyent $nil$.
			Comme $\xi$ est actif, les processus reçoivent tous au moins une fois $v+i~mod~k$. Donc tous les processus actifs exécutent la deuxième branche, et adoptent $v+i+1~mod~k$.
	\end{description}
\end{proof}

\begin{lemma}
	Si $x_\xi(t) = k$, tout processus $p$ fait feu au plus tard au round $t+k i$ où $i$ est le plus petit entier vérifiant $t+k i \geq s_{max} = max \{s_p, p \in \Pi\}$.
\end{lemma}
\begin{proof}
	D'après les lemmes 1 et 2, le système devient 1-monovalent au round $t+1$ et $i~mod~k$-monovalent à tous les rounds $t+i$ suivants.
	À partir du round $s_{max}$, tous les processus sont actifs.
	Ainsi, au premier round $t+k i$ vérifiant $t+k i \geq s_{max}$, le système est $k$-monovalent, donc au round $t+k i +1$, tous les processus envoyent $k$ à leurs voisons sortants.
	Ainsi, les processus ne reçoivent que des $k$, donc tous les processus font feu.
\end{proof}

\begin{lemma}
	Si le système est monovalent au round $t$, tous les processus auront fait feu au plus tard au round $max(t, s_{max})+k$.
\end{lemma}
\begin{proof}
	D'après le lemme 2, le système est toujours monovalent au round $max(t, s_{max})$ et tous les processus sont actifs.
	Donc, au plus après $k$ rounds, le système entrera dans un état $k$-monovalent, et au round suivant, tous les processus enverront $k$.
	Ainsi, tous les processus qui n'auront pas encore fait feu, feront feu.
	Donc, au plus tard au round $max(t, s_{max})+k$, tous les processus auront fait feu.
\end{proof}

On introduit la fonction $C(t) = (x_\xi(t), \{v \in \mathds{Z}/k\mathds{Z}, \exists p \in \Pi, x_p(t) = v\})$.
Ainsi, si $C(t) = (v_0, V)$, le système est V-valent au round $t$.

On introduit un ensemble $\mathcal{F}$ de configurations dites "favorables".
Ainsi, $C(t) = (v, V) \in \mathcal{F} \equiv v = k \vee |V| = 1$.
Les deux lemmes précédents montrent que si une exécution atteint une configuration favorable, la vivacité est garantie.

\begin{lemma}
	Toute exécution de cet algorithme est sûr vis-à-vis du problème de synchronisation modulo k lorsque les prédicats $\mathcal{P}_{\xi-nek}$ et $\mathcal{P}_{non-inf}$ sont vérifiés.
\end{lemma}
\begin{proof}

	On suppose que $p$ fait feu au round $r$, et est le premier. Nécessairement, $\xi \in HO(p,r)$.
	La ligne 12 du code montre que la valeur envoyé par $\xi$ est 0.
	Donc tous les processus ont reçus un 0. Soit $p$ un processus actif quelquonque. Au round $r$, il execute, dans sa fonction de transition, l'une des quatre branches possibles.
	\begin{itemize}

		\item Il ne peut pas exécuter la troisième branche car $M(\xi) = 0$.
		\item S'il exécute la première, deuxième ou la quatrième branche, il adopte 0.

	\end{itemize}

	On montre maintenant par récurrence sur $i$ la proposition suivante :

	$$\forall i \in \mathds{N}, \forall p \in \mathcal{A}_{r+i}, x_p(r+i) = i~mod~k$$

	\textbf{Initialisation : } voir ci-dessus

	\textbf{Hérédité :}
	On suppose que $\forall p \in \mathcal{A}_{r+i}, x_p(r+i) = i~mod~k$.

	On veut montrer que $\forall p \in \mathcal{A}_{r+i+1}, x_p(r+i+1) = i+1~mod~k$.

	Soit $M$ la fonction de réception de $p$ au round $r+i+1$.
	L'hypothèse de récurrence montre que $M(\Pi) \subseteq \{nil, \bot, i+1~mod~k\}$.
	De plus, $M(\xi) = i+1~mod~k$. Donc $p$ exécute la première ou la deuxième branche de la fonction de transition, et adopte $i+1~mod~k$.
	Cela achève la récurrence.

	La sûreté découle de cette proposition.

\end{proof}

Pour les trois lemmes ci-dessous, on se fixe une exécution de l'algorithme.
On dit que le système est synchonisé au round $i$ si tous les processus actifs ont la même valeur de compteur dans $\mathds{Z}/k\mathds{Z}$.

\begin{lemma}
	Sous le prédicat $\mathcal{P}_{\xi-nek}$,
	si au round $i$, $\xi$ est actif et le système est synchonisé, il l'est également au round $i+1$.
\end{lemma}
\begin{proof}
	Si le système est synchronisé au round $i$, au round $i+1$, tous les processus enverront une même valeur $v$.
	Donc, pour tout $p \in \mathcal{A}_{i+1}$, on aura $M_p^{i+1}(\Pi) \subseteq \{v+1~mod~k, nil, \bot\}$.
	Le fait que $\xi$ est actif induit $v+1~mod~k \in M_p^{i+1}(\Pi)$.
	Donc la deuxième branche de la fonction de transition sera exécutée.
	La synchronisation est donc maintenue.
\end{proof}
\begin{lemma}
	Sous les prédicats $\mathcal{P}_{\xi-nek}$ et $\mathcal{P}_{non-inf}$,
	si au round $i$, $\xi$ est actif et le système est synchonisé, tous les processus font feu.
\end{lemma}
\begin{proof}
	Le lemme précédent montre qu'à partir du round $i$, le système reste synchonisé. De plus, à chaque round, la valeur des compteurs est incrémentée.
	Soit $r$ le round à partir duquel tous les processus sont actifs.
	Ainsi, lors d'un round $i' \geq max(i, r)$, le système atteindra un état dans lequel tous les compteurs valent $k-1$.
	On aura alors, au round suivant, $\forall p \in \Pi, M_p^{i'}(\Pi) = \{0\}$.
	Donc tous les processus font feu.
\end{proof}

On montre maintenant la vivacité dans un cas simple.
Soit $\mathcal{P}_{star} \equiv \exists \xi \in \Pi, \forall p \in \Pi, \forall r \in \mathds{N}, HO(p, r) = \{\xi\}$

\begin{theorem}
	L'algorithme résout la synchronisation modulo $k$ lorsque les prédicats $\mathcal{P}_{star}$ et $\mathcal{P}_{non-inf}$ sont vérifiés.
\end{theorem}
\begin{proof}
	$\xi$ s'active en temps fini d'après le prédicat $\mathcal{P}_{non-inf}$.
	Soit $r$ le round auquel $\xi$ s'active.
	Au round $r$, $\xi$ exécute nécessairement la troisième branche de la fonction de transition, donc $x_\xi(r) = k-1$.
	Ainsi, au round $r+1$, $\xi$ enverra 0 à tous, donc le système se synchronisera sur 0.
	Le lemme précédent montre la vivacité, et le lemme 1 la sûreté.
\end{proof}

On montre maintenant la vivacité dans le cas qui nous intéresse.

\begin{theorem}
	L'algorithme résout la synchronisation modulo $k$ lorsque les prédicats $\mathcal{P}_{\xi-nek}$ et $\mathcal{P}_{non-inf}$ sont vérifiés.
\end{theorem}
\begin{proof}
	$\xi$ s'active en temps fini d'après le prédicat $\mathcal{P}_{non-inf}$.
	Soit $r$ le round auquel $\xi$ s'active.
	L'algorithme est conçu de telle manière que chaque processus n'exécute la troisième branche de la fonction de transition qu'une seule fois.
	Donc il existe un round $r' \geq r$ à partir duquel plus aucun processus n'exécute cette troisième branche.

	\begin{description}

		\item[Cas 1 : ] $\xi$ a exécuté la troisième branche de la fonction de transition au cours d'un round $t \geq r$.

			Dans ce cas, au round $t+1$, $\xi$ envoie 0 à tous, donc tous les processus s'alignent sur ce 0. Le système est maintenant synchonisé. Le lemme précédent montre la vivacité.

		\item[Cas 2 : ] $\xi$ n'a jamais exécuté la troisième branche, et n'a pas exécuté la quatrième branche au delà du round $r'$.

			Dans ce cas, au delà du round $r'$, $\xi$ n'exécute que les deux premières branches. Un raisonnement similaire à la preuve du théorème 1 montre la vivacité.

		\item[Cas 3 : ] $\xi$ n'a jamais exécuté la troisième branche, mais a exécuté la quatrième branche au round $t \geq r'$.

			Ainsi, au round $t+1$, $\xi$ envoie 1 à tous. Comme $t \geq r'$, les processus ne peuvent exécuter que la deuxième et la quatrième branche.
			Les seules valeurs présentes dans le système à la fin du round $t+1$ sont donc 0 et 1.
			On distingue deux sous-cas possibles :

			\begin{description}

				\item[Sous-cas 3.1] $x_\xi(t+1) = 1$

					Dans ce cas, on montre par récurrence sur $i$ que
					$$\forall i \in \mathds{N}, x_\xi(t+i) = i~mod~k \wedge \forall p \in \mathcal{A}_{t+i}, x_p(t+i) \leq x_\xi(t+i)$$

					\textbf{Initialisation : } voir ci-dessus.

					\textbf{Hérédité : } Pour $i$ donné, on suppose que 
					$$x_\xi(t+i) = i~mod~k \wedge \forall p \in \mathcal{A}_{t+i}, x_p(t+i) \leq x_\xi(t+i)$$
					L'hypothèse de récurrence montre que les messages envoyés au round $t+i+1$ sont compris entre 1 et $i+1~mod~k$.
					Le processus $\xi$ reçoit toujours sa propre valeur, à savoir $i+1~mod~k$, mais n'en reçoit jamais d'autre, celà provoquerait l'exécution de la troisième branche.
					On a pourtant supposé que $\xi$ n'exécutait jamais cette branche. Cela prouve $x_\xi(t+i+1) = i+1~mod~k$.

					Les autres processus reçoivent au round $t+i+1$ des valeurs comprises entre 1 et $i+1~mod~k$, cela prouve $\forall p \in \mathcal{A}_{t+i}, x_p(t+i) \leq x_\xi(t+i)$.

					Cette récurrence montre qu'au round $t+k$, $\xi$ envoie 0 à tous, ce qui synchrones le système. Le lemme précédent assure la vivacité.

				\item[Sous-cas 3.2] $x_\xi(t+1) = 0$

					Dans ce cas, $\xi$ envoie 1 à tous les processus au round $t+i+1$. Les valeurs envoyées par les autres processus sont nécessairement 1 et 2.
					$\xi$ ne peut pas avoir reçu un 2, car on a supposé que $\xi$ n'a jamais exécuté la troisième branche. Donc la valeur de $\xi$ à la fin du round est nécessairement 1.
					Pour les autres processus, les seules valeurs possibles sont 0 et 1. On est donc ramené au cas précédent.
			\end{description}
	\end{description}

	Cette disjonction de cas achève donc la vivacité. La sûreté étant garantie par le lemme 1.
\end{proof}

\end{document}
