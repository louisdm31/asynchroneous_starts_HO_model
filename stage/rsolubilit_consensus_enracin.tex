\documentclass{article}
\usepackage{amsmath}
\usepackage[noend]{sources_Heard-Of/distribalgo}
\usepackage{algorithm}
\usepackage{amssymb}
\usepackage{amsthm}
\usepackage{dsfont}
\usepackage[left=2cm, right=2cm, top=2cm]{geometry}
\usepackage[utf8]{inputenc}
\newtheorem{lemma}{Lemme}
\newtheorem{theorem}{Théorème}
\newtheorem{definition}{Définition}
\setlength{\parskip}{0.22cm}
%\newcommand{\st0}{q^0}


\title{Résolubilité du consensus sous différents prédicats}
\date{17 juin 2020}
\author{Louis Penet de Monterno}
\begin{document}
\maketitle

Le but de ce document est de déterminer la résolubilité du consensus lorsque le graphe dynamique est supposé contenir une étoile.
On envisagera plusieurs variantes du précédent prédicat, que l'on construira, entre autre, à l'aide de transformateurs de prédicats.
On montrera que le consensus n'est pas résoluble, sauf dans la variante la plus forte, à savoir dans le cas où le graphe est constamment en étoile,
où le centre de l'étoile est fixe, et où tous les processus s'activent en temps fini.

\section{Notations}

On considère un prédicat $\Phi$ sur un graphe. On définit les transformateurs de prédicats de la manière suivante.

\begin{definition}
	On définit le prédicat $P_k(\Phi)$ sur un graphe dynamique
	$\mathds{G}$ par $$P_k(\Phi) \equiv \forall t \in \mathds{N}, \mathds{G}(t:t+k) \models \Phi$$
\end{definition}

$\Phi$ est toujours équivalent à $P_k(\Phi)$, puisqu'on peut trivialement construire un algorithme $A_k$ qui traduise $P_k(\Phi)$ en $\Phi$.

\begin{definition}
	On définit le prédicat $P_{borné}(\Phi)$ sur un graphe dynamique
	$\mathds{G}$ par $$P_{borné}(\Phi) \equiv \exists b \in \mathds{N}, \forall t \in \mathds{N}, \mathds{G}(t:t+b) \models \Phi$$
\end{definition}
\begin{definition}
	On définit le prédicat $P_{fini}(\Phi)$ sur un graphe dynamique
	$\mathds{G}$ par $$P_{fini}(\Phi) \equiv  \forall t \in \mathds{N}, \exists b \in \mathds{N}, \mathds{G}(t:t+b) \models \Phi$$
\end{definition}

Le prédicat $\mathcal{P}_{borné}(\Phi)$ est une version affaiblie de $\Phi$, où $\Phi$ est vérifié avec délai borné, mais inconnu des processus.
Le prédicat $\mathcal{P}_{fini}(\Phi)$ est une version davantage affaiblie de $\Phi$, où $\Phi$ est vérifié avec délai inconnu des processus, et pouvant varier au cours du temps.

On définit les prédicats suivants :
\begin{itemize}
	\item Le graphe $G = (V, E)$ vérifie le prédicat $\Phi_{nek}$ si et seulement si $\exists p \in \Pi, \forall q \in \Pi, (p, q) \in E$.
	\item Le graphe dynamique $\mathds{G}$ vérifie le prédicat $\mathcal{P}_{nek}$ si pour tout $\forall n \in \mathds{N}, \mathds{G}_n \models \Phi_{nek}$. 
	\item Étant donné $\xi \in \Pi$. Le graphe $G = (V, E)$ vérifie le prédicat $\Phi_{\xi-nek}$ si et seulement si $\forall q \in \Pi, (\xi, q) \in E$.
	\item Le graphe dynamique $\mathds{G}$ vérifie le prédicat $\mathcal{P}_{\xi-nek}$ si pour tout $\forall n \in \mathds{N}, \mathds{G}_n \models \Phi_{\xi-nek}$. 
	\item Le calendrier d'activation $\mathcal{A}$ vérifie le prédicat $\mathcal{P}_{non-inf}$ si et seulement si $\exists k \in \mathds{N}, \mathcal{A}_k = \Pi$.
\end{itemize}

\section{Théorèmes d'impossibilité}

Ce premier théorème montre que le prédicat $\mathcal{P}_{non-inf}$ est indispensable pour la résolution du consensus dans un graphe avec étoile recouvrante.

\begin{theorem}
	Il n'existe pas d'algorithme qui résolve le consensus sous le prédicat $\mathcal{P}_{\xi-nek}$, même si les processus connaissent la taille du réseau.
\end{theorem}
\begin{proof}
	On suppose qu'un tel algorithme existe, et on construit une exécution qui viole l'accord.

	\begin{description}
		\item[Exécution 0 : ] Le graphe dynamique est une étoile constante, centrée sur $\xi$. $\xi$ est passif pour toujours. Les autres processus ont 0 comme valeur initiale.
			Et ils s'activent en temps fini. La terminaison assure qu'il décident tous 0 en temps fini.

		\item[Exécution 1 : ] Le graphe dynamique est une étoile constante, centrée sur $\xi$. $\xi$ est passif pour toujours. Les autres processus ont 1 comme valeur initiale.
			Et ils s'activent en temps fini. La terminaison assure qu'il décident tous 1 en temps fini.

		\item[Exécution 2 : ] Le graphe dynamique est une étoile constante, centrée sur $\xi$. $\xi$ est passif pour toujours. Les autres processus ont différentes valeurs initiales.
			En particulier, un processus $p_0$ a 0 comme valeur initiale, et $p_1$ a 1.
			Et ils s'activent en temps fini. Pour $p_0$, ce scénario est indistinguable du scénario 0, donc $p_0$ décide 0.
			Pour $p_1$, ce scénario est indistinguable du scénario 1, donc $p_1$ décide 1.  Cela viole l'accord.
	\end{description}
\end{proof}

Ce deuxième théorème montre que la racine de l'étoile recouvrante doit être fixe pour que le consensus soit résoluble.

\begin{theorem}
	Il n'existe pas d'algorithme qui résolve le consensus sous les prédicats $\mathcal{P}_{nek}$ et $\mathcal{P}_{non-inf}$, même si les processus connaissent la taille du réseau.
\end{theorem}
\begin{proof}
	On suppose qu'un tel algorithme existe, et on construit une exécution qui viole l'accord.
	Soient $p_1, p_2 \in \Pi$.
	\begin{description}

		\item[Exécution 1 :] $\mathds{G}_r$ est pour tout $r$ égal à l'étoile centrée en $p_1$. $p_1$ s'active en temps fini, donc la terminaison assure que $p_1$ décidera en un round $t$.
		\item[Exécution 2 :] Pour tout $r \leq t$, $\mathds{G}_r$ est l'étoile centrée en $p_1$. $p_1$ s'active au même round que dans l'exécution précédente.
			Pour tout $r > t$, $\mathds{G}_r$ est l'étoile centrée en $p_2$. On suppose que $p_2$ s'active au round $t+1$.
			Dans ces conditions, $p_2$ ne reçoit jamais de message. La terminaison et l'intégrité assurent que $p_2$ décide finalement sa valeur initiale,
			tandis que $p_1$ a déjà décidé. Dans ces conditions, l'accord est impossible à assurer.

	\end{description}
\end{proof}

Enfin, ce dernier théorème montre que le prédicat $\mathcal{P}_{\xi-nek}$ doit être vérifié à chaque round, et non après un certain délai inconnu des processus.

\begin{theorem}
	Il n'existe pas d'algorithme qui résolve le consensus sous les prédicats $P_{borné}(\Phi_{\xi-nek})$ et $\mathcal{P}_{non-inf}$, même si les processus connaissent la taille du réseau.
\end{theorem}
\begin{proof}
	On suppose qu'un tel algorithme existe, et on construit une exécution qui viole l'accord.
	Soient $p_1, p_2 \in \Pi$.
	\begin{description}

		\item[Exécution 1 :] $\mathds{G}_n$ est pour tout $n$ égal à l'étoile centrée en $p_1$. On exécute l'algorithme $A$.
			$p_1$ s'active en temps fini, donc la terminaison assure que $p_1$ décidera en un round $r_1$.
		\item[Exécution 2 :] $\mathds{G}_n$ est pour tout $n$ égal à l'étoile centrée en $p_2$. On exécute l'algorithme $A$.
			$p_2$ s'active en temps fini, donc la terminaison assure que $p_2$ décidera en un round $r_2$.
		\item[Exécution 3 :] Pour tout $n \leq r_0 + r_1$, $\mathds{G}_n$ est le graphe sans aucune arête. Pour tout $n > r_0 + r_1$, $\mathds{G}_n$ est l'étoile centrée en $p_1$.
			$p_1$ et $p_2$ s'activent au même round que lors des deux précédentes exécutions.
			Pour $p_1$, ce scénario est indistinguable du scénario 1, donc $p_1$ décidera au round $r_1$.
			Pour $p_2$, ce scénario est indistinguable du scénario 1, au moins jusqu'au round $r_2$. Donc $p_2$ décidera au round $r_2$.
			Dans ces conditions, l'accord est impossible à assurer.

	\end{description}
\end{proof}

\section{Exécution en série}

Pour la section suivant, on aura besoin de la notion d'exécution en série d'algorithme.
Intuitivement, étant donné deux algorithmes $A$ et $B$, l'algorithme $A;B$ consiste à exécuter $A$, et dès que $A$ atteint un état final, commencer à exécuter $B$.

On suppose donné un algorithme $A = (States_p^A, Init_p^A, S_p^A, T_p^A)$ et un algorithme $B = (States_p^B, Init_p^B, S_p^B, T_p^B)$ (voir document sur la preuve de "one-third rule").
On suppose que $States_p^A$ contient un sous-ensemble $F \subseteq States_p^A$ d'états finaux. On suppose que $T_p^A$ est stable sur $F$.

\begin{definition}
	On définit l'algorithme $A;B$ comme le tuple $(States_p^{AB}, Init_p^{AB}, S_p^{AB}, T_p^{AB})$ où :
	\begin{itemize}

		\item $States_p^{AB} = (States_p^A \times Init_p^B) \cup (F \times States_p^B)$
		\item $Init_p^{AB} = Init_p^A \times Init_p^B$
		\item $S_p^{AB}$ définit par, pour tout $s = (s_A, s_B)$, 
			$$S_p^{AB}(s) = \left \{ \begin{array}{l ll}
				\langle S_p^A(s_A), nil        \rangle & \mbox{si}~s_A \notin F & \{A~\mbox{en cours d'exécution}\} \\
				\langle S_p^A(s_A), S_p^B(s_B) \rangle & \mbox{si}~s_A \in F    & \{B~\mbox{en cours d'exécution}\} \end{array} \right$$
			\item $T_p^{AB}$ définit par, pour tout $s = (s_A, s_B) \in States_p^{AB}$, pour tout $M = (M_A, M_B)$, 
			$$T_p^{AB}(s, M) = \left \{ \begin{array}{l ll}
				\langle T_p^A(s_A, M_A), s_B             \rangle & \mbox{si}~s_A \notin F & \{A~\mbox{en cours d'exécution}\} \\
				\langle T_p^A(s_A, M_A), T_p^B(s_B, M_B) \rangle & \mbox{si}~s_A \in F    & \{B~\mbox{en cours d'exécution}\} \end{array} \right$$

	\end{itemize}
\end{definition}

\section{Algorithme "Uniforme Voting"}

Il reste maintenant à traiter le cas le plus favorable, à savoir le cas où les prédicats $\mathcal{P}_{\xi-nek}$ et $\mathcal{P}_{non-inf}$ sont vérifiés.

\begin{algorithm}[htb]
\scriptsize{
\begin{distribalgo}[1]
\begin{tabular}{ll}
\begin{minipage}{33em}


\INDENT{\textbf{Initialization:}}
  \STATE $x_p := v_p$ ~~~~~~~~\{\emph{$v_p$ is the initial value of $p$}\}
  \STATE $vote_p \in V\cup\{ ? \}$, initially $?$
  \STATE $phase_p = true$

\ENDINDENT
\BLANK

\INDENT{\textbf{Round $r$:}}
	\INDENT{$S_p^r:$}
		\IF{$phase_p$}
			\STATE send $\langle x_p , vote_p \rangle$ to all processes
		\ELSE
			\STATE send $\langle x_p \rangle$ to all processes
		\ENDIF
	\ENDINDENT
	\BLANK
	\INDENT{$T_p^r:$}

		\IF{$phase_p$}
			\IF{$M(q) = \langle v, v \rangle$}
				\STATE $x_p:= v$ ~~~~~~~~\{un vote reçu\}
			\ELSE
				\STATE $x_p :=$ smallest  $w$ from  $\langle w , ? \rangle$ received
			\ENDIF
			\IF{$M(\Pi) = \langle v, v \rangle$}
				\STATE $DECIDE(v)$ ~~~~~~~~\{décider si $p$ ne reçoit que des votes identiques\}
			\ENDIF
			\STATE $vote_p :=\ ?$
		\ELSE
			\STATE $ x_p := min M(\Pi) \setminus \{nil\}$ ~~~~~~~~\{plus petites valeur reçue\}
			\IF{$M(\Pi) = \{v\}$}
				\STATE $vote_p := v$ ~~~~~~~~\{toute les valeurs reçues identiques, aucun nil\}
			\ENDIF
		\ENDIF
		\STATE $phase_p := \neg phase_p$
	\ENDINDENT
\ENDINDENT

\end{minipage}
\end{tabular}

\caption{The {\em UniformVoting} algorithm}
\label{unifvotfig}
\end{distribalgo}
}
\end{algorithm}

On pose le prédicat $\mathcal{P}_{sync-mod-2} \equiv \exists c \in \{0, 1\}, \forall k \in \mathds{N}, \mathcal{A}_{2k+c} = \mathcal{A}_{2k+c+1}$.

\subsection{Intégrité}
\begin{lemma}
	Toute exécution de l'algorithme "Uniforme Voting" vérifie l'intégrité lorsque le prédicat $\mathcal{P}_{sync-mod-2}$ est vérifié.
\end{lemma}
\begin{proof}
	Il suffit de montrer par récurrence qu'à chaque round, les valeurs stockées dans les processus actifs appartiennent nécessairement à l'ensemble des valeurs initiales.
\end{proof}

\subsection{Accord}

\begin{lemma}
	Toute exécution de l'algorithme "Uniforme Voting" vérifie l'accord lorsque le graphe dynamique vérifie le prédicat $\mathcal{P}_{\xi-nek}$
	et le calendrier d'activation vérifie $\mathcal{P}_{sync-mod-2}$.
\end{lemma}
\begin{proof}
	Si un processus $p$ décide une valeur $v$ au round $r$, nécessairement, le processus $\xi$ au centre de l'étoile a voté $v$ au round $r$.
	Donc, au round $r$, tous les processus actifs adoptent $v$.
	Donc toute valeur décidée ultérieurement sera nécessairement $v$. Pour s'en convaincre, on montre aisément par récurrence sur $k$ la proposition suivante :
	$$\forall k \in \mathds{N}, \forall q \in \mathcal{A}_{r+2k}, vote_p(2k) = v \wedge x_q(r+2k) = v \wedge x_q(r+2k+1) = v$$.

	Cela achève l'accord.
\end{proof}

\subsection{Terminaison}

	\begin{lemma}
		Étant donné une exécution vérifiant $\mathcal{P}_{\xi-nek}$ et $\mathcal{P}_{sync-mod-2}$.
		Soit $r$ le round auquel $\xi$, le centre de l'étoile, s'active.
		La suite $(x_\xi(r+k))_{k \in \mathds{N}}$, représentant la succession des valeurs de $\xi$, est décroissante.
	\end{lemma}
	\begin{proof}
		Soit $k \in \mathds{N}$. Soit $M$ la fonction de réception de $\xi$ au round $r+k+1$.
		\begin{itemize}

			\item Si $k+1$ est impair, la ligne 19 du pseudo-code assure que $x_\xi(r+k+1) = min(M(\Pi) \setminus \{nil\})$.
				Or $M(\xi) = x_\xi(r+k)$.  Donc $x_\xi(r+k+1) \leq x_\xi(r+k)$.
			\item Si $k+1$ est pair et si $\exists q \in \mathcal{A}_{r+k+1}, M(q) = \langle v, v \rangle$,
				on a nécessairement $v = x_\xi(r+k)$, puisque $\xi$ a envoyé $x_\xi(r+k)$ au round $r+k$ (cf. lignes 20-21).
				Donc $x_\xi(r+k+1) = x_\xi(r+k)$.
			\item Sinon la ligne 11 du pseudo-code assure que $x_\xi(r+k+1) = min(M(\Pi) \setminus \{nil\})$.
				Or $M(\xi) = \langle x_\xi(r+k), * \rangle$.
				Donc $x_\xi(r+k+1) \leq x_\xi(r+k)$.

		\end{itemize}
		Dans tous les cas, $x_\xi(r+k+1) \leq x_\xi(r+k)$.
	\end{proof}
		

	\begin{lemma}
		Toute exécution de "Uniforme Voting" vérifie la terminaison lorsque le graphe de communication vérifie $\mathcal{P}_{\xi-nek}$
		et lorsque le calendrier d'activation vérifie $\mathcal{P}_{non-inf}$ et $\mathcal{P}_{sync-mod-2}$.
	\end{lemma}
	\begin{proof}
		Soit $r_0$ le round auquel tous les processus sont actifs.

		La suite $(x_\xi(r_0+k))_{k \in \mathds{N}}$ est décroissante et ne contient que des valeurs initiales. Donc elle se stabilise sur une valeur $v$ à partir d'un certain round $r_0+k_0$.
		On considère maintenant le round $r_0+2k_0$, qui est un round d'échange de vote.
		$\xi$ envoie $\langle v, * \rangle$ à tous les processus. Les lignes 8 à 11 du pseudo-code assurent que $\forall q \in \mathcal{A}_{r_0+2k_0}, x_q(r_0+2k_0) \leq v$.

		Soit $M$ la fonction de réception de $\xi$ lors du round $r_0+2k_0+1$.
		Comme $x_\xi(r_0+2k_0+1) = v$, la ligne 19 assure que $\xi$ n'a pas reçu de message inférieur à $v$.
		Donc $M(\Pi) = \{v\}$. Donc $vote_\xi(r_0+2k_0+1) = v$. Donc au round $r_0+2k_0+2$, la ligne 9 assure que tous les processus adoptent $v$.

		Au rounds $r_0+2k_0+3$, les processus ne reçoivent que des messages contenant $v$, donc au rounds $r_0+2k_0+4$, tous les processus votent $v$, donc tous les processus décident.
	\end{proof}
\section{Conclusion}

\begin{theorem}
	L'algorithme "Uniforme Voting" exécuté en série à la suite de l'algorithme "SyncMod" avec $k = 4$
	résout le consensus lorsque les prédicats $\mathcal{P}_{\xi-nek}$ et $\mathcal{P}_{non-inf}$ sont vérifiés.
\end{theorem}
\begin{proof}
	Tout d'abord, grâce à la synchronisation, les phases d'échange de vote et d'échange de valeurs de l'algorithme "Uniforme Voting" seront synchronisées.
	Les trois sections précédentes montrent que l'algorithme "Uniforme Voting" permet ensuite de résoudre le consensus.

\end{proof}
\end{document}
