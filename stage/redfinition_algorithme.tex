\documentclass{article}
\usepackage{amsmath}
\usepackage{algpseudocode}
\usepackage{amssymb}
\usepackage{dsfont}
\usepackage[left=2cm, right=2cm, top=2cm]{geometry}
\usepackage[utf8]{inputenc}

\title{Redéfinition d'un algorithme dans le cadre des départs asynchrones}
\date{30 mars 2020}
\author{Louis Penet de Monterno}

\begin{document}
  \maketitle


  \section{Définition d'un processus}

  \subsection{départs synchrones}

  \begin{description}

	  \item[$\bullet$] un ensemble d'états $states_p$
	  \item[$\bullet$] un ensemble d'états initiaux  $init_p \subset state_p$
	  \item[$\bullet$] un ensemble de messages M
	  \item[$\bullet$] pour tout $p \in \Pi$, une fonction d'émission
		  $S_p : states_p \rightarrow \Pi \rightarrow M$
	  \item[$\bullet$] pour tout $p \in \Pi$, une fonction de transition
		  $T_p : states_p \rightarrow (\Pi \rightarrow (1 \oplus M)) \rightarrow states_p$,
		  où $\Pi \rightarrow (1 \oplus M)$ est le type d'une fonction partielle
		  de type $\Pi \rightarrow M$.

  \end{description}


  \subsection{départs asynchrones, forme générale}

  \begin{description}

	  \item[$\bullet$] un ensemble d'états actifs
		  $states_p$, ainsi qu'un état inactif, noté $sleep_p \notin state_p$.
	  \item[$\bullet$] un ensemble d'états initiaux  $init_p \subset state_p$
	  \item[$\bullet$] un ensemble de messages M
	  \item[$\bullet$] pour tout $p \in \Pi$, une fonction d'émission
		  $S_p : states_p \rightarrow \Pi \rightarrow M$
	  \item[$\bullet$] pour tout $p \in \Pi$, une fonction de transition
		  $T_p : state_p \rightarrow (\Pi \rightarrow (1 \oplus \{ \bot \} \oplus M))
		  \rightarrow state_p$,
		  où $\Pi \rightarrow (1 \oplus \{ \bot \} \oplus M)$ est le type d'une fonction partielle
		  de type $\Pi \rightarrow M$, pouvant prendre une valeur supplémentaire notée $\bot$.

  \end{description}

\section{Définition d'une exécution}

On définit une exécution comme la donnée d'un tuple
$((s_p)_{p \in \Pi}, \mathds{G}, (in_p)_{p \in \Pi})$ où :

\begin{description}

	\item[$\bullet$] $(s_p)_{p \in \Pi}$ donne le round auquel les processus déparrent.
		$s_p \in \mathds{N}$.
	\item[$\bullet$] $\mathds{G}$ est le graphe dynamique indexé sur $\mathds{N}$.
	\item[$\bullet$] $in_p$ est l'état initial de p, vérifiant $\forall p \in \Pi, in_p \in init_p$.

\end{description}

\section{Sémantique des exécutions}

Une excécution $((s_p)_{p \in \Pi}, \mathds{G}, (in_p)_{p \in \Pi})$ détermine inductivement
$\Gamma_p(n)$ l'état de chaque processus à chaque round :

\begin{description}

	\item[$\bullet$] $\forall p \in \Pi,  n \in \mathds{N},
		n < s_p \Rightarrow \Gamma_p(n) = sleep_p$.
	\item[$\bullet$] $\forall p \in \Pi, \Gamma_p(s_p) = in_p$.
	\item[$\bullet$] $\forall p \in \Pi,  n \in \mathds{N}, n \geq s_p \Rightarrow
		\Gamma_p(n+1) = T_p (\Gamma_p(n) ,V_p^n)$ où $V_p^n$ est l'ensemble des messages reçus par p.
	\item[$\bullet$] $V_p^n(q)$ non défini si $q \notin HO(p, n)$.
	\item[$\bullet$] $V_p^n(q) = \bot$ si $n < s_q$.
	\item[$\bullet$] $V_p^n(q) = S_p (\Gamma_p(n))$ sinon.

\end{description}
\end{document}
