\documentclass{beamer}
\usepackage[utf8]{inputenc}
\usepackage{amsmath}
\usepackage[noend]{sources-Heard-Of/distribalgo}
\usepackage{algorithm}
\usepackage{amssymb}
\usepackage{listings}
\usepackage{amsthm}
\usepackage{dsfont}
\usepackage{stmaryrd}

\title{Vérification et preuve formelle dans le modèle Heard-Of}
\date{8 septembre 2020}
\author{Louis Penet de Monterno \linebreak \scriptsize{stage encadré par Bernadette Charron-Bost}}
\institute{LIX}

\begin{document}


\begin{frame}
\frametitle{Algorithme de Coulouma-Godard}
\end{frame}

\begin{frame}
\frametitle{Algorithme de Coulouma-Godard}
\end{frame}
\begin{frame}
\frametitle{Algorithme de Coulouma-Godard}
\end{frame}
\begin{frame}
\frametitle{Algorithme de Coulouma-Godard}
\end{frame}
\begin{frame}
\frametitle{Algorithme de Coulouma-Godard}
\end{frame}
\begin{frame}
\frametitle{Algorithme de Coulouma-Godard}
\end{frame}
\begin{frame}
\frametitle{Problème du Firing-Squad}
	\begin{description}
	   \item[Sûreté : ] Les processus font feu au même round.
	   \item[Vivacité : ] Tous les processus qui s'activent en temps fini, font feu en temps fini. 
	\end{description}

\end{frame}
\begin{frame}
\frametitle{Départs synchrones - départs asynchrones}
Émergence de deux notions de résolubilité :
	\begin{itemize}
		\item Résolubilité forte, tolérant les départs asynchrones.
		\item Résolubilité faible, nécessitant des départs synchrones.
	\end{itemize}
\end{frame}
\end{document}
