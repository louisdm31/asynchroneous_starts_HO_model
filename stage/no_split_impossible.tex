\documentclass{article}
\usepackage{amsmath}
\usepackage{algpseudocode}
\usepackage{amssymb}
\usepackage{dsfont}
\usepackage[left=2cm, right=2cm, top=2cm]{geometry}
\usepackage[utf8]{inputenc}

\title{Consensus avec départs asynchrones dans un système non-éclaté, et infiniment souvent uniforme}
\date{7 avril 2020}
\author{Louis Penet de Monterno}

\begin{document}
  \maketitle

  \textbf{Théorème.} Il n'existe pas d'algorithme de consensus tolérant les départs asynchrones
  correct et vivace sous le prédicat $\mathcal{P}_{no-split} \bigcap \mathcal{P}_{S-unif^\infty}$

  \textbf{Preuve.}

  On se place dans le cas du consensus binaire, avec $ | \Pi | = 2$.
  La preuve est généralisable avec $n > 2$.

  Soient deux processus $p_1$, $p_2$. On suppose d'abord que :

  \begin{description}

	  \item[$\bullet$] $p_1$ et $p_2$ se réveillent dès le round 0.
	  \item[$\bullet$] $ HO(p_1, n) = \{ p_1 \}$ et $HO(p_2, n) = \{ p_1, p_2\}$.

  \end{description}

  Cette configuration vérifie le prédicat en hypothèse.
  La vivacité implique que $p_1$ décide au bout d'un temps, noté t.

  \vspace{1cm}
  On se place dans une autre configuration vérifiant :

  \begin{description}

	  \item[$\bullet$] $p_1$ se réveille dès le round 0.
	  \item[$\bullet$] $p_2$ se réveillent au round t+2.
	  \item[$\bullet$] $\forall n \in \mathds{N} \leq t+1 , 
		  HO(p_1, n) = \{ p_1 \}$ et $HO(p_2, n) = \{ p_1, p_2\}$.
	  \item[$\bullet$] $\forall n \in \mathds{N} > t+1 ,
		  HO(p_2, n) = \{ p_2 \}$ et $HO(p_1, n) = \{ p_1, p_2\}$.

  \end{description}

  Pendant les t premiers rounds, du point de vue de $p_1$,
  cette configuration est indistingable de la précédente.
  Donc $p_1$ décidera au round t. Quand $p_2$ se réveille au round t+2,
  $p_1$ a déjà décidé, mais $p_2$ n'a aucun moyen de connaître cette valeur.
  Donc l'accord est impossible à garantir.


\end{document}
