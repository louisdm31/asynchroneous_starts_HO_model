\documentclass{article}
\usepackage{amsmath}
\usepackage[noend]{sources_Heard-Of/distribalgo}
\usepackage{algorithm}
\usepackage{amssymb}
\usepackage{amsthm}
\usepackage{dsfont}
\usepackage[left=2cm, right=2cm, top=2cm]{geometry}
\usepackage[utf8]{inputenc}
\newtheorem{lemma}{Lemme}
\newtheorem{theorem}{Théorème}
\newtheorem{definition}{Définition}
\setlength{\parskip}{0.22cm}
%\newcommand{\st0}{q^0}


\title{Résolubilité du consensus sous différents prédicats}
\date{17 juin 2020}
\author{Louis Penet de Monterno}
\begin{document}
\maketitle

\section{Théorèmes d'impossibilité}

Soit $\xi \in \Pi$ un pocessus fixé.
On pose $\Phi$ le prédicat $\forall t \in \mathds{N}, \forall q \in \Pi, path(\mathds{G}(t : t+k), \xi, q)$, où $path(G, \xi, q)$ signifie qu'il existe un chemin de $\xi$ vers $q$ dans le graphe $G$.

\begin{theorem}
	Le consensus n'est pas fortement résoluble sous l'hypothèse $P_k(\Phi)$ où $k$ est un entier quelquonque.
\end{theorem}
\begin{proof}
	Si la racine est passive pour toujours, le prédicat n'est pas exploitable.
\end{proof}

On pose $\Psi$ le prédicat $\forall t \in \mathds{N}, \exists p \in \Pi, \forall q \in \Pi, path(\mathds{G}(t : t+k), p, q)$.
\begin{theorem}
	Le consensus n'est pas résoluble sous l'hypothèse $P_k(\Psi)$ où $k$ est un entier quelquonque.
\end{theorem}
\begin{proof}
	On suppose qu'un algorithme $A$ résoud le consensus sous le prédicat $P_k(\Psi)$. Soient $p_1, p_2 \in \Pi$.
	\begin{itemize}

		\item \textbf{Scénario 1 :} $\mathds{G}_n$ est pour tout $n$ égal à l'étoile centrée en $p_1$. On exécute l'algorithme $A$. La terminaison assure que $p_1$ décidera en un round $r$.
		\item \textbf{Scénario 2 :} Pour tout $n \leq r$, $\mathds{G}_n$ est l'étoile centrée en $p_1$. Pour tout $n > r$, $\mathds{G}_n$ est l'étoile centrée en $p_2$. On suppose que $p_2$
			s'active au round $r+1$. Dans ces conditions, $p_2$ ne reçoit jamais de message. La terminaison et l'intégrité assurent que $p_2$ décide finalement sa valeur initiale,
			tandis que $p_1$ a déjà décidé. Dans ces conditions, l'accord est impossible à assurer.

	\end{itemize}
\end{proof}

\begin{theorem}
	Le consensus n'est pas résoluble sous l'hypothèse $P_{borné}(\Phi)$.
\end{theorem}
\begin{proof}
	On suppose qu'un algorithme $A$ résoud le consensus sous le prédicat $P_{borné}(\Phi)$. Soient $p_1, p_2 \in \Pi$.
	\begin{itemize}

		\item \textbf{Scénario 1 :} $\mathds{G}_n$ est pour tout $n$ égal à l'étoile centrée en $p_1$. On exécute l'algorithme $A$. La terminaison assure que $p_1$ décidera en un round $r_1$.
		\item \textbf{Scénario 2 :} $\mathds{G}_n$ est pour tout $n$ égal à l'étoile centrée en $p_2$. On exécute l'algorithme $A$. La terminaison assure que $p_2$ décidera en un round $r_2$.
		\item \textbf{Scénario 3 :} Pour tout $n \leq r_0 + r_1$, $\mathds{G}_n$ est le graphe sans aucune arête. Pour tout $n > r_0 + r_1$, $\mathds{G}_n$ est l'étoile centrée en $p_1$.
			Pour $p_1$, ce scénario est indistinguable du scénario 1, donc $p_1$ décidera au round $r_1$.
			Pour $p_2$, ce scénario est indistinguable du scénario 1, au moins jusqu'au round $r_2$. Donc $p_2$ décidera au round $r_2$.
			Dans ces conditions, l'accord est impossible à assurer.

	\end{itemize}
\end{proof}


\section{Algorithme "Uniforme Voting"}

Dans cette section, on propose une adaptation de l'algorithme  "Uniforme Voting", prévue pour fonctionner avec des départs asynchrones.
Soit $V$ un ensemble totalement ordonné.

\begin{itemize}

	\item Pour tout processus $p$, on pose
		$State_p = V \times Bool \times Bool \times V \times Bool \times Bool \times Bool$.
	\item Pour tout processus $p$, on pose
		$Init_p = \{\langle v, False, False, v, False, False, False \rangle, v \in V\}$
	\item Les messages sont de type $V \times (V \times Bool \uplus \{nil\})$.


	\item On définit $fst : V \times (V \times Bool \uplus \{nil\}) \uplus \{nil\} \rightarrow V \uplus \{nil\}$ par
	\begin{itemize}
		\item $fst(nil) = nil$
		\item $fst(\langle v, w \rangle) = v$
	\end{itemize}

	\item On définit $snd : V \times (Bool \times Bool \uplus \{nil\}) \uplus \{nil\} \rightarrow (V \times Bool) \uplus \{nil\}$ par
	\begin{itemize}
		\item $fst(nil) = nil$
		\item $fst(\langle v, w \rangle) = w$
	\end{itemize}
\end{itemize}

On définit l'algorithme "Uniforme Voting" à l'aide du pseudo-code suivant :
\begin{algorithm}[htb]
\scriptsize{
\begin{distribalgo}[1]
\begin{tabular}{ll}
\begin{minipage}{41em}


\INDENT{\textbf{Initialization:}}
	\STATE $x^1_p :=\, v_p$ ~~~~\{\emph{$v_p$ is the initial value of $p$}\}
	\STATE $vote^1_p \in Bool$, initiallement $False$
	\STATE $dec^1_p \in Bool$, initialement $False$
	\BLANK
	\STATE $x^2_p :=\, v_p$
	\STATE $vote^2_p \in Bool$, initiallement $False$
	\STATE $dec^2_p \in Bool$, initialement $False$

\ENDINDENT
\BLANK

\INDENT{\textbf{Fonctions sur une instance}}
	\INDENT{$T^1_p(M, ref ~ x, ref ~ vote, ref ~ dec):$}
		\STATE $ x := min M(\Pi) \setminus \{nil\}$ 
		\IF{$M(\Pi) = \{v\}$} 
			\STATE $vote := True$ ~~~~\{tous les messages valent $v$, aucun $nil$\}
		\ENDIF
	\ENDINDENT
	\BLANK

	\INDENT{$T^2_p(M, ref ~ x, ref ~ vote, ref ~ dec):$}

		\IF{$\exists v \in V, q \in \Pi, M(q) = \langle v, True \rangle$}
			\STATE $x := v$ ~~~~\{au moins un vote reçu\}
		\ELSE
			\STATE $x := fst (min (M(\Pi) \setminus {nil}))$ ~~~~\{plus petite valeur reçue\}
		\ENDIF
		\IF{$M(\Pi) = \{\langle v, True \rangle\}$}
			\STATE $dec := True$ ~~~~\{tous les messages sont des votes pour $v$\}
		\ENDIF
		\STATE $vote_p := False$
	\ENDINDENT
\ENDINDENT

\end{minipage}
&
\begin{minipage}{31em}

\INDENT{\textbf{Round $r=0\,$:}}
	\INDENT{$S_p:$}
		\STATE send $\langle nil, \langle  x^1_p , False \rangle \rangle$ to all processes
	\ENDINDENT
	\INDENT{$T_p(M):$}
		\STATE $T^2(snd \circ M, x^1_p, vote^1_p, dec^1_p)$
	\ENDINDENT
\ENDINDENT

\INDENT{\textbf{Round $r=2\phi+1\,$:}}
	\INDENT{$S_p:$}
		\STATE send $\langle\, x^1_p\, ,\, \langle \, x^2_p \, ,\, vote^2_p \, \rangle \, \rangle$ to all processes
	\ENDINDENT
	\BLANK
	\INDENT{$T_p(M):$}
		\STATE $T^1(fst \circ M, x^1_p, vote^1_p, dec^1_p)$
		\STATE $T^2(snd \circ M, x^2_p, vote^2_p, dec^2_p)$
		\IF{$dec^1_p \wedge dec^2_p $}
			\STATE $DECIDE(min ~ x^1_p ~ x^2_p)$
		\ENDIF
	\ENDINDENT
\ENDINDENT

\INDENT{\textbf{Round $r=2\phi+2\,$:}}
	\INDENT{$S_p:$}
		\STATE send $\langle x^2_p , \langle x^1_p, vote^1_p \rangle \rangle$ to all processes
	\ENDINDENT
	\BLANK
	\INDENT{$T_p(M)$}
		\STATE $T^1_p(fst \circ M, x_p^2, vote^2_p, dec^2_p)$
		\STATE $T^2_p(snd \circ M, x_p^1, vote^1_p, dec^1_p)$
		\IF{$dec^1_p \wedge dec^2_p $}
			\STATE $DECIDE(min ~ x^1_p ~ x^2_p)$
		\ENDIF
	\ENDINDENT
\ENDINDENT

\end{minipage}
\end{tabular}

\caption{The {\em UniformVoting} algorithm}
\label{unifvotfig}
\end{distribalgo}
}
\end{algorithm}

\textbf{Observations :} Cet algorithme consiste en fait à exécuter deux instances de "Uniforme Voting" en parallèle.
L'état est donc la concaténation de deux sous-états.
Les deux fonctions $T^1$ et $T^2$ définissent le comportement de chaque instance induviduellement.
Au round $0$, seule la première instance démarre.
Aux rounds impairs, la première instance effectue sa phase d'échange de vote, tandis que la seconde, sa phase d'échange de valeurs.
Aux rounds pairs, c'est l'inverse.
À chaque round, les processus reçoivent des messages de la forme $\langle x_1 , \langle x_2, vote_2 \rangle \rangle$.
Si $M$ est la fonction de réception, pour un round donné, et un processus fixé, $fst \circ M$ est la fonction de réception restreinte à l'échange de valeurs.
De même $snd \circ M$ est celle restreinte à l'échange de vote.
On constate que, à chaque round, les instances qui en sont à la phase d'échange de valeur communiquent exclusivement entre elles, et de même pour celles qui en sont à l'échange de vote.
On note donc $\mathcal{U}_{vote}^n$, qui à chaque processus $p$ associe l'objet $\langle x^i_p, vote^i_p, dec^i_p \rangle$
correspondant à l'instance qui en est à l'échange de vote au round $n$, ou $nil$ si inactif. On note de même $\mathcal{U}_{val}^n$.

\subsection{Intégrité}
\begin{lemma}
	$\forall n \in \mathds{N}, \forall p \in \mathcal{A}_{n+1}, \forall x \in V, \mathcal{U}_{val}^n(p) = \langle x, *, * \rangle \Rightarrow x \in \{\sigma^0_q, q \in \Pi\}$
	(resp. $\mathcal{U}_{vote}^n(p)$).
\end{lemma}
\begin{proof}

	On montre ce lemma par récurrence sur $n$.

	\textbf{Initialization :} Triviale, car $p \in \mathcal{A}_1 \Rightarrow x^1_p(0) = x^2_p(0) = \sigma^0(p)$.

	\textbf{Hérédité :} On suppose, pour un $n$ fixé, que, pour tout $p \in \mathcal{A}_{n+1}$, pour tout $x \in V$ :
	\begin{itemize}

		\item $\mathcal{U}_{val}^n(p)  = \langle x, *, * \rangle \Rightarrow x \in \{\sigma^0_q, q \in \Pi\}$
		\item $\mathcal{U}_{vote}^n(p) = \langle x, *, * \rangle \Rightarrow x \in \{\sigma^0_q, q \in \Pi\}$.

	\end{itemize}
	Soit $p \in \mathcal{A}_{n+2}$. On montre maintenant les deux parties de l'invariant.
	\begin{itemize}

		\item On suppose que $\mathcal{U}^{n+1}_{val}(p) = \langle x, *, * \rangle$.
			En vertu de la ligne 10 du pseudo-code, on a $x = min (M(\Pi) \setminus \{nil\})$ avec $M = fst \circ M^{n+1}_p$.
			On applique la seconde hypothèse au procesus qui a émis ce vote.
			Ainsi, il existe un procesus $q$ tel que, si $\mathcal{U}_{vote}^n(q) = \langle x', *, *\rangle$, on obtient $x = x' \in \{\sigma^0_q, q \in \Pi\}$.
			Cela prouve la première partie de l'invariant.
		\item On suppose que $\mathcal{U}^{n+1}_{vote}(p) = \langle x, *, * \rangle$.
			\begin{itemize}
				\item Si il existe un $q$ tel que $M(q) = \langle x', True, * \rangle$, on applique la première hypothèse au processus $q$.
					On obtient $x = x' \in \{\sigma^0_q, q \in \Pi\}$ avec $M = snd \circ M^{n+1}_p$.
				\item Sinon, $x = min (M(\Pi) \setminus \{nil\})$ (cf. ligne 17 dans le pseudo-code).
					On considère le procesus $q$ tel que $min (M(\Pi) \setminus \{nil\}) = M(q)$,
					et on applique à $q$ la première hypothèse. Ainsi, si $\mathcal{U}_{val}^n(q) = \langle x', *, * \rangle$, on a $x = x' \in \{\sigma^0_q, q \in \Pi\}$.
			\end{itemize}
			Dans tout les cas, cela prouve la deuxième partie de l'invariant.
	\end{itemize}
\end{proof}

\begin{lemma}
	Toute exécution de l'algorithme "Uniforme Voting" vérifie l'intégrité.
\end{lemma}
\begin{proof}
	En utilisant les deux versions de ce lemme, on montre que $\forall n \in \mathds{N}, \forall p \in \mathcal{A}_n, min (x^1_p(n), x^2_p(n)) \in \{\sigma^0_q, q \in \Pi\}$.
	Cela prouve l'intégrité.
\end{proof}

\subsection{Accord}

\begin{lemma}
	Toute exécution de l'algorithme "Uniforme Voting" vérifie l'accord lorsque le graphe dynamique vérifie le prédicat $\exists \xi \in \Pi, \forall n \in \mathds{N}, \xi \in K(n)$.
\end{lemma}
\begin{proof}
	Supposons que la première instance de $p$ décide au round $n$ une valeur $v$, et que cette instance soit la première à décider parmi les instances dans $\mathcal{U}^n_{vote}$.
	On sait que $\xi \in \mathcal{A}_n$, sinon tous les processus recevrait au moins un message $nil$ à chaque rounds, cela empêcherait tout vote, donc toute décision.
	On montre par récurrence sur $k$ que
	$\forall k \in \mathds{N}, \forall q \in \mathcal{A}_{n+2k},
	\mathcal{U}_{vote}^{n+2k }(q) = \langle v, *, * \rangle \wedge
	\mathcal{U}_{val}^{n+2k+1}(q) = \langle v, True, * \rangle$.

	\textbf{Initialization :} La ligne 18 du pseudo-code montre que, puisque $p$ a décidé, $snd \circ M^n_p(\Pi) = \{\langle v, True \rangle\}$, Donc $snd \circ M^n_p(\xi) = \langle v, True \rangle$.
		Ainsi, au round $n$, $\xi$ a envoyé à tous les processus un message de la forme $\langle *, \langle v, True \rangle \rangle$.
		La ligne 15 du pseudo-code montre que, pour tout $q \in \mathcal{A}_n$, $\mathcal{U}_{vote}^n(q)$ est de la forme $\langle v, *, * \rangle$.

		On montre maintenant que pour tout $q \in \mathcal{A}_{n+1}, \mathcal{U}^{n+1}_{val}(q)$ est de la forme $\langle v, True, * \rangle$.
		La ligne 10 du pseudo-code montre que si $\mathcal{U}^{n+1}_{val} = \langle v', *, * \rangle$, on a $v' = min(M(\Pi) \setminus \{nil\})$ avec $M = fst \circ M_q^{n+1}$.
		De plus les instances s'activent exclusivement lors des phrases d'échanges de votes. Donc $M^{-1}(V) = \mathcal{A}_{n}$.
		D'après le paragraphe précédent, on obtient $M(\Pi) \setminus \{nil\} = \{v\}$. Donc $\mathcal{U}^{n+1}(q)$ est de la forme $\langle v, True, * \rangle$.
		Cela clos l'initialisation.

	\textbf{Hérédité :} Pour $k$ fixé, on suppose que $\forall q \in \mathcal{A}_{n+2k},
		\mathcal{U}_{vote}^{n+2k }(q) = \langle v, *, * \rangle \wedge
		\mathcal{U}_{val}^{n+2k+1}(q) = \langle v, True, * \rangle$.
		Soit $q \in \mathcal{A}_{n+2k+2}$.
		Lors du round $n+2k+2$, tous les processus reçoivent de la part de $\xi$, un message de la forme $\langle *, \langle v, True \rangle \rangle$
		Donc pour tout $q \in \mathcal{A}_{n+2k+2}$, $\mathcal{U}_{vote}^{n+2k+2}(q)$ est de la forme $\langle v, *, * \rangle$.
		De la même manière que pour l'initialisation, on en déduit que $\mathcal{U}^{n+2k+3}(q)$ est de la forme $\langle v, True, * \rangle$.
		Cela clos l'hérédité.

	À partir de cette proposition, on montre que toute décision ultérieure au round $n$ décidra nécessairement $v$. Cela prouve l'accord de chacune des instances de l'algorithme,
	donc cela prouve l'accord de l'algorithme dans son ensemble.
\end{proof}

\subsection{Terminaison}

	\begin{lemma}
		Soit $r$ le round auquel $\xi$ s'active.
		Pour toute exécution vérifiant $\forall n \in \mathds{N}, \xi \in K(n)$,
		la suite $(x^1_\xi(n+r))_{n \in \mathds{N}}$ (resp. $(x^2_\xi(n+r))_{n \in \mathds{N}}$) est décroissante.
	\end{lemma}
	\begin{proof}
		Soit $n \in \mathds{N}$.
		\begin{itemize}

			\item Si $n+1$ est pair, la ligne 10 du pseudo-code assure que $x^1_\xi(n+r+1) = min(M(\Pi) \setminus \{nil\})$ avec $M = fst \circ M_\xi^{n+r+1}$.
				Or $M_\xi^{n+1}(\xi) = \langle x^1_\xi(n), * \rangle$.
				Donc $x^1_\xi(n+1) \leq x^1_\xi(n)$.
			\item Si $n+1$ est impair et si $\exists v \in V, \exists q \in \mathcal{A}_{n+r+1}, M(q) = \langle v, True \rangle$ avec $M = snd \circ M_\xi^{n+r+1}$,
				on a nécessairement $v = x^1_\xi(n+r)$, puisque $\xi$ a envoyé $x^1_\xi(n+r)$ au round $n+r$ (cf. lignes 11-12).
				Donc $x^1_\xi(n+r+1) = x^1_\xi(n+r)$.
			\item Sinon la ligne 17 du pseudo-code assure que $x^1_\xi(n+r+1) = min(fst \circ M(\Pi) \setminus \{nil\})$ avec $M = snd \circ M_\xi^{n+r+1}$.
				Or $M_\xi^{n+r+1}(\xi) = \langle x^1_\xi(n+r), * \rangle$.
				Donc $x^1_\xi(n+r+1) \leq x^1_\xi(n+r)$.

		\end{itemize}
		Dans tous les cas, $x^1_\xi(n+r+1) \leq x^1_\xi(n+r)$.
	\end{proof}
		

	\begin{lemma}
		Toute exécution de "Uniforme Voting" vérifie la terminaison lorsque le graphe de communication vérifie $\forall n \in \mathds{N}, \xi \in K(n)$
		et lorsque le calendrier d'activation vérifie $\exists r_0 \in \mathds{N}, \mathcal{A}_{r_0} = \Pi$.
	\end{lemma}
	\begin{proof}
		On montre l'accord pour chacune des deux instances de l'algorithme.
		En combinant le lemme 1 et le lemme précédent, on montre que la suite $(x^1_\xi(n+r_0))_{n \in \mathds{N}}$ se stabilise sur une valeur $v$ à partir d'un certain round $n+r_0$.
		Au round $2n+r_0+1$, l'instance $\mathcal{U}^{2n+r_0+1}_{val}(\xi)$ envoie donc un message de la forme $\langle v, * \rangle$ à tous les processus.
		La ligne 10 du pseudo-code assure que $\forall q \in \mathcal{A}_{2n+r_0+1}$,
		on a $\mathcal{U}^{2n+r_0+1}_{val}(q)$ de la forme $\langle v', *, * \rangle$ avec $v' \leq v$.
		Au round suivant, on obtient de même $\mathcal{U}^{2n+r_0+2}_{vote}(q)$ de la forme $\langle v', *, * \rangle$ avec $v' \leq v$.

		La fonction de réception $M_\xi^{2n+r_0+3}$ vérifie donc
		$\forall q \in \mathcal{A}_{2n+r_0+3}, fst \circ M_\xi^{2n+r_0+3}(q) \leq v$.
		Comme, au delà du round $n+r_0$, la suite $(x^1_\xi(n+r_0))_{n \in \mathds{N}}$ s'est stabilisée, on a $x^1_\xi(2n+r_0+3) = v$.
		La ligne 10 du pseudo-code assure que $\forall q \in \mathcal{A}_{2n+r_0+3}, fst \circ M_\xi^{2n+r_0+3}(q) = v$.
		Donc $\mathcal{U}_{val}^{2n+r_0+3}(\xi) = \langle v, True, * \rangle$.
		Donc, d'après la ligne 15 du pseudo-code, pour tout $q \in \mathcal{A}_{2n+r_0+4}$, on a $\mathcal{U}_{vote}^{2n+r_0+4}(q)$ de la forme $\langle v, *, * \rangle$
		Au round suivant, d'après les lignes 10-12 du pseudo-code, on obtient $\mathcal{U}^{2n+r_0+5}_{val}(q)$ de la forme $\langle v, True, * \rangle$.
		Au round suivant, d'après les lignes 17-19 du pseudo-code, on obtient $\mathcal{U}^{2n+r_0+6}_{vote}(q)$ de la forme $\langle v, *, True \rangle$.
		Ainsi, toutes les instances de $\mathcal{U}_{vote}^{2n+r_0+6}$ décident.
		Cela prouve la terminaison.

	\end{proof}



\begin{theorem}

	L'algorithme "Uniform Voting" résoud le consensus lorsque :
	\begin{itemize}

		\item le graphe de communication vérifie $P_k(\Phi)$, avec $k$ connu par les processus,
		\item les processus connaissent $\mathcal{N}$ une borne supérieure sur la taille du système,
		\item le calendrier d'activation vérifie $\exists r_0 \in \mathds{N}, \mathcal{A}_{r_0} = \Pi$.

	\end{itemize}
\end{theorem}

\begin{proof}

	Comme les processus connaissent $k$, il existe un algorithme $A_k$, qui traduit $P_k(\Phi)$ en $\Phi$.

	Comme les processus connaissent $\mathcal{N}$, il existe un algorithme $B_\mathcal{N}$ qui traduit $\Phi$ en $\exists \xi \in \Pi, \forall r \in \mathds{N}, \xi \in K(r)$.

	À l'aide de ce dernier prédicat, et de la troisième hypothèse, on montre que l'algorithme "Uniforme Voting" résoud le consensus.
\end{proof}
\end{document}
