\documentclass{beamer}
\usepackage[utf8]{inputenc}
\usepackage{amsmath}
\usepackage[noend]{sources-Heard-Of/distribalgo}
\usepackage{algorithm}
\usepackage{amssymb}
\usepackage{listings}
\usepackage{amsthm}
\usepackage{dsfont}
\usepackage{stmaryrd}

\title{Vérification et preuve formelle dans le modèle Heard-Of}
\date{8 septembre 2020}
\author{Louis Penet de Monterno \linebreak \scriptsize{stage encadré par Bernadette Charron-Bost}}
\institute{LIX}

\begin{document}

\begin{frame}
\frametitle{Problème du consensus}
\begin{description}
	\item[Intégrité :] La valeur décidée par chaque processus est l'une des valeurs initiales.
	\item[Accord :] Tous les processus décident la même valeur.
	\item[Terminaison :] Tous les processus décident en temps fini.
	\item[Irrévocabilité :] Les décisions ne sont jamais modifiées ni annulées.
\end{description}
\end{frame}

\begin{frame}
\frametitle{Modèle Heard-Of}
\end{frame}

\begin{frame}
\frametitle{Algorithme One-Third rule}
Première contribution de ce stage :
\begin{itemize}
	\item L'algorithme "One-Third rule" résout le consensus même lorsque les départs sont asynchrones.
	\item La preuve a été vérifiée par Isabelle.
\end{itemize}
\tiny{
On définit les prédicats :
\begin{itemize}
	\item $P_{S-unif-\infty} \equiv$ il existe infiniment souvent un sous-ensemble dont les membres s'entendent mutuellement exclusivement.
	\item $P_{supermaj-\infty} \equiv$ chaque processus entend infiniment souvent une supermajorité de voisins entrants.
\end{itemize}
On obtient alors le résultat suivant :

	\textbf{Théorème : } L'algorithme "One-Third rule" résout  le consensus lorsque les processus connaissent la taille du réseau,
	et lorsque les prédicats $P_{S-unif-\infty}$ et $P_{supermaj-\infty}$ sont vérifiés.
}
\end{frame}

\begin{frame}
\frametitle{Algorithmes de consensus}
Algorithmes de consensus connus :
	\begin{itemize}
		\item{\makebox[6cm][l]{Algorithme "One Third rule"} \scriptsize{Benchi, Launay, Guidec (2015)}}
		\item{\makebox[6cm][l]{Algorithme "Uniform Voting"} \scriptsize{Ben-Or (1983)}}
		\item{\makebox[6cm][l]{Algorithme "Last Voting" (Paxos)} \scriptsize{Lamport (2001)}}
	\end{itemize}
\end{frame}

\begin{frame}
\frametitle{Problème de synchronisation modulo $k$}
	\begin{description}
		\item[Sûreté : ] Les processus faisant feu le font lors de rounds qui sont congrus modulo $k$. 
		\item[Vivacité : ] Tous processus qui s'active en temps fini, fait feu en temps fini. 
	\end{description}
\end{frame}

\begin{frame}
\frametitle{Algorithme SyncMod}
	Deuxième contribution de ce stage : 
	\begin{itemize}
		\item Identification du problème de synchronisation modulo $k$.
		\item Construction de l'algorithme SyncMod.
		\item Preuve de correction de l'algorithme SyncMod, également vérifiée par Isabelle.
	\end{itemize}
\end{frame}


\begin{frame}
\frametitle{Algorithme de Coulouma\-Godard}
\end{frame}

\begin{frame}
\frametitle{Conclusion}
\end{frame}

\end{document}
