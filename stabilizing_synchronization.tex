\documentclass[11pt,letterpaper]{article}
\special{papersize=8.5in,11in}
\setlength{\parskip}{1em}

\usepackage{graphicx}
\usepackage[utf8]{inputenc}
\usepackage[pdftex,pdfpagelabels,bookmarks,hyperindex,hyperfigures]{hyperref}
\usepackage{fullpage}
\usepackage{amsmath,amssymb,amsthm,dsfont}
\newtheorem{thm}{Theorem}
\newtheorem{pro}[thm]{Proposition}
\newtheorem{lem}[thm]{Lemma}
\newtheorem{cor}[thm]{Corollary}
\newtheorem{defi}[thm]{Definition}

\title{Stabilisation in rooted communication graph}
\author{
	Bernadette Charron-Bost \\
	LIX, Palaiseau, France
\and
	Stephan Metz \\
	LORIA, Nancy, France
\and
	Louis Penet de Monterno \\
	LIX, Palaiseau, France
}
\date{\today}
\begin{document}
  \maketitle

\section{Infinite-state synchronization}

Let $\theta$ be a cut-off function.
We give the following protocol:

\begin{align}
	& $$C_i^{t+1} = 1+min_{j \looparrowright i} C_j^t$$ \label{eq:counter} \\
	& $$M_i^{t+1} = 1+M_i^t$$ \\
	& $$\forall l \in \mathds{N}, AGE^{t+1}_i[l+1] \leftarrow 1+min_{j \looparrowright i} AGE_j^t[l]$$ \label{eq:age_rec} \\
	& $$AGE_i^{t+1}[C_i^{t+1}] \leftarrow 0$$ \label{eq:age_init} \\
	& $$y_i^{t+1} \leftarrow max \{l, AGE_i^{t+1}[l] \leq \theta(M_i^{t+1})\} $$ \label{eq:result}
\end{align}

\begin{thm}
	This protocol stabilizes the synchronized clock's behaviour if the communication graph is rooted.
\end{thm}

\begin{proof}
	Let $m$ be $min\{C_u^0, u \in Roots(G)\}$.

	\noindent Claim 1:
	$\forall t > D, \forall i \in Roots(G), \forall j \in N_G, C_j^t \leq C_i^0+t$

	There exists a path $(i = i_0, i_1, \dots, i_h = j)$ from $i$ to $j$ in $G$.
	We can show by induction on this path that $\forall l \leq h, C_{i(l)}^l \leq C_i^0+l$.
	The induction case comes from the \eqref{eq:counter}.

	\noindent Claim 2:
	$\forall t > D, \forall i \in Roots(G), C_i^t = m+t$

	This claim follows from the previous claim, applied to the node $i$ satisfying $C_i^0 = m$.

	\noindent Claim 3:
	$\forall t > 2D, \forall j \in N_G, AGE^t_i[m+t] \leq D$

	Let $i$ be a node from $Roots(G)$.
	There exists a path $(i = i_0, i_1, \dots, i_h = j)$ of length $h \leq D$ from $i$ to $j$ in $G$.
	The previous claim shows that $C_i^{t-h} = m+t-h$.
	We can show by induction on this path that $\forall l \leq h, AGE_{i(l)}^{t-h+l}[m+t-h] \leq l$.
	The induction case comes from the \eqref{eq:age_rec}, and the base case from \eqref{eq:age_init}.

	\noindent Claim 4:
	$\forall t > D, \forall j \in N_G, C_j^t \leq m+t$

	This claim follows from the first claim, applied to the node $i$ satisfying $C_i^0 = m$.

	\noindent Claim 5:
	$\forall t > D, \forall j \in N_G, \forall m' > m, AGE_j^t[m'+t] \geq t-D$

	This claim can be proved by induction over $t-D$.
	For the induction case, we know from the previous claim that no node may reset $AGE_j[m'+t]$ to 0 with \eqref{eq:age_init}.
	The line \eqref{eq:age_rec} terminates the induction case.

	\noindent Claim 6:
	$\forall i \in N_G, \exists T \in \mathds{N}, \forall t > T, t-D > \theta(M_i^t) > D$

	Since $\theta$ is a cut-off function, it verifies $\lim_{t \rightarrow \infty} \theta(t) = \infty$.
	Then eventually $\theta(M_i^t) = \theta(M_i^0+t) > D$.
	Moreover $\lim_{t \rightarrow \infty} t-\theta(t) = \infty$.
	Then $\lim_{t \rightarrow \infty} M_i^t - \theta(M_i^t) = \infty$.
	Eventually $t-D > \theta(M_i^t)$.

	\noindent Claim 7:
	$\forall i \in N_G, \exists T \in \mathds{N}, \forall t > T, y_i^t = m+t$

	This claim results from \eqref{eq:result} and claims 3, 5 and 6.
	The correctness of the protocol results from this last claim.

% \section{A detailed proof of Boldi-Vigna's protocol}

\section{Synchronization modulo $P$ with dynamic topology}

We execute the protocol (6) on a dynamic graph, assumed to be complete with delay $D$.

\begin{lem}
	For all $k \in \mathds{N}$, one of the following statements is true:
	\begin{enumerate}
		\item $\forall i \in N_G, \forall j \in N_G, C_i^{2kD} \equiv C_j^{2kD} [k]$
		\item $min_i M_i^{2kD} \geq g^k(M)$
	\end{enumerate}
\end{lem}
\begin{proof}
	For $k=0$, we have $min_i M_i^0 = M = g^0(M)$, and
	statement (ii) is true. Since statement (i) is stable, to make
	the inductive step we can assume that statement (ii) is true
	and statement (i) is false for $k$.
	But if there are two clocks
	with different values modulo $P$ at time $2kD$, then for every
	processor $i$ there is a processor $j$ that
	will apply $g$ at some point between round $2kD$ and $2kD+D$
	(if we consider a path $(i, i_1, \dots, i_{D-1} = j)$ in the time interval $[2kD, 2kD+D]$,
	we can show by contradiction that some processor on this path received non-congruent values
	at some point in the interval).
	We obtain
	$M_j^{2kD+D} \geq g(min_i M_i^{2kD}) \geq g^{k+1}(M)$.

	In $D$ more steps, this lower bound will propagate to all processes, that is
	$min_i M_i^{2(k+1)D} \geq g^{k+1}(M)$
\end{proof}

\begin{thm}
	The protocol (6) stabilizes on a modulo $P$-synchronized clock's behaviour if the dynamic graph is complete with delay $D$.
\end{thm}
\begin{proof}
	Let $k = g^*(D)$.
	At time $2kD$ either clocks are equal
	modulo $P$ , or by Lemma 9 we have $\forall i \in N_G, M_i^{2kD} \geq g^k(M)$.
	In this case,

	\begin{align*}
		P(M_i^{2kD})^2 - C_i^{2kD} & \geq P(g^k(M))^2-(PM^2+2kD) \\
		& \geq P(g^k(M)-M)(g^k(M)+M)-2kD \\
		& \geq Pg^k(0)(g^k(0)+2M)-2kD \\
		& \geq PD(D+2)-2g^*(D)D \\
		& \geq 2D(D+2)-2D^2 > D
	\end{align*}

	Thus, since the guesses are nondecreasing, between time $2kD$
	and $2kD + D$ clock values will not "wrap", so by Proposition 1
	at time $kD + D$ all clocks will have value $min_i C_i^{2kD}+D$.
\end{proof}

Now we assume that the graph is rooted with delay $D$.
Using the proof of \ref{thm:fixed_rooted}, and the previous lemma, we can show the following theorem.

\begin{thm}
	The protocol (6) stabilizes on a modulo $P$-synchronized clock's behaviour if the dynamic graph is rooted with delay $D$.
\end{thm}

\section{Synchronization modulo $P$ with fixed roots in dynamic graphs}

We assume that the communication topology is given by a dynamic graph $\mathds{G}$
such that for any $i \in \mathds{N}$, the set $Roots(\mathds{G}_i)$ is a constant set noted $Roots(\mathds{G})$.
We assume that $\mathds{G}$ is centerd with delay $D$.

\begin{lem}
	If $Roots(\mathds{G})$ is already stabilized in round 0, then
	for all $k \in \mathds{N}$, for all $i \in N_\mathds{G}$,
	one of the following statements is true:
	\begin{enumerate}
		\item $C_i^{kD} \equiv C_{Roots(\mathds{G})}^{kD} [k]$
		\item $M_i^{(k+1)D} \geq g^{k+1}(M)$
	\end{enumerate}
	where $C_{Roots(\mathds{G})}^t \in \mathds{Z}/P\mathds{Z}$ is the common counter value of the nodes in $Roots(\mathds{G})$ in round $t$.
\end{lem}
\begin{proof}
	We prove this lemma by induction over $k$.

	For $k = 0$, we fix $i \in N_\mathds{G}$, and we assume that $i$ is not synchronized with the roots.
	There must existe a path $j = j(0), j(1), \dots, j(D) = i$ between $j$ and $i$ in the round interval $[0, D]$.
	Let $j(l)$ be the first node on this path such that $C_{j(l)}^l \not\equiv C_{j(l+1)}^l [k]$.
	We obtain $M_{j(l+1)}^{l+1} = g(max_{u \looparrowright j(l+1)} M_u^l) \geq g(M)$.
	We finally have $M_i^D \geq g(M)$.

	For the induction case, we assume that a fixed node $i$ is not synchronized with $Roots(\mathds{G})$ in round $(k+1)D$.
	We also assume that $\forall i \in N_\mathds{G}, C_i^{(k+1)D} \not\equiv C_{Roots(\mathds{G})} \Rightarrow M_i^{(k+1)D} \geq g^{k+1}(M)$.
	There must existe a path $j = j(0), j(1), \dots, j(D) = i$ between $j \in N_{Roots(\mathds{G})}$ and $i$ in the round interval $[(k+1)D, (k+2)D]$.
	Let $j(l)$ be the first node on this path such that $C_{j(l)}^{(k+1)D+l} \neg\equiv C_{j(l+1)}^{(k+1)D+l+1} [k]$.
	There must be a path from a node $u$ to $j(l+1)$ in the round interval $[(k+1)D, (k+1)D+l+1]$ where $C_u^{(k+1)D} \not\equiv C^{(k+1)D}_{Roots(\mathds{G})}$.
	By induction hypothesis, we have $M_u^{(k+1)D} \geq g^{k+1}(M)$.
	Since there exists a path from $u$ to $i$ in the round interval $[(k+1)D, (k+2)D]$,
	we obtain $M_i^{(k+2)D} \geq g(M_u^{(k+1)D}) \geq g^{k+2}(M)$.
\end{proof}

\begin{thm}
	The protocol (6) stabilizes on a modulo $P$-synchronized clock's behaviour if the dynamic graph is centered with delay $D$.
\end{thm}
\begin{proof}
	\noindent $Roots(\mathds{G})$ is a strongly connected component. It must stabilize, according to the Boldi-Vigna's proof.
	Without loss of generality, we assume that $Roots(\mathds{G})$ has already stabilized in round 0.

	\noindent Claim 1: In round $D$, all nodes in $N_{Roots(\mathds{G})}$ hold the same value $M$ as their $M_i^D$ variable, and will never adopt a different value.
	Moreover $\forall i \in N_\mathds{G}, M_i^t \geq M$.

	\noindent To prove this claim, we consider the node $i \in N_{Roots(\mathds{G})}$ holding the maximum $M_i^0$ value among $Roots(\mathds{G})$.
	Its value must have propagated to the whole network in round $D$.
	Let $k = min \{l \in \mathds{N}, g^l(M) \geq M+\frac{2D}{P}\}$.

	\noindent Claim 2: At the latest in round $kD+D$ and beyond, we have $\forall i \in N_\mathds{G}, C_i^t \equiv C_{Roots(\mathds{G})}^t [P] \vee PM_i^t \geq PM+2D$

	\noindent This results from the previous lemma.

	\noindent Claim 3: $\forall i \in N_\mathds{G}, \forall l \geq D, C_i^l \leq PM+D$.

	\noindent The counters in $Roots(\mathds{G})$ are upper-bounded by $PM$.
	Moreover there always exists a path from $Roots(\mathds{G})$ to $i$.
	This claim can be proved by induction over this path.

	\noindent Claim 4: At the latest in round $kD+D+PM$, some node $i \in N_{Roots(\mathds{G})}$ reaches $C_i^t = 0$.

	\noindent This results from the fact that as long as no node from $Roots(\mathds{G})$ rewinds to 0,
	$min_{i \in N_{Roots(\mathds{G})}} C_i^t$ grows by 1 at each round. Moreover it is bound by $PM$.

	\noindent Let $t$ be a round which verifies this claim.

	\noindent Claim 5: $\forall i \in N_\mathds{G}, \forall l \geq D, C_i^{t+l} \leq l$.

	This can be proved by induction over the path from the node $j$ reaching $C_j^t = 0$ to $i$ in the round interval $[t, t+l]$.

	\noindent Claim 6: For any $D \geq l \geq PM+2D$, the global state verifies $\forall i \in N_\mathds{G}, C_i^{t+l} \not\equiv C_{Roots(\mathds{G})}^{t+l} \Rightarrow C_i^{t+l} \geq l-D$.

	\noindent This claim can be proved by induction over $l-D$, using claim 2 and 5 to rule out a possible rewind to 0.

	\noindent Claim 7: At the latest in round $kD+D+PM+PM+2D$, the system will be synchronized.

	\noindent This results from claim 3 and 6.
\end{proof}
\end{document}
