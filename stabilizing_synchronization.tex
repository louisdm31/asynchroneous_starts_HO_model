\documentclass[11pt,letterpaper]{article}
\special{papersize=8.5in,11in}
\setlength{\parskip}{1em}

\usepackage[utf8]{inputenc}
\usepackage[pdftex,pdfpagelabels,bookmarks,hyperindex,hyperfigures]{hyperref}
\usepackage{fullpage}
\usepackage{amsmath,amssymb,amsthm,dsfont}
\newtheorem{thm}{Theorem}
\newtheorem{pro}[thm]{Proposition}
\newtheorem{lem}[thm]{Lemma}
\newtheorem{cor}[thm]{Corollary}
\newtheorem{defi}[thm]{Definition}

\title{Stabilisation in rooted communication graph}
\author{Bernadette Charron-Bost\textsuperscript{1} \and Louis Penet de Monterno\textsuperscript{1}}
\date{\textsuperscript{1} \'Ecole polytechnique, 91128 Palaiseau, France\\~\\ \today}

\begin{document}
  \maketitle

\section{Infinite-state synchronization}

Let $\theta$ be a cut-off function.
We give the following protocol:
\begin{align}
	& $$C_i^{t+1} = 1+min_{j \looparrowright i} C_j^t$$ \label{eq:counter} \\
	& $$M_i^{t+1} = 1+M_i^t$$ \\
	& $$\forall l \in \mathds{N}, AGE^{t+1}_i[l+1] \leftarrow 1+min_{j \looparrowright i} AGE_j^t[l]$$ \label{eq:age_rec} \\
	& $$AGE_i^{t+1}[C_i^{t+1}] \leftarrow 0$$ \label{eq:age_init} \\
	& $$y_i^{t+1} \leftarrow max \{l, AGE_i^{t+1}[l] \leq \theta(M_i^{t+1}\}) \label{eq:result} $$
\end{align}

\begin{thm}
	This protocol stabilizes the synchronized clock's behaviour if the communication graph is rooted.
\end{thm}

\begin{proof}
	Let $m$ be $min\{C_u^0, u \in Roots(G)\}$.

	\noindent Claim 1:
	$\forall t > D, \forall i \in Roots(G), \forall j \in N_G, C_j^t \leq C_i^0+t$

	There exists a path $(i = i_0, i_1, \dots, i_h = j)$ from $i$ to $j$ in $G$.
	We can show by induction on this path that $\forall l \leq h, C_{i(l)}^l \leq C_i^0+l$.
	The induction case comes from the \eqref{eq:counter}.

	\noindent Claim 2:
	$\forall t > D, \forall i \in Roots(G), C_i^t = m+t$

	This claim follows from the previous claim, applied to the node $i$ satisfying $C_i^0 = m$.

	\noindent Claim 3:
	$\forall t > 2D, \forall j \in N_G, AGE^t_i[m+t] \leq D$

	Let $i$ be a node from $Roots(G)$.
	There exists a path $(i = i_0, i_1, \dots, i_h = j)$ of length $h \leq D$ from $i$ to $j$ in $G$.
	The previous claim shows that $C_i^{t-h} = m+t-h$.
	We can show by induction on this path that $\forall l \leq h, AGE_{i(l)}^{t-h+l}[m+t-h] \leq l$.
	The induction case comes from the \eqref{eq:age_rec}, and the base case from \eqref{eq:age_init}.

	\noindent Claim 4:
	$\forall t > D, \forall j \in N_G, C_j^t \leq m+t$

	This claim follows from the first claim, applied to the node $i$ satisfying $C_i^0 = m$.

	\noindent Claim 5:
	$\forall t > D, \forall j \in N_G, \forall m' > m, AGE_j^t[m'+t] \geq t-D$

	This claim can be proved by induction over $t-D$.
	For the induction case, we know from the previous claim that no node may reset $AGE_j[m'+t]$ to 0 with \eqref{eq:age_init}.
	The line \eqref{eq:age_rec} terminates the induction case.

	\noindent Claim 6:
	$\forall i \in N_G, \exists T \in \mathds{N}, \forall t > T, t-D > \theta(M_i^t) > D$

	Since $\theta$ is a cut-off function, it verifies $\lim_{t \rightarrow \infty} \theta(t) = \infty$.
	Then eventually $\theta(M_i^t) = \theta(M_i^0+t) > D$.
	Moreover $\lim_{t \rightarrow \infty} t-\theta(t) = \infty$.
	Then $\lim_{t \rightarrow \infty} M_i^t - \theta(M_i^t) = \infty$.
	Eventually $t-D > \theta(M_i^t)$.

	\noindent Claim 7:
	$\forall i \in N_G, \exists T \in \mathds{N}, \forall t > T, y_i^t = m+t$

	This claim results from \eqref{eq:result} and claims 3, 5 and 6.
	The correctness of the protocol results from this last claim.

\section{Synchronization modulo $P$}

	We consider the protocol (6) from Boldi-Vigna's paper.

\begin{thm}
	The protocol (6) stabilizes on a modulo $P$-synchronized clock's behaviour if the communication graph is rooted.
\end{thm}
\begin{proof}
	Let $D$ be the diameter of widest connected component of $G$.
	We consider the condensation graph of $G$.
	This graph is well-founded, so we can prove by induction over its node that the underlying components ultimately stabilize.

	For the base base, we have to prove that the protocol (6) stabilizes on $Roots(G)$.
	The proof is done in Boldi-Vigna's paper.

	For the induction case, we consider $G'$ a strongly connected component from $G$.
	By induction hypothesis, we know that every node in the incoming components ultimately stabilize.
	Without loss of generality, we assume that they are already stabilized in the round 0.
	Let $i(1), \dots, i(h)$ be the incoming neighbours of $G'$.
	Let $M'$ be $max \{M_{i(l)}^0, 1 \leq l \leq h\}$.
	Since the incoming connected components are already stabilized, their nodes ultimately adopt the highest $M_i^0$ possible,
	and never increase their value.
	Without loss of generality, we also assume that $M^0_{i(1)}, \dots, M^0_{i(h)}$ are already stabilized.
	Let $k$ be $g^*(M'+D)$.

	For any $i \in N_{G'}$, we have:
	\begin{align*}
		P(M_i^{kD})^2 - C_i^{kD} & \geq Pg^k(0)(g^k(0)+2M)-kD \\
		& \geq P(M'+D)(M'+D+2M)-(M'+D)D \\
		& \geq PM'^2+D^2+2DM'-(M'+D)D \\
		& \geq PM'^2+D
	\end{align}

	Let $V_{kD}$ be the set $\{C_i^{kD}, i \in N_{G'}\}$.
	In round $kD+PM'^2$, the nodes from $G'$ either have adopted a counter value from $i(1), \dots, i(h)$,
	or keep a counter value from $V_{kD}$.
	Moreover, the previous inequality guarantees that the values from $V_{kD}$ will not "wrap" between round $kD$ and $kD+PM'^2+D$.
	Then $V_{kD+PM'^2} = \{C_{i(1)}^{kD+PM'^2} + rP, r \in \mathds{N}\} \cup \{v+PM'^2, v \in V_{kD}\}$.

	In round $kD+PM'^2$, if a node $i \in N_{G'}$ is not synchronized with $i(1), \dots, i(h)$,
	its counter value must be greater than $PM'^2$.
	Whereas the counter values of $i(1), \dots, i(h)$ are bounded by $PM'^2$.
	This shows that every node in $G'$ will ultimately synchronize with $i(1), \dots, i(h)$.
	This terminates the induction case.
\end{proof}

\end{document}
