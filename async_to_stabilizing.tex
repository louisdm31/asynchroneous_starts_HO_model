\documentclass[11pt,letterpaper]{article}
\special{papersize=8.5in,11in}

\usepackage[utf8]{inputenc}
\usepackage[ruled, linesnumbered, noline]{algorithm2e}
\usepackage[pdftex,pdfpagelabels,bookmarks,hyperindex,hyperfigures]{hyperref}
\usepackage{fullpage}
\usepackage{amsmath,amssymb,amsthm,dsfont}
\usepackage{graphicx}
\newtheorem{thm}{Theorem}
\newtheorem{pro}[thm]{Proposition}
\newtheorem{lem}[thm]{Lemma}
\newtheorem{cor}[thm]{Corollary}
\newtheorem{defi}[thm]{Definition}

\title{Asynchronous starts and self-stabilization, one tail for two models}
\author{
	Bernadette Charron-Bost \\
	LIX, Palaiseau, France
\and
	Stephan Merz \\
	LORIA, Nancy, France
\and
	Louis Penet de Monterno \\
	LIX, Palaiseau, France
}
\date{\today}

\begin{document}
  \maketitle

This document defines a reduction from a asynchronous-starts model to the self-stabilizing models:
given an algorithm written in the asynchronous-starts framework, we construct a self-stabilizing algorithm which solves the same problem.
These models (formally defined in first two sections) are widely used in distributed computing.
The set of possible input values is noted $C$, whereas $Y$ is the set of possible output values.

\section{Asynchronous starts model}

We consider an algorithm $A$ in the asynchronous starts model.
This algorithm is defined by the following data:

\begin{itemize}
	\item A set of states $X$.
	\item A subset of initial state $X_{init} \in X$.
	\item A transition function $\delta^c : X \times X^\oplus \rightarrow X$ where the $\oplus$ operator constructs the set of possible multi-sets and $c \in C$.
\end{itemize}

\noindent An execution in this model is defined by:
\begin{itemize}
	\item An activation schedule $(s_i)_{i \in \Pi} \in (\mathds{N} \cup \{\infty\})^\Pi$.
	\item An input vector $(c_i)_{i \in \Pi} \in C^\Pi$.
	\item A family of initial states $(q_i)_{i \in \Pi} \in X_{init} \subseteq X$.
	\item A graph $\mathds{G} = (\Pi, E_t)$ where $\forall t \in \mathds{N}, E_t \subseteq \Pi \times \Pi$.
\end{itemize}

\noindent The states of the nodes is inductively derived from this data by stating $\forall i \in \Pi, x_i^0 = x_{passive}$ and:

$$x_i^{t+1} = \left \{ \begin{array}{l l} x_{passive} & \mbox{if}~t < s_i \\
	x_{init} & \mbox{if}~t = s_i \\
\delta^{c_i}(x_i^t, \{x_j^t, (j,i) \in E_{t+1}\}) & \mbox{otherwise} \end{array} \right.$$

\section{Self-stabilizing model}

We now consider an algorithm $A$ in the self-stabilizing model.
This algorithm is defined by the following data:

\begin{itemize}
	\item A set of states $X$.
	\item A transition function $\delta^c : X \times X^\oplus \rightarrow X$.
\end{itemize}

\noindent An execution in this model is defined by:
\begin{itemize}
	\item A family of initial states $(x_i^0)_{i \in \Pi}$ where $x_i^0$ is an arbitrary state in $X$.
	\item An input vector $(c_i)_{i \in \Pi} \in C^\Pi$.
	\item A graph $\mathds{G} = (\Pi, E_t)$ where $\forall t \in \mathds{N}, E_t \subseteq \Pi \times \Pi$.
\end{itemize}

The state of the nodes is inductively derived from this data by stating:

$$x_i^{t+1} = \delta^{c_i}(x_i^t, \{x_j^t, (j,i) \in E_{t+1}\}).$$

Although the initial state of a node in this model is arbitrary, its input value is viewed as a static value,
that the node can access in every round. That's why the transition function is indexed by $c$.

\section{Stabilizing problems}

We consider $\mathcal{C} \subseteq \mathcal{G} \times C^\Pi$, where $\mathcal{G}$ is the set of possible dynamic graphs.
We consider a function $Out : X \rightarrow Y$ which returns the output of a node
(for example, in leader election problem, $Y = \{0,1\}$, and the leader must ultimately return 1, while non-leaders must return 0).
We define the specification of the problem as a mapping from any tuple $\mathds{G}, (c_i)_{i \in \Pi} \in \mathcal{C}$
to a suffix-closed subset $S_{\mathds{G}, (c_i)_{i \in \Pi}} \subseteq (Y^\Pi)^\omega$.
$S_{\mathds{G}, (c_i)_{i \in \Pi}}$ is a set of sequences of output vectors.
We say that an algorithm $A$ (in both presented models) \textit{stabilizes} to $S$ if, for any tuple $\mathds{G}, (c_i)_{i \in \Pi} \in \mathcal{C}$,
for any execution of $A$ in this graph and with this input, there exists some bound $T \in \mathds{N}$ such that we have:

$$\langle Out(x^T_i), i \in \Pi \rangle, \langle Out(x^{T+1}_i), i \in \Pi \rangle, \dots \in S_{\mathds{G}, (c_i)_{i \in \Pi}}$$

where $\langle Out(x^T_i), i \in \Pi \rangle \in Y^\Pi$ is the vector of outputs of the nodes in round $T$.
We will refer to $T$ as the \textit{quiescence time}.
For the leader election problem, the specification would look like:

Here, each line corresponds to one of the possible output (i.e. one line for each possible leader).
This problem do not require input values.
Every execution which stabilizes to some leader will have a suffix in this set.

\section{Translation in self-stabilizing model}

We assume we have an asynchronous-start algorithm $A$ which stabilizes to some specification $S$.
The aim of this document is to construct an self-stabilizing algorithm $Stab(A)$ which stabilizes to $S$.

$Stab(A)$ emulates a growing number of instances of $A$, stored in a variable $INST$.
The set of states of $Stab(A)$ is the set of finite sequences of $X$.
At each rounds, each node:
\begin{enumerate}
	\item updates its existing instances by applying the transition function of $A$,
	\item creates a new instance by appending the initial state $x_{init}$ to its $INST$ variable,
	\item picks some instance, according to some function $g$, and returns its output value.
\end{enumerate}

The operator $|\cdot|$ is used to get the length of the sequence.
Initially, every node holds an arbitrary sequence of instances, of various length (see figure \ref{fig:fig1}), and the sequence of instances grows in each round (see figure \ref{fig:fig2}).

\begin{figure}[h!]
	\includegraphics[width=0.8\textwidth]{images/instances-1}
	\caption{Initial state}
	\label{fig:fig1}
\end{figure}

\begin{figure}[h!]
	\includegraphics[width=0.8\textwidth]{images/instances-2}
	\caption{Fourth round}
	\label{fig:fig2}
\end{figure}

You can see on figure \ref{fig:fig2} that the first four instances are (at least in one node) inherited from the initial state.
In the self-stabilizing model, the initial state is arbitrary, whereas the algorithm $A$ is supposed to be correct only if every node has been initialized by $x_{init}$.
As a consequence, the output of these early instances is not reliable.
In opposition, the fifth instance and every subsequent instances are initialized by line \ref{line:create}
(in figure \ref{fig:fig2}, the node $i$ initialize its fifth instance in round 1, whereas $j$ initialize its in round 3).
Since the algorithm $A$ is assumed to tolerate asynchronous starts, we know that the latest instances will eventually stabilize to the expected output vector.
We make the additional assumption that the nodes stabilize \textit{within bounded delay} (more details in next section).

\begin{figure}[h!]
	\includegraphics[width=0.8\textwidth]{images/instances-3}
	\caption{Fifteenth round}
	\label{fig:fig3}
\end{figure}

In the long run (see figure \ref{fig:fig3}), a finite number of early instances are unreliable, whereas a finite number of late instances have not stabilized yet.
There is a growing segment in the middle of the sequences (bold green instances on the figure \ref{fig:fig3}) which have already stabilized to the expected output vector.
We just need to pick one of those instances, and returns its output value.
That's why we parameter $Stab(A)$ by a function $g : \mathds{N} \rightarrow \mathds{N}$ verifying:

\begin{itemize}
	\item $\lim_{t \rightarrow \infty} g(t) = \infty.$
	\item $\lim_{t \rightarrow \infty} t-g(t) = \infty.$
\end{itemize}

\begin{algorithm}[h!]
	\DontPrintSemicolon
	\textbf{Variables:} \;
	\Indp
		$c_i \in C$ \tcc*[f]{input variable} \;
		$INST_i \in X^*$ \tcc*[f]{state variable} \;
		$y_i \in Y$ \tcc*[f]{output variable} \;
	\BlankLine
	\Indm
	\textbf{At each round:} \;
	\Indp
		\For{$l \in \{0, \dots, |INST_i| \}$}{
			$INST_i[l] \leftarrow \delta^{c_i}(INST_i[l], \{INST_j[l], (j,i) \in E_t \wedge |INST_j| \geq l\})$ \label{line:update}
		}
		$INST_i \leftarrow INST_i + [x_{init}]$ \tcc*[f]{append new instance} \label{line:create} \;
		$y_i \leftarrow Out(INST_i[g(|INST_i|)])$ \label{line:output} \;
	\Indm
	\caption{The $Stab(A)$ algorithm} 
\end{algorithm}

\section{Proof}

We consider an execution of $Stab(A)$, defined by $\mathds{G}$, the input vector $(c_i)_{i \in \Pi}$ and some initial sequences $(INST_i)_{i \in \Pi}$.
We consider the expected output vector $v \in Y^\Pi$ (i.e. such that $S_{\mathds{G}, (c_i)_{i \in \Pi}} = \{(v, v, v, \dots)\}$).
We want to prove that the output of $Stab(A)$ converges to $v$.
We assume that $A$ self-stabilizes \textit{with bounded delay}, that is, for every execution,
$T-max\{s_i, i \in \Pi\}$ is bounded by some natural number $T_{max}$.
In other words, once every node is active, $A$ must return the correct output at most within $T_{max}$ rounds.

\begin{lem}
	For any $l \geq max_{i \in \Pi} |INST_i^0|$, we have:
	$$\forall i \in \Pi, \forall t \geq T_{max}+l+1, Out(INST_i^t[l]) = v_i.$$
\end{lem}
\begin{proof}
	We can prove by induction over $t$ that, for any $j$ verifying $\forall i \in \Pi, l \geq |INST_i^0|$,
	the $Stab(A)$ algorithm emulates an execution of $A$ by executing the transition function of $A$ at each round.
	The activation schedule of the instance $l$ of $i$ is given by $s_i = l-|INST_i^0|+1$.
	Since $A$ is assumed to self-stabilize with bounded delay, we know that the output of the instances converge to $v$ from some round $T$ where
	$$T \leq T_{max} + max\{s_i, i \in \Pi\} \leq T_{max} + l + 1.$$
\end{proof}

\begin{thm}
	If $A$ stabilizes to $S$ with bounded delay, then $Stab(A)$ stabilizes to $S$.
\end{thm}
\begin{proof}
	According to previous lemma, for any $i \in \Pi$, in any round $t \in \mathds{N}$, we have
	$$l \in [max_{i \in \Pi} |INST_i^0|, t-T_{max}-1] \Rightarrow Out(INST_i^t[l]) = v_i.$$
	Because $\lim_{t \rightarrow \infty} g(t) = \infty$, we know that $g(|INST_i|)$ ultimately exceeds $max_{i \in \Pi} |INST_i^0|$.
	Because $\lim_{t \rightarrow \infty} t-g(t) = \infty$, we know that $g(|INST_i|)$ is ultimately below $t-T_{max}-1$.
	That proves the stabilization property.
\end{proof}

\end{document}
