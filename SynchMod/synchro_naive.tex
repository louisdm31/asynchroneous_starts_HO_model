
\documentclass{article}
\usepackage{amsmath}
\usepackage[noend]{libHO/distribalgo}
\usepackage{algorithm}
\usepackage{amssymb}
\usepackage{listings}
\usepackage{amsthm}
\usepackage{dsfont}
\usepackage{stmaryrd}
\usepackage[left=2cm, right=2cm, top=2cm]{geometry}
\usepackage[utf8]{inputenc}


\newtheorem{lemma}{Lemma}
\newtheorem{theorem}{Theorem}
\newtheorem{definition}{Definition}
% \usepackage{biblatex}
% \addbibresource{rapport.bib}

\title{Naive synchronization algorithm}
\date{August 2020}
\author{Louis Penet de Monterno - Bernadette Charron-Bost}

\begin{document}

\maketitle

\section{Problem}

Our goal is to solve the non-uniform synchronization problem, defined below.
We consider a system composed of a set of $\Pi$ nodes.
The starts are asynchronous.  A subset $S \subseteq \Pi$ is the subset of correct nodes.
A correct node never crash, and any messages sent by $u \in S$ always reach their destination.
An incorrect node $u \in \Pi \setminus S$ which crashes in round $t$ must :
\begin{itemize}
	\item be correct until round $t-1$. All of its outgoing messages reach their destination.
	\item fail in round $t$ : only a subset of its outgoing messages reach their destination.
	\item be quiet in round $t+1$ and later.
\end{itemize}

We define the non-uniform synchronization problem as the conjunction of two properties:
\begin{description}
	\item[Safety:] If a correct node fires in round $t$, every correct node do so.
	\item[Liveness:] At least one correct node fires.
\end{description}

\section{Algorithm}

The algorithm below is designed to solve this problem when the following assumptions are verified:
\begin{itemize}
	\item No node remains passive forever.	
	\item Every node knows a unique identifier.
\end{itemize}

The general idea is the following: in order to fire, a node waits that every node is active. Any received null prevents a node from firing.
Moreover, when a passive node crashes, this crash may be detected by correct nodes within a range of two rounds (this detection is possible using the $passiv$ variable,
which tracks the set of heard-of passive nodes at each rounds).
The crash of an active node does not matter.
When a node detects a crash, it appends this node to its $crashed$ variable.
In the next round, it sends its $crashed$ variable to all.
Using the received values, a node can can construct a variable $justcrashed$ representing the set of nodes which crashed during the previous round.
They can determine whether a detected crash just happened, or happened during the previous round.
As soon as every passive node has become active, and no crash happened during the two last rounds, every node can fire.

The $activating$ variable prevents a firing while a node is activating, as that could cause safety issues.

\begin{algorithm}[htb]\label{algo:code}
\begin{distribalgo}[1]
\BLANK \INDENT{\textbf{Initialization:}}
	% \STATE $correct_u \leftarrow \Pi$
	\STATE $crashed_u \leftarrow \emptyset$
	\STATE $oldcrash_u \leftarrow \emptyset$
	\STATE $passiv_u \leftarrow \Pi$
	\STATE $activating_u \leftarrow false$

\ENDINDENT \BLANK

\INDENT{\textbf{In each round $t$:}}
	\INDENT{$S_u^t$}
		\IF{$passiv_u = \Pi$}
			\STATE send $\bot$ to all
		\ELSE
			\STATE send $\langle crashed_u \rangle$ to all
		\ENDIF
	\ENDINDENT \BLANK
	\INDENT{$T_u^t$}
		\STATE \textbf{let} $justcrashed = \bigcup\limits_{v \in HO(u,t)} crashed_v(t-1) \setminus oldcrash_u$  \label{line:detected-before}
		\STATE $crashed_u \leftarrow passiv_u \setminus HO(u, t) \setminus justcrashed$ \label{line:detect}
		\IF{no null, no $\bot$ received and $\neg activating_u$ and $crashed_u = \emptyset$ and $justcrashed = \emptyset$}
			\STATE fire and exit
		\ENDIF
		\STATE $passiv_u \leftarrow $ set of node which sent null
		\STATE $activating_u \leftarrow $ exists a node which sent $\bot$
	\ENDINDENT 
\ENDINDENT 

\caption{The non-uniform firing-squad algorithm} \label{algo:R}
\end{distribalgo}

\end{algorithm}

\section{Proof}

\begin{lemma} \label{lem:passiv-wait}
	If a node is passive and still correct in round $t$, no node can fire in round $t+1$ or earlier.
\end{lemma}
\begin{proof}
	In round earlier than $t+1$, every node receives a null, and must refrain from firing.

	If $w$ is passive and correct until round $t$, then at least a subset $I \subseteq \Pi$ receives null or $\bot$ in round $t+1$.
	Those nodes cannot fire at round $t+1$.

	Other nodes received null from $w$ in round $t$ and do not receive any message from $w$ in round $t+1$.
	They must include $w$ in their $crashed$ variable. In round $t+1$, they will receive their own message containing $w$, and will refrain from firing.
\end{proof}

\begin{lemma} \label{lem:split}
	If a passive node $w$ crashes in round $t$, then:
	\begin{itemize}
		\item In round $t$, a subset $I_1 \subseteq \Pi$ will detect that failure, and will include $w$ into their $crashed$ variable.
			In the next round, those nodes will send to all a message containing $w$.
		\item In round $t+1$, the subset $\Pi \setminus I_1$ will detect that failure. This set of nodes will be partitioned into two subset $I_2$ and $\overline{I}_2$.
		\begin{itemize}
			\item The nodes in $I_2$ will receive a message containing $w$ sent by a node in $I_1$. Those nodes will not include $w$ in their $crashed$ variable
				(see lines \ref{line:detected-before} and \ref{line:detect}).
				They will not send a message containing $w$ in round $t+2$ either.
			\item The nodes in $\overline{I}_2$ will not receive a message containing $w$ sent by a node in $I_1$. Those nodes will include $w$ in their $crashed$ variable
				(see lines \ref{line:detected-before} and \ref{line:detect}).
				They will also send a message containing $w$ in round $t+2$.
		\end{itemize}
	\end{itemize}
\end{lemma}

\begin{lemma} \label{lem:no-fire}
	If a passive node $w$ crashes in round $t$, no node can fire in round $t+1$.
\end{lemma}
\begin{proof}
	The previous lemma splits the systems into three subsets relative to the failure of $w$.
	\begin{itemize}
		\item The nodes in $I_1$ cannot fire because in round $t+1$. Indeed in round $t$, they will send a message containing $w$.
			Hence, in round $t+1$, they will receive their own message, and their variable $justcrashed$ will contain $w$ (see line \ref{line:detected-before}).
		\item The nodes in $I_2$ cannot fire because in round $t+1$ because they will receive a message containing $w$,
			then $w$ will belong to their $justcrashed$ variable (see line \ref{line:detected-before}).
		\item The nodes in $\overline{I}_2$ cannot fire because in round $t+1$ because $w$ will belong to their $crashed$ variable (see line \ref{line:detect}).
	\end{itemize}
\end{proof}

\begin{lemma} \label{lem:no-fire-activating}
	If a node $w$ crashes in round $t$ while it was getting activated, no node can fire in round $t+1$.
\end{lemma}
\begin{proof}
	The system can be partitioned into two subset $I$ and $\overline{I}$:
	\begin{itemize}
		\item The nodes in $\overline{I}$ receive the $\bot$ message sent by $w$. They set their $activating$ variable to $true$, and do not fire in round $t+1$.
		\item The nodes in $I$ do not receive the $\bot$ message sent by $w$. They instead add $w$ to their $crashed$ variable, and send it to all.
			In round $t+1$, they must receive their own message containing $w$. They add $w$ to their $justcrashed$ variable, and refrain from firing.
	\end{itemize}
\end{proof}

\begin{theorem}
	The non-uniform firing-squad algorithm solves non-uniform synchronization problem.
\end{theorem}
\begin{proof}
	We first prove safety. We assume that a correct node $u$ fires in round $t$.
	By contradiction, we assume that a correct node $v$ does not fire in round $t$.
	Then, one of the following cases apply:
	\begin{itemize}
		\item $v$ received a null (or a $\bot$) from a node $w$.
			The lemma \ref{lem:passiv-wait} contradict the fact that $u$ fired.
		\item The variable $activating_v(t)$ is true.
			In that case, the lemma \ref{lem:no-fire-activating} contradict the fact that $u$ fired.
		\item $crashed_v(t)$ contains a node $w$. This node crashed either in round $t-1$ or in round $t$.
			In the first case, the lemma \ref{lem:no-fire} or \ref{lem:no-fire-activating} contradict the fact that $u$ fired.
			In the second case, the lemma \ref{lem:passiv-wait} contradict the fact that $u$ fired.
		\item The variable $justcrashed_v(t)$ is not empty. Then $v$ received a non-empty message from $z$ containing $w$ in round $t$. 
			This must result from a crash of node $w$ at round $t-1$ or $t-2$.
			In the first case, the lemma \ref{lem:no-fire} or \ref{lem:no-fire-activating} contradict the fact that $u$ fired.
			The second case is tricky. We derive a contradiction from a series of claims:
			\begin{enumerate}
				\item $w$ was passive in round $t-2$.
					Indeed, if we assume that $w$ was activating in round $t$, the proof of lemma \ref{lem:no-fire-activating} splits the system into $I$ and $\overline{I}$.
					However, neither a node from $I$, nor a node from $\overline{I}$ would have sent a message containing $w$ in round $t$.
				\item The lemma \ref{lem:split} shows that $z$ belongs to $\overline{I}_2$.
				\item Since $\overline{I}_2$ is non-empty, no correct node can belong to $I_1$.
				\item No correct node can belong to $\overline{I}_2$ either. Otherwise, it would have sent to $v$ in round $t-1$ a message containing $w$.
					Then the $oldcrash_v(t)$ would contain $w$, and $w$ would not be included in the variable $justcrashed_v(t)$.
				\item Then $v$ must belong to $I_2$. That means that $v$ received a message containing $w$ in round $t-1$.
					Then the $oldcrash_v(t)$ contain $w$, and that contradicts the fact that $w$ is included in the variable $justcrashed_v(t)$.
			\end{enumerate}
	\end{itemize}

	The proof of liveness is straightforward. As soon as every node correct node has become active, the $passiv$ variable of each node becomes empty forever.
	In the next round, the $activating$ variable becomes false forever, and the $crashed$ variable becomes empty forever.
	In the next round, the sent messages can only be empty, and so becomes the $justcrashed$ variable. At this point, no node can refrain from firing.
\end{proof}

\section{Conclusion}

This algorithm can be used to solve the non-uniform consensus problem in an asynchronous-starts system, using the FloodSet algorithm.

\end{document}
