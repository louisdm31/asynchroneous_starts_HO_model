\documentclass[11pt,letterpaper]{article}
\special{papersize=8.5in,11in}

\usepackage{fullpage}
\usepackage{amsmath,amssymb,amsthm,dsfont}
\usepackage{tikz}
\usetikzlibrary{shapes.symbols,arrows}

\usetikzlibrary{shapes.symbols,arrows}
\usepackage{enumerate}

\usepackage{algorithm,algorithmic}

\renewcommand{\algorithmicrequire}{\textbf{Initialization:}}
\renewcommand{\algorithmicensure}{\textbf{In each round $t$ do:}}

\newcommand{\INITIALLY}{\REQUIRE{}}
\newcommand{\ROUND}{\ENSURE{}}


\DeclareMathOperator{\NC}{NC}
\DeclareMathOperator{\NCL}{NCL}
\DeclareMathOperator{\lcm}{lcm}
\DeclareMathOperator{\Rem}{Rem}
\DeclareMathOperator{\Simp}{Simp}
\DeclareMathOperator{\Crit}{Crit}
\DeclareMathOperator{\Extr}{Extr}
\DeclareMathOperator{\Closed}{Closed}
\DeclareMathOperator{\Cyc}{Closed}
\DeclareMathOperator{\Step}{Step}
\DeclareMathOperator{\Red}{Red}
\DeclareMathOperator{\Inc}{Inc}
\DeclareMathOperator{\Start}{Start}
\DeclareMathOperator{\End}{End}
\DeclareMathOperator{\Edges}{Edges}
\DeclareMathOperator{\Sections}{Sections}
\newcommand{\CDD}{{cd}}
\newcommand{\EP}{{ep}}
\newcommand{\CF}{{cr}}
\newcommand{\nc}{{\mathrm{ nc}}}
\newcommand{\In}{{\mathrm {In}}}
\newcommand{\legendre}[2]{\genfrac{(}{)}{}{}{#1}{#2}}
\newcommand{\cT}{{ (c,T)  }}
\newcommand{\tY}{\tilde{Y}}
\newcommand{\oY}{\overline{Y}}

\newcommand{\IR}{\mathds{R}}
\newcommand{\IN}{\mathds{N}}
\newcommand{\IZ}{\mathds{Z}}
\newcommand{\IQ}{\mathds{Q}}
\newcommand{\IC}{\mathds{C}}
\newcommand{\M}{\mathcal{M}}

\newcommand{\Bcnc}{A^{\Delta}_{\mathrm {c}}}
\newcommand{\Bunu}{\tilde{B}}
\newcommand{\Bms}{\tilde{B}}
\newcommand{\Bep}{A^{\Delta}_\mathrm{e}}
\newcommand{\Benp}{A^{\Delta}_\mathrm{e}}
\newcommand{\Bneone}{A^{\Delta}_\mathrm{r}}
\newcommand{\Bnetwo}{A^{\Delta}_\mathrm{r}^\mathrm{HA}}

\newcommand{\SCC}{\mathcal{C}}
\newcommand{\CP}{\mathcal{P}_\ccirclearrowleft}

\newcommand{\IRmax}{\overline{\IR}}
\newcommand{\IRmin}{{\IR}_{\min}}
\newcommand{\wstar}{{w}_{*}}
\newcommand{\ito}{{i\!\to}}
\newcommand{\Pa}{{\mathcal{W}}}

\newcommand{\real}[1]{\mathbf{N}_{\geqslant {#1}}}
\newcommand{\realrem}[2]{\mathbf{N}_{\geqslant {#1}}^{({#2})}}

\renewcommand{\le}{\leqslant}
\renewcommand{\leq}{\leqslant}
\renewcommand{\ge}{\geqslant}
\renewcommand{\geq}{\geqslant}

\newcommand{\one}{\mathds{1}}

\newcommand{\shlomo}[1]{\comment{\textcolor{red}{Shlomo: #1}}}
\newcommand{\be}[1]{\comment{\textcolor{blue}{b: #1}}}
\newcommand{\comment}[1]{\textcolor{red}{\textbf{#1}}}
\newcommand{\ignore}[1]{}


\newtheorem{thm}{Theorem}
\newtheorem{pro}[thm]{Proposition}
\newtheorem{lem}[thm]{Lemma}
\newtheorem{cor}[thm]{Corollary}
\newtheorem{defi}[thm]{Definition}

\newcommand{\onevec}{{\mathbf{1}}}
\newcommand{\zerovec}{{\mathbf{0}}}

\usepackage[noend]{libHO/distribalgo}
\usepackage{listings}
\usepackage{graphicx}
\usepackage[utf8]{inputenc}
\usepackage[pdftex,pdfpagelabels,bookmarks,hyperindex,hyperfigures]{hyperref}

\usepackage{biblatex}
\addbibresource{dynamic.bib}
\addbibresource{bibase.bib}

\newcommand{\cent}{\gamma}

\newcommand{\dG}{\mathds{G}}
\newcommand{\IN}{\mathds{N}}
\newcommand{\IS}{\mathds{S}}

\newcommand{\In}{\mathrm{In}}
\newcommand{\ts}{s}
\newcommand{\tf}{\phi}
\newcommand{\try}{\tau}
\newcommand{\SM}{{\em SynchMod}$_{\,k}\ $}

\title{Synchronization Modulo $k$ in Dynamic Networks}
\author{Bernadette Charron-Bost\textsuperscript{1} \and Louis Penet de Monterno\textsuperscript{1}}
\date{\textsuperscript{1} \'Ecole polytechnique, 91128 Palaiseau, France\\~\\ \today}

\begin{document}

\maketitle
\tableofcontents

\begin{abstract}
	We define the mod k—synchronization problem as a weakening of the Firing Squad problem,
	where all nodes fire not at the same round, but at rounds that are all equal modulo k.
	We propose an algorithm that achieves mod k—synchronization  in any dynamic network
	with a fixed spanning star. As opposed to the perfect synchronization in
	the Firing Squad problem, mod k—synchronization thus does not require
	any strong connectivity property in the network. 
	Then, we develop our approach for the consensus problem,
	and present different consensus algorithms that all tolerate asynchronous starts.
\end{abstract}

% section_intro (do not modify this comment)
\section{Introduction}

% section_model (do not modify this comment)
\section{Preliminaries}\label{sec:model}

% section_algorithm (do not modify this comment)
\section{The \SM algorithm}

\subsection{The GeneralizedSynchMod algorithm}

\begin{algorithm}[htb]\label{algo:code}
\begin{distribalgo}[1]
\BLANK \INDENT{\textbf{Initialization:}}
	\STATE $c_u \in \mathds{N}$, initially 0
	\STATE $concordant_u \leftarrow false$
	\STATE $nopassive_u \leftarrow false$
	\STATE $ready_u \leftarrow false$
	\STATE $force_u \in \{0, 1, 2\}$, initially 0
	\STATE $level_u \in \{0, 1, 2\}$, initially 0

\ENDINDENT \BLANK

\INDENT{\textbf{In each round $t$:}}
	\STATE send $\langle \overline{c}_u, nopassive_u, concordant_u, force_u, ready_u \rangle$ to all 
	\STATE receive incoming messages
	\STATE $force_u \leftarrow$ maximum forcing level received \label{line:force}
	\STATE $c_u \leftarrow$ minimum received value among maximum force messages \label{line:min-z-end}
	\STATE $nopassive_u \leftarrow$ no null received and all received message tagged non-passive
	\STATE $ready_u \leftarrow$ all received messages were tagged as ready
	\STATE $concordant_u \leftarrow$ all received values are congruent modulo $k$ \label{line:conc-gossip}
	\IF{$c_u \equiv 0 [k] \wedge (nopassive_u \vee force_u = 2)$}
		\IF{$concordant_u \wedge level_u = 0$}
			\STATE $level_u \leftarrow 1$
			\STATE $force_u \leftarrow 1$ \label{line:force2}
			\STATE $c_u \leftarrow 0$
		\ENDIF
		\IF{$force_u = 2 \vee (concordant_u \wedge ready_u)$}
			\STATE $force_u \leftarrow 2$
			\STATE $level_u \leftarrow 2$
			\STATE $c_u \leftarrow 0$
		\ENDIF
		\STATE $concordant_u \leftarrow true$ \label{line:conc-true}
		\STATE $nopassive_u \leftarrow true$ \label{line:nopassive-true}
		\STATE $ready_u \leftarrow level_u > 0$ \label{line:init-ready}
	\ENDIF
\ENDINDENT 

\caption{The generalized \SM algorithm} \label{algo:R}
\end{distribalgo}

\end{algorithm}

We assume that the dynamic graph verifies the predicate $\mathcal{P}_{rooted} \equiv \exists \cent \in \Pi, \exists D > 0, \mathds{G}(t:t+D)$ contains a spanning tree rooted by $\cent$.
We also assume that $2D+1 < k$.

\begin{lem}
	If $nopassive_u(t)$ is true, then for any $i \geq k/2$, there exists a path starting from $\cent$ in round $t-i$ to $u$ in round $t$, and every node on this path is active.
\end{lem}
\begin{proof}
	The existence of the path results from $\mathcal{P}_{rooted}$. 
\end{proof}

\begin{lem} \label{lem:conc-safety}
	If $concordant_u(t+k+1) \wedge \overline{c}_u(t+k+1) = 1$, then, for all $i \in \{1, \dots, k/2+1\}$, the counter of $\cent$ verifies $\overline{c}_\cent(t+i) = i$.
\end{lem}
\begin{proof}
	We have $t+k+1 - (t+i) \geq k+1 - k/2 - 1 \geq k/2 \geq D$.
	Then, using $\mathcal{P}_{rooted}$, for any $i \in \{1, \dots, k/2+1\}$ in the above range, there exists a path from $\cent$ in round $t+i$ to $u$ in round $t+k+1$. 
	We can now show by induction over $k+1-i$ that for any $i \leq j \leq k+1$, the $j^{th}$-node $v(j)$ on this path verifies $c_v(j)(t+j) = j$ and $concordant_v(j)(t+j)$.

	The base case is the hypothesis of the lemma.

	For the induction case, we assume that $i \leq j < k+1$, and $\overline{c}_c(j+1)(t+j+1) = j+1$ and $concordant_c(j+1)(t+j+1)$.
	By the definition of a path, we know that $c(j)$ sent a message $\langle v, concordant_c(j)(t+j), *, * \rangle$ to $c(j+1)$.
	If $i = j = 1$, the induction case is deductible from line \ref{line:conc-true}.
	Otherwise, from line \ref{line:conc-gossip}, we get that $concordant_c(j)(t+j)$. Moreover, the received messages are concordant.
	Then, from lines \label{line:min-z} to \label{line:min-z-end}, we get $\overline{c}_c(j+1)(t+j+1) = v+1$, then $\overline{c}_c(j)(t+j) = v = j$.
\end{proof}

\begin{lem} \label{lem:no-close-level2}
	If $u$ reaches level 2 in round $t+k+1$, no node can reach level 2 in any round between $t+k+2$ and $t+k+k/2+1$.
\end{lem}
\begin{proof}
	We assume that $v$ reaches level 2 in round $t+k+i+1$ where $i \in \{1, \dots, k/2\}$.
	Applying the previous lemma to $u$ and $v$, we get respectively $c_\cent(t+i+1) = i+1$ and $c_\cent(t+i+1) = 1$.
	That contradicts the assumption $i > 0$.
\end{proof}

\begin{lem} \label{lem:ready-safety}
	If node $u$ reaches level 3 in round $t+k$, then $\cent$ must have reached level 2 in round $t$.
\end{lem}
\begin{peoof}
	We assume that $u$ is the first level 3 node. Then, in round $t+k$, the node $u$ may only receive ready messages.
	Using lemma \ref{lem:conc-safety}, we get that $\overline{c}_\cent(t) = 1$.
	Using a reasoning similar to the proof of lemma \ref{lem:conc-safety}, we can show that the messages sent by $\cent$ in round $t+1$ must be ready.
	Otherwise, by induction over a path from $\cent$ to $u$, $u$ would have received a non-ready message.
	According to line \ref{line:init-ready}, $\cent$ must have reached level 2 in round $t$.
\end{peoof}

\begin{lem} \label{lem:safety-force}
	If $\overline{c}_u(t+i) = i$ and $force_u(t+i) > 0$, then a node must have reached level $force_u(t+i)$ in round $t+1$.
\end{lem}
\begin{proof}
	Without loss of generality, we assume $force_u(t+i) = 2$.
	We show this lemma by induction over $i$.
	For the base case, we have to show that the line \ref{line:force2} was executed by $u$ in round $t+1$.
	We just have to prove that $force_u(t+1)$ was different from 2 before the execution of this line, we reason by contradiction.
	If $force_u(t+1)$ was already equal to 2, a node $v$ must have sent a message which looks like $\langle k, *, 2, * \rangle$ (see line \ref{line:force} and \ref{line:min-z}).
	This is impossible because of line \ref{line:back-to-0}.

	For the induction case, we assume that $\overline{c}_u(t+i+1) = i+1$ and $force_u(t+i+1) = 2$.
	Then, a node $v$ must have sent a message which looks like $\langle i, *, 2, * \rangle$ (see line \ref{line:force} and \ref{line:min-z}).
	The proof can now be concluded thanks to the induction hypothesis.
\end{proof}

The system is said to be \textit{v-monovalent} in round $t$ if every node $u$ verifies $\overline{c}_u(t) = v \vee silent_u(t)$.

\begin{lem} \label{lem:monovalent}
	If $\cent$ reaches level 2 in round $t+1$, the system is monovalent from round $t+k/2$, and no node is silent.
\end{lem}
\begin{proof}
	We consider any node $u$.	
	We consider the path $(\cent = v(1), \dots, v(k/2) = u)$ between $\cent$ and $u$.
	We show by induction over $i \in \{1, \dots, k/2\}$ that $\overline{c}_v(i) = i$ and $force_c_(i) = 2$.

	The base case simply results from line \ref{line:force2}.
	For the induction case, we assume that $\overline{c}_v(i) = i$ and $force_c_(i) = 2$.
	Then $v(i)$ sends to $v(i+1)$ a message which looks like $\langle i, *, 2, * \rangle$.
	We get $force_c_(i+1) = 2$.
	In order to prove $\overline{c}_v(i+1) = i+1$, we need to show that $v(i+1)$ cannot receive a message looking like $\langle i' < i, *, 2, * \rangle$.
	That can be deduced from lemmas \ref{lem:safety-force} and \ref{lem:no-close-level2}.

	Using that, we get $\forall u \in \mathcal{A}_{t+k/2}, \overline{c}_u = k/2$. The system is now monovalent.
\end{proof}

\begin{lem}
	If $\cent$ is active and non-silent, and the system is strongly $i$-monovalent in round $t$, it must be weakly $i+h$-monovalent in round $t+h$.
\end{lem}
\begin{proof}
	Since $\cent$ is active

	We do the proof by induction over the round number. The base case holds since a strongly monovalent system is also weakly monovalent.
	For the induction case, we assume that the system is weakly $i+h$-monovalent in round $t+h$.
	We consider a node $u$ which is non-silent in round $t+h+1$.
	If $u$ is silent in round $t+h$, and was active in round $t$
	Using the line 
	If the system is strongly monovalent in round $t$, then in round $t+1$, 
\end{proof}


\begin{lem}
	If $concordant_u(r)$ is true, and $\overline{c}_u(r) = i > k/2$, then, $\forall h \in [r-i, r-D], \overline{c}_\cent(h) = h-r-i$
\end{lem}

\begin{thm} \label{lem:safety} 
	The algorithm is safe.
\end{thm}
\begin{proof}
	We assume that $u$ is the first firing node, in round $r$.
	Then, $concordant_u(r)$ is true.
\end{proof}

% section_cons (do not modify this comment)
\section{Use-cases of $\mathrm{mod}\,n$-synchronization}

% section_conclusion (do not modify this comment)
\section{Conclusion and future work}

% end_section (do not modify this comment)

\printbibliography

\end{document}
