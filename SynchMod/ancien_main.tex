\documentclass[11pt,letterpaper]{article}
\special{papersize=8.5in,11in}

\usepackage{fullpage}
\usepackage{amsmath,amssymb,amsthm,dsfont}
\usepackage{tikz}
\usetikzlibrary{shapes.symbols,arrows}

\usetikzlibrary{shapes.symbols,arrows}
\usepackage{enumerate}

\usepackage{algorithm,algorithmic}

\renewcommand{\algorithmicrequire}{\textbf{Initialization:}}
\renewcommand{\algorithmicensure}{\textbf{In each round $t$ do:}}

\newcommand{\INITIALLY}{\REQUIRE{}}
\newcommand{\ROUND}{\ENSURE{}}


\DeclareMathOperator{\NC}{NC}
\DeclareMathOperator{\NCL}{NCL}
\DeclareMathOperator{\lcm}{lcm}
\DeclareMathOperator{\Rem}{Rem}
\DeclareMathOperator{\Simp}{Simp}
\DeclareMathOperator{\Crit}{Crit}
\DeclareMathOperator{\Extr}{Extr}
\DeclareMathOperator{\Closed}{Closed}
\DeclareMathOperator{\Cyc}{Closed}
\DeclareMathOperator{\Step}{Step}
\DeclareMathOperator{\Red}{Red}
\DeclareMathOperator{\Inc}{Inc}
\DeclareMathOperator{\Start}{Start}
\DeclareMathOperator{\End}{End}
\DeclareMathOperator{\Edges}{Edges}
\DeclareMathOperator{\Sections}{Sections}
\newcommand{\CDD}{{cd}}
\newcommand{\EP}{{ep}}
\newcommand{\CF}{{cr}}
\newcommand{\nc}{{\mathrm{ nc}}}
\newcommand{\In}{{\mathrm {In}}}
\newcommand{\legendre}[2]{\genfrac{(}{)}{}{}{#1}{#2}}
\newcommand{\cT}{{ (c,T)  }}
\newcommand{\tY}{\tilde{Y}}
\newcommand{\oY}{\overline{Y}}

\newcommand{\IR}{\mathds{R}}
\newcommand{\IN}{\mathds{N}}
\newcommand{\IZ}{\mathds{Z}}
\newcommand{\IQ}{\mathds{Q}}
\newcommand{\IC}{\mathds{C}}
\newcommand{\M}{\mathcal{M}}

\newcommand{\Bcnc}{A^{\Delta}_{\mathrm {c}}}
\newcommand{\Bunu}{\tilde{B}}
\newcommand{\Bms}{\tilde{B}}
\newcommand{\Bep}{A^{\Delta}_\mathrm{e}}
\newcommand{\Benp}{A^{\Delta}_\mathrm{e}}
\newcommand{\Bneone}{A^{\Delta}_\mathrm{r}}
\newcommand{\Bnetwo}{A^{\Delta}_\mathrm{r}^\mathrm{HA}}

\newcommand{\SCC}{\mathcal{C}}
\newcommand{\CP}{\mathcal{P}_\ccirclearrowleft}

\newcommand{\IRmax}{\overline{\IR}}
\newcommand{\IRmin}{{\IR}_{\min}}
\newcommand{\wstar}{{w}_{*}}
\newcommand{\ito}{{i\!\to}}
\newcommand{\Pa}{{\mathcal{W}}}

\newcommand{\real}[1]{\mathbf{N}_{\geqslant {#1}}}
\newcommand{\realrem}[2]{\mathbf{N}_{\geqslant {#1}}^{({#2})}}

\renewcommand{\le}{\leqslant}
\renewcommand{\leq}{\leqslant}
\renewcommand{\ge}{\geqslant}
\renewcommand{\geq}{\geqslant}

\newcommand{\one}{\mathds{1}}

\newcommand{\shlomo}[1]{\comment{\textcolor{red}{Shlomo: #1}}}
\newcommand{\be}[1]{\comment{\textcolor{blue}{b: #1}}}
\newcommand{\comment}[1]{\textcolor{red}{\textbf{#1}}}
\newcommand{\ignore}[1]{}


\newtheorem{thm}{Theorem}
\newtheorem{pro}[thm]{Proposition}
\newtheorem{lem}[thm]{Lemma}
\newtheorem{cor}[thm]{Corollary}
\newtheorem{defi}[thm]{Definition}

\newcommand{\onevec}{{\mathbf{1}}}
\newcommand{\zerovec}{{\mathbf{0}}}

\usepackage[noend]{libHO/distribalgo}
\usepackage{listings}
\usepackage{graphicx}
\usepackage[utf8]{inputenc}
\usepackage[pdftex,pdfpagelabels,bookmarks,hyperindex,hyperfigures]{hyperref}

\usepackage{biblatex}
\addbibresource{dynamic.bib}
\addbibresource{bibase.bib}

\newcommand{\cent}{\gamma}

\newcommand{\dG}{\mathds{G}}
\newcommand{\IN}{\mathds{N}}
\newcommand{\IS}{\mathds{S}}

\newcommand{\In}{\mathrm{In}}
\newcommand{\ts}{s}
\newcommand{\tf}{\phi}
\newcommand{\try}{\tau}
\newcommand{\SM}{{\em SynchMod}$_{\,k}\ $}

\title{Synchronization Modulo $k$ in Dynamic Networks}
\author{Bernadette Charron-Bost\textsuperscript{1} \and Louis Penet de Monterno\textsuperscript{1}}
\date{\textsuperscript{1} \'Ecole polytechnique, 91128 Palaiseau, France\\~\\ \today}

\begin{document}

\maketitle
\tableofcontents

\begin{abstract}
	We define the mod k—synchronization problem as a weakening of the Firing Squad problem,
	where all nodes fire not at the same round, but at rounds that are all equal modulo k.
	We propose an algorithm that achieves mod k—synchronization  in any dynamic network
	with a fixed spanning star. As opposed to the perfect synchronization in
	the Firing Squad problem, mod k—synchronization thus does not require
	any strong connectivity property in the network. 
	Then, we develop our approach for the consensus problem,
	and present different consensus algorithms that all tolerate asynchronous starts.
\end{abstract}

% section_intro (do not modify this comment)
\section{Introduction}

Distributed algorithms are often designed in a synchronous computing model, in which computation
	is divided into communication {\em closed rounds}:
	any message  sent at some round  can be received only at that round.
In this model it is classically assumed that each run of an algorithm is started by all nodes simultaneously, i.e., at the same round,
	or even at round one.
For instance, most of synchronous consensus algorithms
	(e.g.,~\cite{PSL80,DS83,ST87}), as well as many distributed algorithms for dynamic networks (e.g.,~\cite{KLO10,KMO11})
	require synchronous starts.

This assumption makes the sequential composition of two distributed algorithms $A;B$
	-- in which each node starts executing $B$ when it has completed the execution of~$A$ --
	quite problematic.
Indeed, nodes start the algorithm~$B$ asynchronously when the algorithm~$A$ terminates asynchronously,
	and the properties of~$B$ are no more guaranteed in this context of asynchronous starts.

This leads to the problem of simulating synchronous starts, classically referred to as
	the {\em firing squad problem}:
Each node is  initially  {\em passive} and then become {\em active}  at an unpredictable round.
The goal is to guarantee that   the nodes, when all active, eventually synchronize
	by {\em firing} -- i.e., entering a designated state for the first time -- at the same round.

Unfortunately, the impossibility result in~\cite{CBM18} demonstrates that the firing squad problem
	is not solvable without a strong connectivity property of the network, namely, there exists  some positive
	integer~$T$ such that communication graph within  every period of $T$ consecutive
	rounds is strongly connected and the delay~$T$ is known.
In many situations,  this connectivity property is not guaranteed:
	as an example, in the dynamic graphs corresponding to any system model with benign failures,
	a node  that experiments permanent and complete send omissions
	is  constantly a sink in the  communication graph.

However, with a closer look at many distributed algorithms designed in the round-based model,  we see that
	these algorithms actually  do not require perfect synchronous starts, and still work under the weaker condition that all the nodes
	start executing the algorithms in rounds with numbers that are equal modulo~$k$, for some positive integer~$k$.
The corresponding synchronization problem, that we call ${\mathrm{mod}\,k}$  \emph{synchronization}, is formally specified as follows:
	\begin{description}
	\item[Termination:]   If all nodes are eventually active, then  every node eventually fires.
	\item[ $\mathbf{\mathrm{mod}\,k}$-simultaneity:]  If two nodes fire at round~$t$ and $t'$, then $t' \equiv t \mod k$.
	\end{description}

Indeed, let $A$ be an algorithm  organized into regular  \emph{phases} consisting of  a fixed number  $k$ of consecutive rounds:
	the sending and transition functions of every node at round~$t$ are entirely determined by the value of~$t$ modulo~$k$.
Moreover, assume that $A$ has been proved correct (with respect to some given specification) when all nodes start $A$
	synchronously (at round one) and with any dynamic graph in a family~${\cal G}$ that is stable  under
	addition of arbitrary finite prefixes.
For instance, the \emph{ThreePhaseCommit} algorithm for non-blocking atomic commitment~\cite{BHG87},
	as well as the consensus algorithms in~\cite{DLS88} or the \emph{LastVoting} algorithm ~\cite{CBS09}
	-- corresponding to the consensus core of \emph{Paxos} -- fulfill all the above requirements
	for phases of length $k=3$ and $k=4$, respectively, and the family ${\cal G}$ of dynamic graphs in which
	there exists an infinite number of ``good'' communication patterns
	(eg., a sequence of $2k-1$ consecutive communication graphs  in which a majority of nodes is heard by all
	in   each graph).
The use of a ${\mathrm{mod}\,k}$-synchronization algorithm on the top of the algorithm $A$ yields a new algorithm that executes
	exactly like $A$ does, after a finite preliminary period during which every node becomes active and fires.
The above property on the set of dynamic graphs ${\cal G}$ then guarantees this variant of $A$ to be correct with
	asynchronous starts and dynamic graphs in ${\cal G}$.

Another typical example for which perfect synchronization can be weakening into synchronization modulo $k$ is
	the development of the basic \emph{rotating coordinator} strategy in the context of asynchronous starts.
Roughly speaking, this strategy consists in the following: if nodes have unique identifiers in $\{1,\dots,n\}$,
	the coordinator at round~$t$ is the node whose identifier is  $t$ modulo~$n$.
For that, 	 each node~$u$ maintains a local counter $c_u$
	whose current value is the number of rounds where it has been active.
At each round, the  coordinator of $u$ is the  node with the identifier that is equal to the current value of $c_u$ modulo~$n$.
Since there may be only one coordinator per round, such a selection rule requires synchronous starts.
Clearly, with the use of a ${\mathrm{mod}\,n}$-synchronization algorithm in a preliminary phase and a
	counter for each node that now counts the number of rounds elapsed since the node fired, the above scheme correctly\footnote{%
	With respect to a specification that lets a passive node to be  a coordinator.}
	implements the rotating coordinator strategy in the context of asynchronous starts.

% section_model (do not modify this comment)
\section{Preliminaries}\label{sec:model}
 
\subsection{The computational model}
	
We consider a networked system with a {\em fixed} set $V$ of $n$ nodes.
We assume a round-based computational model  in the spirit of the Heard-Of model~\cite{CBS09}, 
	in which point-to-point communications are organized into \emph{synchronized rounds}: 
	each node can send messages  to all nodes and can receive messages sent  by some of the nodes.
Rounds are communication closed in the sense that no node receives messages in round~$t$ that are sent 
	in a round different from~$t$. 
The collection of \emph{possible} communications (which nodes can communicate to which nodes) at each round $t$
	is modelled by a directed graph (digraph, for short) with a set of nodes equal to~$V$.
The digraph at round~$t$ is  denoted $\dG(t)=(V,E_t)$, and is called the \emph{communication graph at round}~$t$. 
The set of $u$'s incoming neighbors in the digraph $\dG(t)$ is denoted by $\In_u(t)$.

We assume a self-loop at each node in all these digraphs  since every node can communicate with 
	itself instantaneously.	
The sequence of such digraphs~$\dG=\left (\dG(t) \right )_{t\in\IN}$ is called a {\em dynamic graph}~\cite{CFQS11:TVG}. 

In round $t$ ($t = 1, 2 , \ldots $), each node~$ u $ successively
	(a) broadcasts  messages determined by its state at the beginning of round~$ t $
	(b) receives \emph{some} of the messages sent to it,
	and finally (c) undergoes an internal transition to a new state.
A  \emph{local algorithm} for a node corresponds to a pair of
	a \emph{sending function} that determines the messages to be sent in step~(a)
	and a \emph{transition function} for state updates in step (c).
An \emph{algorithm} for the set of nodes~$V$ is a collection of local algorithms, one per node.

We also introduce the notion of  \emph{start schedules}
	that are collections~$\IS= \left (s_u \right )_{u \in V}$,
	where each~$s_u$ is  a positive integer or is equal to $\infty$.

	
The execution of the algorithm $ A $  with the dynamic graph $\dG$ and the start schedule $\IS$ then proceeds
	as follows:
Each node~$u$ is initially  \emph{passive}. 
If $s_u = \infty$, then  the node~$u$ remains passive forever.
Otherwise, $s_u $ is a positive integer, and $u$ becomes {\em active} 
	at the beginning of round~$s_u$, sets up its local variables.
In  round~$t$ $(t = 1,2\dots)$, a passive  node
	sends only heartbeats, corresponding to  \emph{null} messages,  and  cannot change its state. 	
An active node 	applies its sending function in~$A$ to its current state to generate the message to be sent to all nodes,
	then it receives the messages sent by its incoming neighbors in the directed graph~$\dG(t)$, and finally 
	applies its transition function ${\cal T}_u$ in~$A$ to its current state and the list of messages it has just received,
	(including the null messages from passive nodes) to go to  a next state. 
Since each local algorithm is deterministic, an execution of the algorithm~$A$  is entirely determined 
	by the initial state of the network,  the dynamic graph $\dG$,
	and  the  start schedule~$\IS$.
	
The states ``passive'' and ``active'' do not refer to any physical notion, and are relative to the algorithm under consideration:
	as an example, if two algorithms $A$ and $B$ are sequentially executed according to the order ``$A$ followed by $B$'',
	then at some round, a node may be active w.r.t. $A$ while it is passive w.r.t. $B$.
In such a situation, the node  is integrally part of the system and can send messages, but  these messages are empty 
	with respect to the semantics of~$B$.
	
\subsection{Network model and start model}

A \emph{network model} is any non-empty set of dynamic graphs with a permanent self-loop at each node.
We will focus on the specific network model of \emph{centered} dynamic graphs~$\dG$ defined as follows: 
	there exists some node~$\cent$ in $V$ such that every digraph $\dG(t)$ has a spanning star centered 
	at~$\cent$, i.e., 
	$$\exists \cent \in V, \, \forall u \in V, \, \forall t \in \mathds{N}, \ \cent \in \In_u(t) .$$
Note that there may be several fixed centers for~$\dG$.

As demonstrated in~\cite{CBS09}, centered dynamic graphs in the Heard-Of model captures the
	classical model of synchronous systems with a (fixed) complete communication graph and 
	 at most $n-1$ faulty senders, including the case of crashes.

Similarly, we define a \emph{start model} as a non-empty set of start schedules.
A start schedule $\IS = (s_u)_{u\in V}$ is \emph{complete} if every $s_u$ is finite, i.e.,
	no node is passive forever, yielding the model of complete start schedules.
Synchronous starts correspond to complete start schedules $\IS = (s_u)_{u\in V}$ with
	equal start rounds.	
The property of synchronous starts can be relaxed into \emph{$\mathrm{mod}\,k$-synchronous starts},
	where $k$ is any positive integer: for every pair of nodes~$u$ and $v$, it holds that $s_u \equiv s_v \!\mod k$.

% section_algorithm (do not modify this comment)
\section{The \SM algorithm}

In the \SM algorithm, each node  maintains a local clock modulo $k$ with values in $\{ \overline{1}, \dots,  \overline{k} \}$.
It fires in the first round at which all the local clocks it has just heard of  are all equal to~$\overline{k} $ (line~\ref{line:fire}).
The first time a node receives discrepant clocks from its neighbors, it tries to force firing  by setting its clock to  $\overline{k} $
	(lines~\ref{line:try}-\ref{line:try+1});
	thereafter, that  just leads it to roll back its clock to 1 (line~\ref{line:tried}).
Otherwise, it receives agreed values with its own clock and then increments it by one modulo $k$ (line~\ref{line:agreed}).
Let us again stress on the fact that at each round~$t$, every active node receives the value of its local clock.
The pseudo-code of the local code of the agent~$u$ is given in Algorithm~1.

As we will see below, the difficult point in the correctness proof of the \SM algorithm is liveness.
However, right now, let us point out some properties of the algorithm that enable liveness.
First, if all the nodes  agree on the same value for their local clocks 
	-- in which case the system will be said to be \emph{monovalent} -- and if they are all active, 
	then the system remains monovalent forever.
Moreover, the common value of the local clocks is incremented by one at every later round and thus eventually 
	reaches the value~$\overline{k} $ (cf. Lemma \ref{lem:mono_liv}).
The key point of the algorithm and of its ``forced firing procedure'' lies in the fact that if all the communication graphs  
	contain a star centered at~$\cent$, then when  $\cent$ is active, its local clock necessarily becomes equal 
	to~$\overline{k}$, and every active node will eventually fire. 

% \textbf{Note:} The definition of an execution requires that a passive node always sends null. The algorithm itself requires that, at the round of its activation, a node always sends $\bot$.
% This is designed such that node just activated does not disturb a monovalent configuration with an initial value in $\mathds{Z}/k\mathds{Z}$.

\begin{algorithm}[htb]\label{algo:code}
\begin{distribalgo}[1]
\BLANK \INDENT{\textbf{Initialization:}}
	\STATE $\overline{c}_u \in \mathds{Z}/k\mathds{Z} \cup \{\bot\}$, initially $\bot$
	\STATE $tried_u \leftarrow false$
	\STATE $fired_u \leftarrow false$

\ENDINDENT \BLANK

\INDENT{\textbf{In each round $t$:}}
	\STATE send $\langle \overline{c}_u \rangle$ to all 
	\STATE receive incoming messages
	\IF{all the received messages are equal to $\overline{k}$ and $\neg fired_u$ }
		\STATE $fired_u \leftarrow true$ \label{line:fire}
	\ENDIF
	\IF{the received messages other than $null $ and $\bot$ are all equal to $i \in \{ \overline{1}, \dots, \overline{k} \}$ }
		\STATE $\overline{c}_u \leftarrow \overline{i+1} $ \label{line:agreed}
	\ELSE \IF{$\neg tried_u $ and no received message is $\overline{k}$ }
		\STATE $\overline{c}_u \leftarrow \overline{k} $  \label{line:try}
		\STATE $tried_u \leftarrow true$   \label{line:try+1}
	\ELSE
		\STATE $\overline{c}_u \leftarrow \overline{1}$ \label{line:tried}
	\ENDIF
	\ENDIF
\ENDINDENT 

\caption{The \SM algorithm} \label{algo:R}
\end{distribalgo}

\end{algorithm}


\subsection{Notation and preliminary lemmas}

In the rest of this section, we fix an execution $\sigma$ of the \SM algorithm associated to a complete activation 
	schedule~${\cal A}$ and a centered dynamic graph~$\dG \in {\cal G}^c$. % a definir 
Let $s^{\max} = \max_{u \in \Pi} s(u) < \infty$ and let~$\cent$ denote one center of~$\dG$.	


For the correctness proof of \SM, we now introduce some additional definitions.
Let $S$ be any subset of $ \mathds{Z}/k\mathds{Z}$.
Round~$t$ in~$\sigma$  is said to be \emph{$S$-valent}  if $S$ is the set of the clock values of active nodes 
	at the end of round~$t$, i.e.,
	$$ S = \{ \overline{i}  \in \mathds{Z}/k\mathds{Z} : \exists u \in \mathcal{A}_t, \ \overline{c}_u (t) = \overline{ i }\,  \}  . $$
The system is said to be $\overline{i}$-\emph{monovalent}  if the system is $\{ \overline{i}\}$-valent.

Let us define $\tf (u)$ to be the round number at which the node~$u$ fires, if any, and let $\tf(u) = \infty$ otherwise.
Similarly, let us define $\try (u)$ to be  the round number at which the node~$u$ tries to force firing
	(line~\ref{line:try}) if any, and let $\try (u)= 0$ otherwise.
It follows that $\try^{\max} =  \max_{u \in \Pi}  \try(u) < \infty$.

In the \SM algorithm, the state of each node $u$ is composed of three variables named $\overline{c}_u$, $fired_u$ and $tried_u$.
For any round $r$, we denote $\overline{c}_u(r)$, $fired_u(r)$ and $tried_u(r)$ respectively the values of these variables in the execution $\sigma$.

\begin{lem}\label{lem:k_mono}
If $\overline{c}_\cent(t) = \overline{k} $, then the round~$t +1$  is $\overline{1}$-monovalent.
\end{lem}
\begin{proof}
If $\overline{c}_\cent(t) = \overline{k} $, then the node $\cent$ sends $\overline{k}$ to all nodes in round $t+1$.
Hence, any active node $u$ at round~$t+1$ receives $\overline{k}$ in this round,
	and so updates its clock $\overline{c}_u$ according either to line~\ref{line:agreed} of to line \ref{line:tried}.
In both cases, it holds that $\overline{c}_u (t+1) =\overline{1}$.
\end{proof}

\begin{lem}\label{lem:mono_mono}
	If the center $\cent$ is active in round $t$  and round~$t$ is $\overline{i}$-monovalent, 
	then any subsequent round~$ t + h$ is $\overline{i+ h}$-monovalent.
\end{lem}

\begin{proof}
The proof is by induction on $h \in \mathds{N}$.
\begin{enumerate}
	\item The base case $s=0$ corresponds to the assumption in the lemma.
	\item Induction step:  assume that the round $t+h$ is $\overline{i+h}$-monovalent,
		and let~$u$ be  any active node in round~$t+h+1$.
		The center $\cent$ is active in round $t +h$,
		and thus sends the value $\overline{i+h}$ to~$u$ in round~$ t+h+1$.
		Therefore, the node~$u$ can receive only this value (in addition of null and $\bot$),
		and thus updates $\overline{c}_u$ according to line~\ref{line:agreed}. 
		It follows that $ \overline{c}_u(t+h+1) = \overline{i+h+1}$ as required.
\end{enumerate}
\end{proof}

\begin{lem}\label{lem:k_liv}
If the center $\cent$ is active at round $t$  and $\overline{c}_\cent(t) = \overline{k} $, 
	then every node fires no later than round $\max(t, s^{\max}) + k $.
\end{lem}
\begin{proof}
Lemmas \ref{lem:k_mono} and \ref{lem:mono_mono} show that the system is $\overline{1}$-monovalent in round $t+1$
	and $\overline{i}$-monovalent in every  following round~$t+i$.
It follows that there is a round $r$, with $r \in [ \max ( t, s^{\max} )  , \max (t , s^{\max} ) + k  - 1] $,
	that  is $\overline{k}$-monovalent and at which every node is active.
In round $r$, every node, including $\cent$, sends $ \overline{k} $.
According to line~\ref{line:fire}, every node fires no later than round~$r+1$.
\end{proof}
 
\begin{lem}\label{lem:mono_liv}
If  round $t$ is a monovalent round in which the center $\cent$ is active, then 
	every node fires before round $max( t, s^{\max} )+k + 1$.
\end{lem}
\begin{proof}
Lemma \ref{lem:mono_mono} guarantees that for every non negative integer~$i$,
	the round~$t +i$ is monovalent.   
Hence, there exists some round~$r$, with $r \in [ max( t, s^{\max} ) , max( t, s^{\max} )+ k - 1 ] $ 
	that is  $\overline{k}$-monovalent.
In this round, every node, including $\cent$, sends $\overline{k}$.
According to line~\ref{line:fire}, every node fires no later than round~$r+1$.
\end{proof}


\begin{lem}\label{lem:mono_bi}
	Any round $t > \max (s(\cent), \try^{\max}) + 1$  is either monovalent or
	$S$-valent with  $S = \{1, \overline{i} \, \}$.
\end{lem}
\begin{proof}
Let $t \geq \max (t_s(\cent), \try^{\max}) + 1$, and let $u$ be an active node at round~$t$.
Since $t \geq s(\cent)+1$, we have $ \overline{c}_\cent(t - 1) = \overline{i -1 } \in \mathds{Z}/k\mathds{Z} $,
	and the node~$u$ receives the value $ \overline{i - 1}$ at round~$t$.
There are two cases to consider.
\begin{enumerate}
\item The node $u$ receives no value else than $\overline{i -1}$ and $\bot$ at round~$t$.
 Then $u$ executes line~\ref{line:agreed}, and $\overline{c}_u(t) = \overline{ i }$, i.e.,
 	round~$t$ is $ \overline{ i }$-monovalent.
\item Otherwise, $u$ executes line~\ref{line:tried} since it has already tried to force firing ($t > \try(u)$).
Therefore, $\overline{c}_u(t) = \overline{1}$, which shows that round~$t$ is $\{1, \overline{ i } \, \}$-valent.
\end{enumerate}
\end{proof}

\subsection{Correctness proof}

We are now in position to prove the correctness of the \SM algorithm for any integer~$k$ greater than 2
	under the conditions of a complete activation schedule and a  centered dynamic graph.
The safety property is a direct consequence of the above lemmas.
For the liveness property, Lemmas~\ref{lem:k_liv} and~\ref{lem:mono_liv} lead us to define the notion 
	of a \emph{good round}  as either a monovalent round or a round in which  the  counter of any 
	center reaches  the value $\overline{k}$:
	liveness is then enforced by the existence of a good round.

\begin{thm}\label{thm:k>2}
Under the conditions of complete activation schedules and centered dynamic graphs,
	the \SM algorithm solves the $\mathrm{mod}\,k$-synchronization problem for any integer~$k$ greater than 2.
\end{thm}

\begin{proof}
Let $\alpha$ be an execution of the \SM algorithm with a dynamic graph centered at~$\cent$ 
	and a complete activation schedule.
	
For the safety property, let us assume that some nodes fire in $\alpha$, and let~$u$ be the first node that fires in round $\tf(u)$.
Since $\cent $ is an incoming neighbor of~$u$, the firing rule (line~\ref{line:fire}) implies that $\cent$ is active
	and has sent the value  $\overline{k}$ in round $\tf(u)$.
By Lemma \ref{lem:k_mono},   round $\tf(u)$ is $\overline{1}$-monovalent, and
	Lemma \ref{lem:mono_mono} shows that for every round $\tf(u)+i$ is $\overline{i}$-monovalent.
If a node $v$ fires in a later round $\tf(v) > \tf(u)$, the round $\tf(v)-1$  is $\overline{k}$-monovalent (line~\ref{line:fire});
	and thus $\tf(u)$ and $\tf(v)$ are congruent modulo $k$.
		
For the liveness property, it suffices to prove that  $\alpha$ contains a good round.
Let us first observe that if $\cent$ tries to force firing (line~\ref{line:try+1}) at round~$t$,
	 then $ \overline{c}_{\cent}(t) = \overline{k}$ and round $t$ is a good round.
Thus let us assume  $\try(\cent) = 0$, and  let $t \geq \max (s(\cent) , \try^{\max}) +1$.
Lemma~\ref{lem:mono_bi}  shows that either (1) round~$t$ is monovalent, 
	or (2) $ \overline{c}_{\cent}(t) = \overline{i} \neq  \overline{1} $ and round~$t$ is $ \{\overline{1}, \overline{i} \, \}$-valent,
	or (3) $ \overline{c}_{\cent}(t) = \overline{1}$ and round~$t$ is $ \{\overline{1}, \overline{i} \, \}$-valent.
\begin{enumerate}
\item In case (1), round~$t$ is a good round.

\item In case (2), the center~$\cent$ only receives the value $ \overline{c}_{\cent}(t) = \overline{i}$ at round~$t+1$ 
since it never tries to force firing.
Hence, we have $ \overline{c}_{\cent}(t + 1)  = \overline{i +1} $ , and round~$t+1 $ is $ \{\overline{1}, \overline{i+1} \, \}$-valent.
Repeating this argument yields $ \overline{c}_{\cent}(t + k-i )  = \overline{ k } $.
Hence, round~$t +k-i$ is a good round.

\item For case (3),  we consider the following two subcases:
\begin{enumerate}
\item Node~$\cent$ does not receive the value~$ \overline{i} $ at round~$t+1$, and so $ \overline{c}_{\cent}(t + 1)  = \overline{ 2 } $.
Since  $t> \try^{\max} $, we have $  \overline{c}_{u}(t + 1)  = \overline{ 1 } $ or $  \overline{c}_{u}(t + 1)  = \overline{ 2 } $ 
	for every node~$u$.
Moreover, round~$t+1$ is $ \{\overline{1} , \overline{2} \, \}$-valent and meets the above case (2).
It follows that round~$t +k-1$ is a good round.
\item Node~$\cent$ receives the value~$ \overline{i} $ at round~$t+1$.
Since $\try(\cent)= 0$, this case may occur only if $ \overline{i } = \overline{ k }$.
In this case, $  \overline{c}_{u}(t + 1)  = \overline{ 1 } $, and round~$t+1$ is either 
	$ \{\overline{1}\}$-monovalent or  $ \{\overline{1} , \overline{2} \, \}$-valent.
In the latter situation,  round~$t+1$ meets case (3) with $\overline{i}  =\overline{2} $, and hence case (3.a) when  $k>2$.
Therefore, either round~$t+1$ or round~$t+k$ is a good round.
\end{enumerate}
\end{enumerate}
It follows that if $k>2$, then the execution $\alpha$ contains a good round, and
	Lemmas~\ref{lem:k_liv} and~\ref{lem:mono_liv} imply the liveness property of the \SM algorithm.
\end{proof}

\subsection{The $\mathrm{mod}\,2$-synchronization}\label{sec:k=2}

Unfortunately, the \SM algorithm does not work when $k=2$, as demonstrated by its execution 
	with three nodes and the two-periodic dynamic graph 
	$G,H,G,H \cdots, G, H, \cdots$, where $G$ is the communication graph in every even round, and $H$ is the communication graphs in every odd round.
	$G$ and $H$ are given in Figure below.
	The starts schedule is defined by $\mathcal{A}_0 = \emptyset$, $\mathcal{A}_1 = \{u_1\}$ and $\mathcal{A}_i = \Pi$ for any $i>1$.
	The system cycles between two configuration and no node ever fire. That violates the termination property. 
	
\begin{figure}[h]
	\centering
	\includegraphics[width=\textwidth]{contr_exemple_init}
	\caption{First three rounds of non-terminating execution}
\end{figure}

\begin{figure}[h]
	\centering
	\includegraphics[width=0.7\textwidth]{contr_exemple_cycle}
	\caption{Loop in non-terminating execution}
\end{figure}

However,  the $\mathrm{mod}\,k$-synchronization is trivially reducible to the $\mathrm{mod}\,k'$-synchronization 
	if $k$ divides~$k'$.
Hence, the $\mathrm{mod}\,2$-synchronization is solvable with the {\em SynchMod}$_{\,4}$ algorithm 
	in the class of dynamic  graphs with a fixed center.

Then, a  natural question is  whether there exists a better and more direct algorithm for the $\mathrm{mod}\,2$-synchronization
problem in the class of  centered dynamic graphs, using a  different algorithmic approach. 

\section{A possible optimisation}

There exists some scenarios where the majority of nodes are active, and some of them have already fired. Because of some remaining passive node, some active nodes refrain from firing.
In the worst case, a single passive node can prevent several active nodes from firing, during an arbitrarily long time.
In this section, we propose an optimisation which improves time-complexity in the average case.
When a node fires, it learns that the system is monovalent. During the subsequent rounds, it tries to share this knowledge using gossip.
A node knowing that the system is monovalent may fire as soon as its counters is equal to $\overline{1}$. No received null message can delay its firing any longer.

\begin{algorithm}[htb]
\begin{distribalgo}[1]
\BLANK \INDENT{\textbf{Initialization:}}
	\STATE $\overline{c}_u \in \mathds{Z}/k\mathds{Z} \cup \{\bot\}$, initially $\bot$
	\STATE $tried_u \leftarrow false$
	\STATE $fired_u \leftarrow false$
	\STATE $monovalent_u \leftarrow false$

\ENDINDENT \BLANK

\INDENT{\textbf{In each round $t$:}}
	\STATE send $\langle \overline{c}_u, monovalent_u \rangle$ to all 
	\STATE receive incoming messages
	\STATE $monovalent_u \leftarrow monovalent_u \vee$ all the received messages are equal to $\overline{k}~\vee$ one true flag received \label{line:detect-monovalent}
	\IF{$monovalent_u \wedge \overline{c}_u = k$}
		\STATE $fired_u \leftarrow true$
	\ENDIF
	\IF{the received messages other than $null $ and $\bot$ are all equal to $i \in \{ \overline{1}, \dots, \overline{k} \}$ }
		\STATE $\overline{c}_u \leftarrow \overline{i+1} $
	\ELSE \IF{$\neg tried_u $ and no received message is $\overline{k}$ }
		\STATE $\overline{c}_u \leftarrow \overline{k} $
		\STATE $tried_u \leftarrow true$
	\ELSE
		\STATE $\overline{c}_u \leftarrow \overline{1}$
	\ENDIF
	\ENDIF
\ENDINDENT 

\caption{The optimized \SM algorithm} 
\end{distribalgo}

\end{algorithm}

\begin{thm}
Under the conditions of complete activation schedules and centered dynamic graphs,
	the optimized \SM algorithm solves the $\mathrm{mod}\,k$-synchronization problem for any integer~$k$ greater than 2.
	Moreover, in any execution of the optimized algorithm, no node can fire later than in the corresponding execution of the non-optimized algorithm.
\end{thm}

The proof of the first part of this theorem is exactly the same as the proof of the non-optimized algorithm.
Moreover, the proof of the second part results from a simple observation of the line \ref{line:detect-monovalent}.

\subsection{The GeneralizedSynchMod algorithm}

\begin{algorithm}[htb]\label{algo:code}
\begin{distribalgo}[1]
\BLANK \INDENT{\textbf{Initialization:}}
	\STATE $\overline{c}_u \in \mathds{Z}/k\mathds{Z} \cup \{\bot\}$, initially $\bot$
	\STATE $tried_u \leftarrow false$
	\STATE $concordant_u \leftarrow false$
	\STATE $forced_u \leftarrow false$
	\STATE $ready_u \leftarrow false$
	\STATE $stuck_u \leftarrow 0$

\ENDINDENT \BLANK

\INDENT{\textbf{In each round $t$:}}
	\STATE send $\langle \overline{c}_u, concordant_u, forced_u, ready_u \rangle$ to all 
	\STATE receive incoming messages
	\IF{at least one forced message received, containing value $v$}
		\STATE $forced_u \leftarrow true$
		\STATE $\overline{c}_u \leftarrow v+1$
	\ELSE
		\STATE $\overline{c}_u \leftarrow min \{v+1, \langle v, *, * \rangle received \}$
	\ENDIF
	\IF{$\overline{c}_u < k/2$}
		\STATE $forced_u \leftarrow false$
	\ENDIF
	\STATE $ready_u \leftarrow$ all received messages were tagged as ready
	\STATE $concordant_u \leftarrow$ all received values are equal to $\langle v, true, *, * \rangle$ (hence no null, no $\bot$)
	\IF{$concordant_u \wedge \overline{c}_u = 1 \wedge ready_u$}
		\STATE $fired_u \leftarrow true$
	\ENDIF
	\IF{$concordant_u \wedge \overline{c}_u = k/2 \wedge \neg tried_u$}
		\STATE $tried_u \leftarrow true$
		\STATE $forced_u \leftarrow true$
	\ENDIF
	\IF{$\overline{c}_u = 1$}
		\STATE $concordant_u \leftarrow true$
	\ENDIF
	\IF{$\overline{c}_u = k/2$}
		\STATE $ready_u \leftarrow tried_u$
	\ENDIF
	\IF{$\overline{c}_u < k/2$}
		\STATE $stuck_u \leftarrow 0$
	\ELSE
		\STATE $stuck_u \leftarrow stuck_u + 1$
	\ENDIF
	\IF{$stuck_u > k$}
		\STATE $\overline{c}_u \leftarrow \overline{k} $  \label{line:try}
	\ENDIF
\ENDINDENT 

\caption{The \SM algorithm} \label{algo:R}
\end{distribalgo}

\end{algorithm}

\begin{lemma}
	If $concordant_u(r)$ is true, and $\overline{c}_u(r) = i > k/2$, then, $\forall h \in [r-i, r-D], \overline{c}_\cent(h) = h-r-i$
\end{lemma}

\begin{thm}
	The algorithm is safe.
\end{thm}
\begin{proof}
	We assume that $u$ is the first firing node, in round $r$.
	Then, $concordant_u(r)$ is true.
\end{proof}

% section_cons (do not modify this comment)
\section{Use-cases of $\mathrm{mod}\,n$-synchronization}

\subsection{Algorithms without safety assumption}

Let us consider a given algorithm $A$, which can solve a problem $\mathcal{A}$ when the starts are synchronous.
The correctness of $A$ may depend on some assumptions.
Usually, the assumptions required for safety are expressed as invariants on the dynamic graph: $\forall r \in \mathds{N}, P_{saf}(\mathds{G}_r)$.
Whereas the assumptions required for liveness are expressed as infinitely often verified predicates on the dynamic graph: $\forall r \in \mathds{N}, \exists r' > r, P_{liv}(\mathds{G}_{r'})$.
The case where no invariant is required for safety is easier to deal with. At first we focus on this case.

We assume that $A$ is structured as $k$ rotating phases, which means that $A$ needs $\mathrm{mod}\,k$-synchronization.
We construct the algorithm $SynchMod_k \lozenge A$ in the following way:
we consider a node $u$. 
\begin{enumerate}
	\item At first, $u$ is passive, and sends null messages, according to the model.
	\item From round $s_u$, the node $u$ is active. It starts the \SM algorithm. It sends and receives messages according to the \SM algorithm and ignore the messages related to $A$.
	\item In round $\tf_u$, the node $u$ fires. From this round, $u$ executes \SM in parallel with $A$. Both $A$ and \SM send and receive messages according to their pseudo-code.
		In some cases, $u$ may receive from a neighbor $v$ a message related to \SM, and no message related to $A$. Then $u$ can guess that $v$ has not fired yet.
		The $A$ instance of $u$ interpret that as a null message received from $v$.
\end{enumerate}

We now wonder whether $SynchMod_k \lozenge A$ solves $\mathcal{A}$ with asynchronous starts, and how $A$ should deal with null message.
In the case without required invariant, $A$ could assimilate null messages as lost messages,
and we could reasonably expect that $SynchMod_k \lozenge A$ solves $\mathcal{A}$ with asynchronous starts.
Indeed, the drop of the null messages in $A$ is equivalent with the drop of some edges in the dynamic graph.
The safety proof without invariant is not affected by the modification of the dynamic graph, and should be easily extended from synchronous starts system to asynchronous starts system.
Moreover, if we assume that no node remains passive forever, no edge is dropped in the dynamic graph beyond a certain round.
Then, the liveness proof with synchronous starts should be still valid if applied from that round.

Some notable examples of algorithms which can be adapted with this approach include the LastVoting \cite{CBS09} algorithm (an round-based adaptation of Paxos \cite{paxos}) which solves consensus,
and the ThreePhaseCommit \cite{BT93} algorithm, which solves the database commit problem.

\subsection{Algorithms with safety assumption}

When the safety proof of an algorithm relies on some invariant, the drop of edges in the dynamic graph might break the safety proof.
A possible fix could consist in changing the way the algorithm $A$ deals with null message. This must be done in a case-by-case basis.
We exemplify that with an adaptation of the UniformVoting \cite{CBS09} \cite{Ben83} algorithm.


\begin{algorithm}[htb]
\begin{distribalgo}[1]
\begin{tabular}{ll}
\begin{minipage}{33em}


\INDENT{\textbf{Initialisation:}}
	\STATE $x_u := v_u$ ~~~~~~~~\{\emph{$v_u$ is the initial value of $p$}\}
	\STATE $vote_u \in V\cup\{ ? \}$, initially $?$
	\STATE $phase_u = true$ ~~~~~~\{\textit{variable which organises the two-phase rotation}\}

\ENDINDENT
\BLANK

\INDENT{\textbf{Round $r$:}}
	\IF{$phase_u$}
		\STATE send $\langle x_u , vote_u \rangle$ to all
	\ELSE
		\STATE send $\langle x_u \rangle$ to all
	\ENDIF
	\STATE receive incoming messages
	\IF{$phase_u$}
		\IF{a node voted for $v$}
			\STATE $x_u:= v$ \label{line:adopt_vote}
		\ELSE
			\STATE $x_u :=$ smallest  $w$ from  $\langle w , ? \rangle$ received \label{line:min_vote}
		\ENDIF
		\IF{every node voted for $v$, none sent null} \label{line:cond1}
			\STATE $DECIDE(v)$
		\ENDIF
		\STATE $vote_u :=\ ?$
	\ELSE
		\STATE $ x_u :=$ minimum value received (excluding null) \label{line:min_val}
		\IF{every node sent $v$, none sent null} \label{line:cond2}
			\STATE $vote_u := v$
		\ENDIF
	\ENDIF
	\STATE $phase_u := \neg phase_u$
\ENDINDENT

\end{minipage}
\end{tabular}

\caption{The {\em AdaptedUniformVoting} algorithm}
\label{unifvotfig}
\end{distribalgo}
\end{algorithm}

This pseudo-code is almost identical to the pseudo-code used in the systems with synchronous starts.
However, you can see that in lines \ref{line:cond1} and \ref{line:cond2}, the effect of the null messages is precised.
In that case, the null messages are not dropped. They prevent the nodes from voting and deciding.

It is possible to prove the following theorem:

\begin{thm}
	The $SynchMod_4 \lozenge AdaptedUniformVoting$ algorithm solves the consensus problem under the condition of a centered dynamic graph and a complete activation schedule.
\end{thm}

The proof of validity and agreement for UniformVoting given in \cite{CBS09} can easily be transposed to \newline $SynchMod_4 \lozenge AdaptedUniformVoting$.
The proof of termination, which is radically different, is detailed below.

\begin{proof}
	We consider an execution of $SynchMod_4 \lozenge AdaptedUniformVoting$.
	Let $r$ be the round in which $\cent$ fires.
	In round $r+1$, the $AdaptedUniformVoting$ algorithm is started by $\cent$.

	We claim that the sequence $(x_\cent(r+i+1))_{i \in \mathds{N}}$ is decreasing.
	\begin{enumerate}
		\item If $i+2$ is odd, the line \ref{line:min_val} guarantees that $x_\cent(r+i+2) \leq x_\cent(r+i+1)$.
		\item If $i+2$ is even and a node $u$ voted for $v$, the only possibility is $v = x_\cent(r+i+1)$. This is because $\cent$ sent $x_\cent(r+i+1)$ to $u$ in round $r+i+2$.
			Then $x_\cent(r+i+2) = x_\cent(r+i+1)$.
		\item If $i+2$ is even and no node voted, the line \ref{line:min_vote} guarantees that $x_\cent(r+i+2) \leq x_\cent(r+i+1)$.
	\end{enumerate}
	That proves the claim.

	Since this series is decreasing, and the set of initial decision values is finite, it must reach a minimum.
	We consider the earliest round $r_0 \geq \phi^{max}$ in which every node have already fired and the series has stabilized to $v$.

	Without loss of generality, we assume that the round $r_0+1$ is a round of exchange of value (i.e. $\forall u \in \Pi, \neg phase_u(r_0)$).
	Since $x_\cent(r_0+1) = x_\cent(r_0)$, we know that $\cent$ only receives messages containing $v$ in round $r_0+1$ (see line \ref{line:min_val}).
	Then $\cent$ must vote $v$ in round $r_0+2$ (see line \ref{line:cond2}).
	Then every node must adopt $v$ in round $r_0+2$ (see line \ref{line:adopt_vote}).
	Then, in round $r_0+3$ every node must send $v$ to all.
	Then, in round $r_0+3$ every node can only receive $v$.
	Then, in round $r_0+3$ every node must vote for $v$ (see line \ref{line:cond2}).
	Then, in round $r_0+4$ every node must decide $v$.
	That proves termination.
\end{proof}

\subsection{$\mathrm{mod}\,k$-synchronization and coordinated algorithms}

Some algorithms like Paxos rely on the existence of a shared coordinator in each round.
A simple implementation consists in setting a rotating coordinator: 
each node holds the list of nodes. In the first round, the chosen coordinator is the first node in the list.
In the $i^{th}$ round, the chosen coordinator is the $i~mod~k^{th}$ node on the list.
This works out-of-the box when the starts are assumed to be synchronous.
However, when the starts are asynchronous, a prior $\mathrm{mod}\,n$-synchronization is required, where $n$ is the number of nodes in the shared list.
This is another typical use case of the \SM algorithm.

\subsection{A much simpler approach ?}

In the case of the problem of consensus, we have mentioned in a previous paragraph the possibility to use Paxos.
Since we only consider synchronous systems, much simpler algorithms like FloodSet \cite{Lyn96} exists.
As mentioned in \cite{CDDS85}, the FloodSet algorithm can be adapted to solve consensus in a system with asynchronous starts.
Using a variant of the firing squad algorithm (called $AdaptedFiringSquad_f$ here),
where the nodes hold a counter $c_u \in \mathds{N}$, and fire when they reach an upper bound $f$ on the number of failure.
Using $AdaptedFiringSquad_f \lozenge FloodSet$, we get a simple consensus algorithm.
This approach has some shortcomings, though.
Firstly, the nodes have to "know" an shared upper bound on the number of failures.
Then, only the simplest form of failure is supported: the crash-failures.
More importantly, this method achieves \textit{non-uniform} consensus: only the non-crashed nodes are required to decide.

% section_conclusion (do not modify this comment)
\section{Conclusion and future work}

As any complex reasoning by cases, the correctness proof  of the \SM   algorithm, 
	and more specifically the proof of the liveness property, is very error prone. 
This is a typical example of the relevance of formal verification for distributed algorithms. 
Indeed, in a later work~\cite{}, we used the interactive theorem prover Isabelle \cite{Merz12} to encode the complete proof 
	of Theorem~\ref{thm:k>2}, and thus obtained a certificate for  \SM\!\!'s correctness when $k$ is greater than 2.
	
Since $\mathrm{mod}\,2$-synchronization is reducible to $\mathrm{mod}\,4$-synchronization,
	 our algorithm solves the $\mathrm{mod}\,k$-synchronization problem for any positive integer~$k$
	 in the class of  dynamic  graphs with a fixed center.
This class of dynamic graphs plays a crucial role regarding benign failures as it captures 
	the synchronous model with at most $n-1$ faulty senders, including the one with at most $n-1$ crashes.
In the wilder context of dynamic graphs, a natural question is whether the problem is still solvable 
	under weaker connectivity assumptions, in particular, in the class of dynamic graphs with a fixed root, 
	i.e., with a time-varying spanning tree at each round rooted at a fixed node.

% end_section (do not modify this comment)

\printbibliography

\end{document}
