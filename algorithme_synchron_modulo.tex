\documentclass{article}
\usepackage{amsmath}
\usepackage[noend]{sources_Heard-Of/distribalgo}
\usepackage{algorithm}
\usepackage{amssymb}
\usepackage{amsthm}
\usepackage{dsfont}
\usepackage{stmaryrd}
\usepackage[left=2cm, right=2cm, top=2cm]{geometry}
\usepackage[utf8]{inputenc}
\newtheorem{lemma}{Lemme}
\newtheorem{theorem}{Théorème}
\newtheorem{definition}{Définition}
\setlength{\parskip}{0.22cm}
%\newcommand{\st0}{q^0}


\title{Algorithme de synchronisation modulo k}
\date{1 juin 2020}
\author{Louis Penet de Monterno}

\begin{document}

\maketitle

\section{Problème}

Il est courant de rencontrer des algorithmes de consensus, structurés par phases : 
à chaque rounds pairs, les processus exécutent une certaine fonction de transition. Et au round impairs, une autre fonction de transition est exécutée.
Lorsque les départs sont synchrones, tous les processus exécutent la même fonction de transition à chaque round.
Lorsqu'il y a des départs asynchrones, les phases paires des différents processus ne sont plus simultanées, et cela peut compromettre la sûreté de l'algorithme.
Le but de ce document est de proposer un algorithme permettant de synchroniser les différentes phases malgré les départs asynchrones.

\begin{definition}

	Soit un paramètre $k > 1$. Soit $s_{exit} \in States_p$ un état final.
	Étant donnée une exécution d'un algorithme $A$, pour tout $p \in \Pi$, on note $t_p$ le round auquel $p$ atteint $s_{exit}$ ($t_p = \infty$ sinon).
	Cette exécution de l'algorithme $A$ est sûr vis à vis du problème de synchronisation modulo $k$ si,
	$$\exists c \in \mathds{N}, \forall p \in \bigcup\limits_{i \in \mathds{N}} \mathcal{A}_i, t_p \neq \infty \Rightarrow t_p~mod~k = c$$

\end{definition}

\begin{definition}

	Une exécution de l'algorithme $A$ vérifie la terminaison vis à vis du problème de synchronisation modulo $k$ si,
	$$\forall p \in \bigcup\limits_{i \in \mathds{N}} \mathcal{A}_i, t_p \neq \infty$$

\end{definition}
\begin{definition}

	L'algorithme $A$ résoud le problème de synchronisation modulo $n$, si toute exécution de $A$ est sûre et termine.

\end{definition}

\section{Algorithme}

L'objectif est de résoudre ce problème dans les conditions suivantes :
\begin{itemize}

	\item Le graphe dynamique contient une étoile fixe : $\mathcal{P}_{nek-fixed} \equiv \exists \xi \in \Pi, \forall p \in \Pi, \forall r \in \mathds{N}, \xi \in HO(p, r)$
	\item Tous les processus s'activent en temps fini : $\mathcal{P}_{non-inf} \equiv \exists r \in \mathds{N}, \mathcal{A}_r = \Pi$

\end{itemize}

\section{Version simplifiée}

\begin{algorithm}[htb]
\begin{distribalgo}[1]
\BLANK \INDENT{\textbf{Initialization:}}
  \STATE $x_p :=\, 0$ 

\ENDINDENT \BLANK

\INDENT{\textbf{Round $0$:}}
 \INDENT{$S_p:$}
    \STATE send $nil$ to all processes
  \ENDINDENT
  \BLANK
	\INDENT{$T_p(M):$}
	\STATE $x_p = min(M(\Pi) \setminus \{nil\})$ ~~~~\COMMENT{$x_p$ prend la valeur du plus petit message}
  \ENDINDENT
\ENDINDENT \BLANK
\INDENT{\textbf{Round $r+1$:}}
 \INDENT{$S_p:$}
    \STATE send $x_p+1~mod~n$ to all processes
  \ENDINDENT
  \BLANK
	\INDENT{$T_p(M):$}
	\STATE $x_p = min(M(\Pi) \setminus \{nil\})$ ~~~~\COMMENT{$x_p$ prend la valeur du plus petit message}
	\IF{$M(\Pi) = \{0\}$}
	\STATE $EXIT()$ ~~~~\COMMENT{si tous les messages reçus valent $0$, aucun $nil$, l'algorithme termine}
	\ENDIF
  \ENDINDENT
\ENDINDENT \BLANK


\caption{The {\em SyncMod} algorithm} \label{algo:R}
\end{distribalgo}

\end{algorithm}

\begin{lemma}
	Toute exécution de cet algorithme est sûr vis-à-vis du problème de synchronisation modulo k lorsque les prédicats $\mathcal{P}_{nek-fixed}$ et $\mathcal{P}_{non-inf}$ sont vérifiés.
\end{lemma}
\begin{proof}

	Soit $\xi$ le processes au centre de l'étoile.
	On suppose que $p$ termine au round $r$, et est le premier. Nécessairement, $\xi \in HO(p,r)$.
	La ligne 13 du code montre que la valeur envoyé par $\xi$ est $0$.
	Donc tous les processes ont reçus un $0$. Les lignes 7 et 12 assurent que tous les processes adoptent $0$.
	Ainsi, au round $r$, tous les processus ont la même valeur de compteur.
	Aux rounds suivants, on montre triviallement par récurrence que cette uniformité se conserve aux rounds suivants.

	\textbf{Note :} Cela est vrai même pour les processus qui s'activent après le round $r$.
	En effet, au moment de leur activation, les processus adoptent la valeur qu'ils reçoivent avant d'envoyer la leur.

	On montre également par récurrence qu'au round $r+i$, tous les processes ont $i~mod~k$ comme valeur de compteur.
	Donc ils ne peuvent terminer que lorsque $i~mod~k = 0$. Cela prouve la sûreté.

\end{proof}

On montre maintenant la terminaison dans un cas simple.
Soit $\mathcal{P}_{star} \equiv \exists \xi \in \Pi, \forall p \in \Pi, \forall r \in \mathds{N}, HO(p, r) = \{\xi\}$

\begin{lemma}
	Étant donnée une exécution de cet algorithme. On suppose que le graphe dynamique vérifie $\mathcal{P}_{nek-fixed}$.
	On suppose également que $\xi$ termine au round $r$.
	Alors tous les processus non indéfiniment passifs terminent également, lors des rounds ultérieurs à $r$.
\end{lemma}
\begin{proof}
	Si $\xi$ termine au round $r$, $\xi$ s'envoie nécessairement la valeur $0$ à lui-même. Donc $\xi$ envoie $0$ à tous les processus. 
	Donc tous les processes adoptent $0$.
	On suit un raisonnement similaire à celui de la preuve de la sûreté.
	on montre trivialement par récurrence qu'au round $r+i$, tous les processes ont $i~mod~k$ comme valeur de compteur.
	Donc lors des rounds $r+k+i$ vérifiant $i~mod~k = 0$, tous les processus qui n'ont pas déjà terminés terminent.
	Cela prouve la terminaison.
\end{proof}
\begin{theorem}
	Toute exécution de cet algorithme termine vis-à-vis du problème de synchronisation modulo k lorsque les prédicats $\mathcal{P}_{star}$ et $\mathcal{P}_{non-inf}$ sont vérifiés.
\end{theorem}
\begin{proof}
	Soit $r$ le round auquel $\xi$ s'active.
	On montre trivialement par récurrence que la valeur du compteur de $\xi$ au round $r+i$ est $i~mod~k$.
	Ainsi, $\xi$ recevra seulement au round $r+k$ sa propre valeur, à savoir $0$. Donc $\xi$ terminera au round $r+k$. Le lemme précédent achève donc la terminaison.
\end{proof}

\subsection{Version complétée}

Afin de prouver la terminaison plus simplement, on modifie l'algorithme pour se ramener à une situation où le graphe dynamique est fixe à partir d'un certain rang.
L'idée consiste à supposer toute liaison défaillante à partir du moment où un message y est perdu.
On garde donc en mémoire une variable $\Sigma$, conservant l'ensemble des processus étant entendus à travers une liaison sûre.
Si le grahe dynamique vérifie $\mathcal{P}_{nek-fixed}$, le centre de l'étoile $\xi$ est toujours fiable pour tous les processus. Donc $\forall p \in \Pi, \xi \in \Sigma_p(r)$.
À partir d'un certain rang, les ensembles $\Sigma_p$ se stabilisent. Le graphe dynamique est désormais fixe.

\begin{algorithm}[htb]
\begin{distribalgo}[1]
\BLANK \INDENT{\textbf{Initialization:}}
  \STATE $x_p :=\, 0$ 
  \STATE $\Sigma := \Pi$

\ENDINDENT \BLANK

\INDENT{\textbf{Round $0$:}}
 \INDENT{$S_p:$}
    \STATE send $nil$ to all processes
  \ENDINDENT
  \BLANK
	\INDENT{$T_p(M):$}
	\STATE $x_p = min(M(\Pi) \setminus \{nil\})$ ~~~~\COMMENT{$x_p$ prend la valeur du plus petit message}
	\STATE $\Sigma = M^{-1}(\Pi)$ ~~~~\COMMENT{$\Sigma$ devient l'ensemble des processus entendus}
  \ENDINDENT
\ENDINDENT \BLANK
\INDENT{\textbf{Round $r+1$:}}
 \INDENT{$S_p:$}
    \STATE send $x_p+1~mod~n$ to all processes
  \ENDINDENT
  \BLANK
	\INDENT{$T_p(M):$}
	\STATE $x_p = min(M(\Sigma) \setminus \{nil\})$ ~~~~\COMMENT{$x_p$ prend la valeur du plus petit message}
	\STATE $\Sigma = M^{-1}(\Sigma) \cap \Sigma$ ~~~~\COMMENT{les processus non entendus sont considérés fautifs, ils seront ignorés dans le futur}
	\IF{$M(\Pi) = \{0\}$}
	\STATE $EXIT()$ ~~~~\COMMENT{si tous les messages reçus valent $0$, aucun $nil$, l'algorithme termine}
	\ENDIF
  \ENDINDENT
\ENDINDENT \BLANK


\caption{The {\em SyncMod} algorithm} \label{algo:R}
\end{distribalgo}

\end{algorithm}

Exécuter cet algorithme sur un graphe dynamique $\mathds{G}$ revient à exécuter l'algorithme simplifié sur le graphe $\mathds{G}'$, où $\mathds{G}'$ est la version épurée de $\mathds{G}$,
c'est-à-dire la version de $\mathds{G}$ où les arêtes défaillantes au round $r$ sont éliminées aux rounds $r' > r$.
Si $\mathds{G}$ vérifie $\mathcal{P}_{nek-fixed}$, le graphe $\mathds{G}'$ ainsi épuré vérifie également $\mathcal{P}_{nek-fixed}$. Cela permet de montrer les lemmes suivants,
qui sont les simples transpositions des lemmes de sûreté de l'algorithme simplifié.

\begin{lemma}
	Toute exécution de cet algorithme est sûr vis-à-vis du problème de synchronisation modulo k lorsque les prédicats $\mathcal{P}_{nek-fixed}$ et $\mathcal{P}_{non-inf}$ sont vérifiés.
\end{lemma}
\begin{lemma}
	Étant donnée une exécution de cet algorithme. On suppose que le graphe dynamique vérifie $\mathcal{P}_{nek-fixed}$.
	On suppose également que $\xi$ termine au round $r$.
	Alors tous les processus non indéfiniment passifs terminent également, lors des rounds ultérieurs à $r$.
\end{lemma}

Pour montrer la terminaison de l'algorithme amélioré dans le cas $\mathcal{P}_{nek-fixed}$, il suffit de montrer la terminaison lorsque le graphe est fixe.
On peut également ignorer les noeuds à partir desquels $\xi$ n'est pas atteignable. Il faut donc montrer que l'algorithme termine lorsque le graphe est fixe, fortement connexe, et contient une étoile.

Une simulation montre qu'avec l'algorithme amélioré, la convergence su réseau vers une valeur de compteur uniforme est très rapide.
Ci-dessous, la distribution du temps de convergence de 20000 scénarios pour $k = 10$, dans un réseau de $50$ noeuds, 

\begin{table}[ht]
	\begin{center}
	\caption{Distribution du temps de convergence}
	\label{tab:table1}
	\begin{tabular}{c|c} % <-- Alignments: 1st column left, 2nd middle and 3rd right, with vertical lines in between
		\textbf{temps de convergence} & \textbf{nombre d'occurrences}\\
		\hline
		0 & 0\\
		1 & 1859\\
		2 & 16983\\
		3 & 1157\\
		4 & 1\\
		5 & 0\\
		\end{tabular}
	\end{center}
\end{table}
\end{document}
