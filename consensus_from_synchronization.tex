\documentclass[11pt,letterpaper]{article}
\special{papersize=8.5in,11in}

\usepackage[utf8]{inputenc}
\usepackage[ruled, linesnumbered, noline]{algorithm2e}
\usepackage[pdftex,pdfpagelabels,bookmarks,hyperindex,hyperfigures]{hyperref}
\usepackage{amsmath,amssymb,amsthm,dsfont}
\usepackage{fullpage}
\newtheorem{thm}{Theorem}
\newtheorem{pro}[thm]{Proposition}
\newtheorem{lem}[thm]{Lemma}
\newtheorem{cor}[thm]{Corollary}
\newtheorem{defi}[thm]{Definition}

\title{Reduction of auto-stabilizing synchronization problem to consensus problem}
\author{
	Bernadette Charron-Bost \\
	LIX, Palaiseau, France
\and
	Stephan Merz \\
	LORIA, Nancy, France
\and
	Louis Penet de Monterno \\
	LIX, Palaiseau, France
}
\date{\today}

\begin{document}
  \maketitle

\section{Auto-stabilizing consensus problem}

In this document, we consider a network class $\mathcal{C}$.
Since we need to have consensus in $\mathds{Z}$, we choose a set of colours $C = \mathds{Z}$.

We assume the existence of a consensus protocol $A$.
This protocol is caracterized by:
\begin{itemize}
	\item a set of states $X$,
	\item a transition function $\delta^c : X \times X^\oplus \rightarrow X$. Here, the colour $c$ of the nodes influences their transition function.
		$X^\oplus$ is the set of muti-sets of elements of $X$.
	\item an output function $Out : X \rightarrow \mathds{Z}$.
\end{itemize}

The way the transition function is expressed implies that processes are able to distinguish their state in the previous round from the state of their incoming neighbours.
This protocol $A$ is assumed to self-stabilize to the following predicate:
$$Cons = \{\dots \langle -2,-2,\dots \rangle, \langle -1,-1,\dots \rangle, \langle 0,0,\dots \rangle, \langle 1,1,\dots \rangle, \dots\}$$


\section{Translation function}

Let $Trans_h : X \rightarrow X$ be a function parametered by $h \in \mathds{Z}$.
For any $\mathcal{X} \in X^\oplus$, we note $Trans_h(\mathcal{X})$ the application of $Trans_h$ to every elements of $\mathcal{X}$.
$Trans_h$ is said to be a \textit{translation function} of the algorithm $A$ if, for any $h, h' \in \mathds{Z}$, for any $x \in X$, the following conditions apply:
\begin{enumerate}
	\item $Trans_0 = Id_{X \rightarrow \mathds{Z}}$
	\item $Trans_h \circ Trans_{h'} = Trans_{h+h'}$
	\item $\forall \mathcal{X} \in X^\oplus, Trans_h(\delta_{h'}(x, \mathcal{X})) = \delta_{h'+h}(Trans_h(x), Trans_h(\mathcal{X}))$
	\item $Out(Trans_h(x)) = h+Out(x)$
\end{enumerate}

\noindent Intuitively, these properties are simple neutral-element, transitivity and commutativity properties.
We assume that this function is provided.

\section{Construction of synchronization protocol}

We define the protocol $Sync$ with the following pseudo-code, where $y_i$ is the output variable:

\begin{algorithm}[H]
	\DontPrintSemicolon
	\textbf{State space:} \;
	\Indp
		$M_j \in \mathds{Z}$ \;
		$x_j \in X$ \;
		$y_j \in \mathds{Z}$ \;
	\BlankLine
	\Indm
	\textbf{At each round:} \;
	\Indp
		$M_j \leftarrow 1+M_j$ \;
		$x_j \leftarrow \delta_0(x_j, \{Trans_{M_l - M_j}(x_l), l \looparrowright i\})$ \;
		$y_j \leftarrow M_j+Out(x_j)$ \;
	\Indm
\caption{The $Sync$ algorithm} 
\end{algorithm}

\section{Translation lemma}

We consider an execution of $Sync$.
This execution is perfectly caracterized by the initial global state $((M_j^0, x_j^0))_{j \in \Pi}$.
We fix a node $i \in \Pi$.
Using this execution, we construct a new protocol $Cons(i)$ in which the pseudo-code of a node $j \in \Pi$ is:

\begin{algorithm}[H]
	\DontPrintSemicolon
	\textbf{State space:} \;
	\Indp
		$\bar x_j \in X$ \;
		$\bar y_j \in \mathds{Z}$ \;
	\BlankLine
	\Indm
	\textbf{At each round:} \;
	\Indp
		$\bar x_j \leftarrow \delta_{M_j^0 - M_i^0}(\bar x_j, \{\bar x_l, l \looparrowright j\})$ \;
		$\bar y_j \leftarrow Out(\bar x_j)$ \;
	\Indm
	\caption{The $Cons(i)$ algorithm} 
\end{algorithm}

We construct an execution $(\bar x_j^t)_{j \in \Pi, t \in \mathds{N}}$ of this algorithm by fixing the communication graph to $\mathds{G}$,
and fixing the initial global state to $\bar x_j^0 = Trans_{M_j^0 - M_i^0}(x_j^0)$ for every $j \in \Pi$.

\begin{lem}
	$\forall t \in \mathds{N}, \forall j \in \Pi, \bar{x}^t_j = Trans_{M_j^0 - M_i^0}(x_j^t)$
\end{lem}
\begin{proof}
	We show this lemma by induction on $t$.
	The base case results from the choice of initial state of $Cons(i)$.
	For the inductive case, we consider a node $j \in \Pi$.
	We have: 

	\begin{align*}
		\bar x_j^{t+1} &= \delta_{M^0_j-M^0_i}(\bar x_j^t, \{\bar x_l^t, l \looparrowright j\}) \\
		&= Trans_{M^0_j-M^0_i}(\delta_0(Trans_{M^0_i-M^0_j}(\bar x_j^t), \{Trans_{M^0_i-M^0_j}(\bar x_l^t), l \looparrowright j\})) \\
		&= Trans_{M^0_j-M^0_i}(\delta_0(x_j^t, \{Trans_{M^0_l-M^0_j}(x_l^t), l \looparrowright j\})) \\
		&= Trans_{M^0_j-M^0_i}(x_j^{t+1})
	\end{align*}

	\noindent Here we use the pseudo-code of $Cons(i)$, then the properties of $Trans$, then the induction hypothesis, and finally the pseudo-code of $Sync$.
\end{proof}

\section{Reduction of synchronization to consensus}

\begin{thm}
	$Sync$ stabilizes to the behaviour of a synchronized clock.
\end{thm}
\begin{proof}
	$Cons(i)$ is an instance of $A$ where the initial value of $j \in \Pi$ is $M_j^0-M_i^0$.
	Then there exists a round $t$ from which $Out(\bar x^t_j)$ has stabilized to some value $v$.
	For all $t' \geq t$, for any $j \in \Pi$, we have:

	\begin{align*}
		y_j^{t'} &= M_j^{t'} + Out(x_j^{t'}) \\
		&= M_j^{t'} + Out(Trans_{M_i^0-M_j^0}(\bar x_j^{t'})) \\
		&= t' + M_j^0 + Out(Trans_{M_i^0-M_j^0}(\bar x_j^{t'}))  \\
		&= t' + M_j^0 + M_i^0-M_j^0 + Out(\bar x_j^{t'}) \\
		&= t' + M_i^0 + v
	\end{align*}

	\noindent Since $M_i^0+v$ is a constant, the system is synchronized after round $t$.
\end{proof}

\end{document}
