\documentclass[11pt,letterpaper]{article}
\special{papersize=8.5in,11in}

\usepackage[utf8]{inputenc}
\usepackage[pdftex,pdfpagelabels,bookmarks,hyperindex,hyperfigures]{hyperref}
\usepackage{amsmath,amssymb,amsthm,dsfont}
\usepackage{fullpage}
\newtheorem{thm}{Theorem}
\newtheorem{pro}[thm]{Proposition}
\newtheorem{lem}[thm]{Lemma}
\newtheorem{cor}[thm]{Corollary}
\newtheorem{defi}[thm]{Definition}

\title{Reduction of auto-stabilizing synchronization problem to consensus problem}
\author{
	Bernadette Charron-Bost \\
	LIX, Palaiseau, France
\and
	Stephan Merz \\
	LORIA, Nancy, France
\and
	Louis Penet de Monterno \\
	LIX, Palaiseau, France
}
\date{\today}

\begin{document}
  \maketitle

\section{Auto-stabilizing consensus problem}

In this document, we consider a network class $\mathcal{C}$.
Since we need to have consensus in $\mathds{Z}$, we choose a set of colours $C = \mathds{Z}$.

We assume the existence of a consensus protocol $A$.
This protocol is caracterized by:
\begin{itemize}
	\item a set of states $X$,
	\item a transition function $\delta^c : X \times X^\oplus \rightarrow X$. Here, the colour $c$ of the nodes influences their transition function.
	\item an output function $Out : X \rightarrow \mathds{Z}$.
\end{itemize}

This protocol $A$ is assumed to self-stabilize to the following predicate:
$$Cons = \{\dots \langle -2,-2,\dots \rangle, \langle -1,-1,\dots \rangle, \langle 0,0,\dots \rangle, \langle 1,1,\dots \rangle, \dots\}$$

\section{Translation to full-information protocol}

We consider an arbitrary initial state $x_{init} \in X$.
We use the simple view-construction defined in BV to define the protocol $A_{FI}$, whose state space is the set of colored finite trees:

$$\delta^c_{FI}(T_i, T_{i(1)}, T_{i(2)}, \dots) = Node_c(T_i, T_{i(1)}, T_{i(2)}, \dots)$$

where $i(1), i(2), \dots$ are the incoming neighbours of the node $i$, and $Node_{c_i}$ is the constructor of a tree node coloured by $c$.

The output function of this protocol is defined by:

$$Eval(T_i) = \left \{ \begin{array}{l ll}
	\delta^c(Eval(T_i), Eval(T_{i(1)}), Eval(T_{i(2)}), \dots) & \mbox{if}~T_i = Node_c(T_i, T_{i(1)}, T_{i(2)}, \dots) \\
	x_{init} & \mbox{otherwise} \end{array} \right $$

and

	$$Out_{FI} = Out \circ Eval.$$

Here, the $Eval$ function computes the state of every incoming neighbours, applying recursively $\delta$, and $Out_{FI}$ deduces the output from the latest state.

\begin{lem}
	The protocol $A_{FI}$ auto-stabilizes to $Cons$.
\end{lem}
\begin{proof}
	Let $G$ be a network from $\mathcal{C}$.
	We consider $(T_i^t)_{t \in \mathds{N}, i \in \Pi}$ an execution of $A_{FI}$ in $G$.
	Let $x_i^0 = Eval(T^0_i)$ for every $i \in \Pi$.
	We can construct an execution of $A$ using $(x_i^0)_{i \in \Pi}$ as initial values. Now we can show by induction over $t \in \mathds{N}$ that:

	$$\forall t \in \mathds{N}, \forall i \in \Pi, x_i^t = Eval(T_i^t).$$

	We know that $(x^t)_{t \in \mathds{N}}$ is supposed to stabilize to some behaviour in $Cons$.
	Thus, any execution of $A_{FI}$ converges into $Cons$.	
\end{proof}

We recursively define the following translation function, which translates every initial values (i.e. every colours) by some integer $h \in \mathds{Z}$:

$$Trans_h(T_i) = \left \{ \begin{array}{l ll}
	Node_{c+h}(Trans_h(T_i), Trans_h(T_{i(1)}), \dots) & \mbox{if}~T = Node_c(T_i, T_{i(1)}, T_{i(2)}, \dots) \\
	Leaf & \mbox{otherwise} \end{array} \right $$


The protocol $A_{FI}$ is said to be \textit{invariant by translation} if $Out_{FI}(Trans_h(T_i)) = Out_{FI}(T_i)+h$.
In practice, most consensus protocols are invariant by translation.

\section{Construction of synchronization protocol}

Let $M : Tree \rightarrow \mathds{N}$ be the following function:

$$M(T) = \left \{ \begin{array}{l ll}
	1+M(T_0)& \mbox{if}~T = Node_c(T_0, T_1, T_2, \dots) \\
	0 & \mbox{otherwise} \end{array} \right $$


We define the protocol $Sync$, whose state space is $\mathds{Z}$, with the following transition function:

$$\delta_{Sync}(T_i, T_{i(1)}, T_{i(2)}, \dots) = Node_0(T_i, Trans_{M(T_i)-M(T_{i(1)})}(T_{i(1)}), Trans_{M(T_i)-M(T_{i(2)})}(T_{i(2)}), \dots)$$

and

$$Out_{Sync}(T) = M(T) + Out_{FI}(T).$$

\begin{lem}
	For any execution of $Sync$, we have $\forall i \in \Pi, \forall t \in \mathds{N}, M(T_i^t) = M(T_i^0)+t$.
\end{lem}

\section{Translation lemma}

We consider an execution of $Sync$.
This execution is perfectly caracterized by the initial state $(T_i^0)_{i \in \Pi}$.
Using this execution, we construct a new protocol $Cons(i)$ in which the transition function of a node $j \in \Pi$ is:

$$\delta_{Cons(i)}(T_i, T_{i(1)}, T_{i(2)}, \dots) = Node_{M(T_j^0)-M(T_i^0)}(T_i, T_{i(1)}, T_{i(2)}, \dots)$$

\noindent We also set $Out_{Cons(i)} = Out_{FI}$.
We construct $(\bar T_i^t)_{t \in \mathds{N}, i \in \Pi}$ an exection of this protocol, by choosing the following initial values:

$$\bar{T}_j^0 = Trans_{M(T_j^0) - M(T_i^0)}(T_j^0).$$

\begin{lem}
	$\forall t \in \mathds{N}, \forall j \in \Pi, \bar{T}^t_j = Trans_{M(T_j^0) - M(T_i^0)}(T_j^t)$
\end{lem}
\begin{proof}
	We show this lemma by induction on $t$.
	The base case results from the definition of $Cons(i)$.
	For the inductive case, we consider a node $j \in \Pi$ and $j(1), j(2), \dots$ its incoming neighbours.
	We have (using the definition of $\delta_{Cons(i)}$, then the induction hypothesis, and finally $\delta_{Sync}$):

	\begin{align*}
		\bar T_j^{t+1} &= Node_{M(T^0_j)-M(T^0_i)}(\bar T_j, \bar T_{j(1)}, \bar T_{j(2)}, \dots) \\
		&= Trans_{M(T^0_j)-M(T^0_i)}(Node_0(Trans_{M(T^0_i)-M(T^0_j)}(\bar T_j), Trans_{M(T^0_i)-M(T^0_{j(1)})}(\bar T_{j(1)}),  \dots)) \\
		&= Trans_{M(T^0_j)-M(T^0_i)}(Node_0(T_j, Trans_{M(T^0_j)-M(T^0_{j(1)})}(T_{j(1)}), \dots)) \\
		&= Trans_{M(T^0_j)-M(T^0_i)}(T_j^{t+1})
	\end{align*}
\end{proof}

\section{Reduction of synchronization to consensus}

\begin{lem}
	$Cons(i)$ stabilizes to $Cons$.
\end{lem}
\begin{proof}
	$Cons(i)$ is an instance of $A_{FI}$ where the initial value of $j \in \Pi$ is $M_j^0-M_i^0$.
\end{proof}
\begin{thm}
	If $A_{FI}$ is invariant by translation, then $Sync$ stabilizes to the behaviour of a synchronized clock.
\end{thm}
\begin{proof}
	There exists a round $t$ in which the execution of $Cons(i)$ has stabilized to some value $v$.
	For all $t' \geq t$, for any $j \in \Pi$, we have:
	$$Out_{Sync}(T_j^{t'}) = M(T_j^{t'}) + Out_{FI}(T_j^{t'}) = M(T_j^{t'}) + Out_{FI}(Trans_{M(T_i^0)-M(T_j^0)}(\bar T_j^{t'})).$$

	\noindent Since $A_{FI}$ is invariant by translation, we have:
	$$Out_{FI}(Trans_{M(T_i^0)-M(T_j^0)}(\bar T_j^{t'})) = M(T_i^0)-M(T_j^0) + Out_{FI}(\bar T_j^{t'}) = M(T_i^0)-M(T_j^0) + v.$$

	\noindent We obtain:
	$$Out_{Sync}(T_j^{t'}) = M(T_j^{t'}) + M(T_i^0)-M(T_j^0) + v = t' + M(T_i^0) + v.$$
\end{proof}

\end{document}
